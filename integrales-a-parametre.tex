\documentclass{yann}

\newcommand\DerPart[2]{\frac{∂#1}{∂#2}}
\newcommand\I{i}

\begin{document}
\title{Intégrales à paramètre}
\maketitle

\Para{Contexte}

Soit $I$ et $J$ deux intervalles (non vides) de $ℝ$ et
\[ \Fonction{f}{I×J}𝕂{(x,t)}{f(x,t).} \]
On s'intéresse à la fonction $F$ définie sur $I$ par
\[ F(x) = ∫_J f(x,t) \D t. \]

% -----------------------------------------------------------------------------
\section{Continuité}

\Para{Théorème}[continuité sous le signe $∫$]

On suppose:
\begin{enumerate}
\item
  $f$ est continue par rapport à $x$, c.-à-d.

  \begin{itemize}
  \item
    pour tout $t∈J$ fixé, la fonction $x \mapsto f(x,t)$ est continue sur $I$.
  \end{itemize}
\item
  $f$ est continue par morceaux par rapport à $t$, c.-à-d.

  \begin{itemize}
  \item
    pour tout $x∈I$ fixé, la fonction $t \mapsto f(x,t)$ est continue par morceaux sur $J$.
  \end{itemize}
\item
  \emph{Hypothèse de domination:}
  Il existe une fonction $\Fnφ{J}{\Rp}$ telle que

  \begin{itemize}
  \item
    pour tout $(x,t)∈I×J$, on a $\Abs{f(x,t)}≤φ(t)$,
  \item
    $φ$ est continue par morceaux et intégrable sur $J$.
  \end{itemize}
\end{enumerate}

Alors:
\begin{itemize}
\item
  La fonction $F \colon x \mapsto ∫_J f(x,t) \D t$ est définie et continue sur $I$.
\end{itemize}

\Para{Remarque}[version locale]

On peut remplacer l'hypothèse de domination par:
\begin{itemize}
\item
  \emph{Hypothèse de domination locale:}
  Pour tout segment $K$ inclus dans $I$,
  il existe une fonction $φ_K \colon J \to \Rp$ telle que

  \begin{itemize}
  \item
    pour tout $(x,t)∈K×J$, on a $\Abs{f(x,t)}≤φ_K(t)$,
  \item
    $φ_K$ est continue par morceaux et intégrable sur $J$.
  \end{itemize}
\end{itemize}

% -----------------------------------------------------------------------------
\section{Dérivabilité}

\Para{Théorème}[dérivation sous le signe $∫$]

On suppose:
\begin{enumerate}
\item
  $f$ est de classe $\CC1$ par rapport à $x$, c.-à-d.

  \begin{itemize}
  \item
    pour tout $t∈J$ fixé, la fonction $x \mapsto f(x,t)$
    est de classe $\CC1$ sur $I$.
  \end{itemize}
\item
  $f$ est continue par morceaux et intégrable par rapport à $t$, c.-à-d.

  \begin{itemize}
  \item
    pour tout $x∈I$ fixé, la fonction $t \mapsto f(x,t)$
    est continue par morceaux et intégrable sur $J$.
  \end{itemize}
\item
  $\DerPart fx$ est continue par morceaux par rapport à $t$, c.-à-d.

  \begin{itemize}
  \item
    pour tout $x∈I$ fixé, la fonction $t \mapsto \DerPart fx$
    est continues par morceaux sur $J$.
  \end{itemize}
\item
  \emph{Hypothèse de domination:}
  Il existe une fonction $φ\colon J \to \Rp$ telle que

  \begin{itemize}
  \item
    pour tout $(x,t)∈I×J$, on a $\left| \DerPart fx(x,t) \right| ≤ φ(t)$,
  \item
    $φ$ est continue par morceaux et intégrable sur $J$.
  \end{itemize}
\end{enumerate}

Alors:
\begin{itemize}
\item
  La fonction $F \colon x \mapsto ∫_J f(x,t) \D t$ est définie et de classe $\CC1$ sur $I$.
\item
  Pour tout $x∈I$, on a \[ F'(x) = ∫_J \DerPart fx(x,t) \D t. \]
\end{itemize}

\Para{Remarque}[version locale]

On peut remplacer l'hypothèse de domination par:
\begin{itemize}
\item
  \emph{Hypothèse de domination locale:}
  Pour tout segment $K$ inclus dans $I$,
  il existe une fonction $φ_K \colon J \to \Rp$ telle que

  \begin{itemize}
  \item
    pour tout $(x,t)∈K×J$, on a $\left| \DerPart fx(x,t) \right| ≤ φ_K(t)$,
  \item
    $φ_K$ est continue par morceaux et intégrable sur $J$.
  \end{itemize}
\end{itemize}

\Para{Théorème}[extension aux fonctions de classe $\CC p$]

On suppose:
\begin{enumerate}
\item
  $f$ est de classe $\CC p$ par rapport à $x$, c.-à-d.

  \begin{itemize}
  \item
    pour tout $t∈J$ fixé, la fonction $x \mapsto f(x,t)$ est de classe $\CC p$ sur $I$.
  \end{itemize}
\item
  pour tout $0≤k<p$,
  $\frac{∂^k f}{∂x^k}$ est continue par morceaux
  et intégrable par rapport à $t$, c.-à-d.

  \begin{itemize}
  \item
    pour tout $k∈\Dcro{0,p-1}$ et pour tout $x∈I$ fixés,
    la fonction $t \mapsto \frac{∂^k f}{∂x^k} (x,t)$
    est continue par morceaux et intégrable sur $J$.
  \end{itemize}
\item
  $\frac{∂^p f}{∂x^p}$ est continue par morceaux par rapport à $t$, c.-à-d.

  \begin{itemize}
  \item
    pour tout $x∈I$ fixé, la fonction $t \mapsto \frac{∂^p f}{∂x^p} (x,t)$
    est continue par morceaux sur $J$.
  \end{itemize}
\item
  \emph{Hypothèse de domination:}
  Il existe une fonction $φ\colon J \to \Rp$ telle que

  \begin{itemize}
  \item
    pour tout $(x,t)∈I×J$, on a $\left| \frac{∂^p f}{∂x^p} (x,t) \right| ≤ φ(t)$,
  \item
    $φ$ est continue par morceaux et intégrable sur $J$.
  \end{itemize}
\end{enumerate}

Alors:
\begin{itemize}
\item
  La fonction $F \colon x \mapsto ∫_J f(x,t) \D t$ est définie et de classe $\CC p$ sur $I$.
\item
  Pour tout $k∈\Dcro{0,p}$, pour tout $x∈I$,
  on a \[ F^{(k)}(x) = ∫_J \frac{∂^k f}{∂x^k}(x,t) \D t. \]
\end{itemize}

\Para{Remarque}[version locale]

On peut remplacer l'hypothèse de domination par:
\begin{itemize}
\item
  \emph{Hypothèse de domination locale:}
  Pour tout segment $K$ inclus dans $I$,
  il existe une fonction $φ_K \colon J \to \Rp$ telle que

  \begin{itemize}
  \item
    pour tout $(x,t)∈K×J$,
    on a $\left| \frac{∂^p f}{∂x^p} (x,t) \right| ≤ φ_K(t)$,
  \item
    $φ_K$ est continue par morceaux et intégrable sur $J$.
  \end{itemize}
\end{itemize}

% -----------------------------------------------------------------------------
\section{Exercices}

\Exercice

On considère la fonction définie par \[ f(x) = ∫_0^1 e^{-x/t} \D t. \]
\begin{enumerate}
\item
  Déterminer l'ensemble de définition de $f$.
\item
  Montrer que $f$ est continue sur son ensemble de définition.
\item
  Montrer que $f$ est de classe $\CC1$ sur $\Rps$.
\item
  Montrer que $f$ est de classe $\CC∞$ sur $\Rps$.
\item
  Vérifier que $∀x > 0$, \[ f''(x) = \frac{e^{-x}}{x}. \]
\end{enumerate}

\Exercice

On cherche à calculer l'intégrale de Gauß
\[ ∫_0^{+∞} e^{-x^2} \D x. \]
Pour cela, on considère les fonctions définies par
\[ f(x) = \left(∫_0^x e^{-t^2} \D t \right)^2
  \quad\text{et}\quad
g(x) = ∫_0^1 \frac{e^{-x^2(1+t^2)}}{1+t^2} \D t. \]
\begin{enumerate}
\item
  Montrer que $f$ et $g$ sont définies et de classe $\CC1$ sur $ℝ$.
\item
  Calculer $f'$ et $g'$.
\item
  En déduire que $∀x∈ℝ$, \[ f(x) + g(x) = \fracπ4. \]
\item
  En déduire la valeur de l'intégrale de Gauß.
\end{enumerate}

\Exercice[théorème de division des fonctions $\CC∞$]

Soit $\Fn{f}ℝℝ$ de classe $\CC∞$ et
\[ g(x) = \begin{cases}
    \dfrac{f(x)-f(0)}{x} & \text{si } x≠0 \\
    \hfil f'(0)               & \text{si } x=0.
\end{cases} \]
\begin{enumerate}
\item
  Vérifier que $g(x) = ∫_0^1 f'(tx) \D t$.
\item
  En déduire que $g$ est de classe $\CC∞$.
\item
  \emph{Généralisation.}
  Montrer que la fonction
  \[ g_n \colon x \mapsto \frac{1}{x^n}
  \left( f(x) - ∑_{k=0}^{n-1} \frac{x^k}{k!} f^{(k)}(0) \right) \]
  se prolonge en une fonction de classe $\CC∞$ en $0$.
\end{enumerate}

\Exercice

Montrer qu'il existe un unique réel $x∈[0,π]$ tel que
\[ ∫_0^π \cos(x\sinθ) \Dθ = 0. \]
et calculer une valeur approchée de $x$ à $10^{-2}$ près.

\emph{Indication:} Montrer que
\[ ∫_0^π \cos(x\sinθ) \Dθ = 2∫_0^1 \frac{\cos(xt)}{√{1-t^2}} \D t. \]

\Exercice[la fonction gamma]

Soit \[ Γ(x) = ∫_0^{+∞} t^{x-1} e^{-t} \D t. \]
\begin{enumerate}
\item
  Montrer que $Γ$ est de classe $\CC∞$ sur $\Rps$.
\item
  Montrer que $Γ$ est convexe.
\item
  Montrer que $\ln◦Γ$ est convexe.
\item
  Montrer que $∀x∈\Rps$, $Γ(x+1) = xΓ(x)$.
\item
  Montrer que $∀n∈ℕ$, $Γ(n+1) = n!$.
\item
  Effectuer le changement de variables $t = n + y√n$
  dans l'intégrale $Γ(n+1)$,
  pour obtenir une expression de la forme
  \[ Γ(n+1) = n^n e^{-n} √n ∫_{-∞}^{+∞} f_n(y) \D y. \]
\item
  Soit $φ(y) = \begin{cases}
      e^{-y^2/2}  & \text{si } y < 0 \\
      (1+y)e^{-y} & \text{si } y≥0.
  \end{cases}$
  Vérifier que $∀n∈\Ns$, $∀y∈ℝ$, $\Abs{f_n(y)}≤φ(y)$.
\item
  En déduire la \emph{formule de Stirling}
  \[ n! \Sim_{n\to+∞} \left( \frac{n}{e} \right)^n √{2πn}. \]
\end{enumerate}

\Exercice

Soit $\Fn{f}{\Rp}ℝ$ continue telle que
$∫_0^{+∞} f$ converge absolument.
On pose \[ φ(a) = ∫_0^{+∞} e^{-at}f(t) \D t. \]
\begin{enumerate}
\item
  Montrer que $φ$ est définie sur $\Rp$.
\item
  Montrer que $φ$ est continue sur $\Rp$.
\item
  Montrer que $φ$ est de classe $\CC∞$ sur $\Rps$.
\end{enumerate}

\Exercice[le même en nettement plus dur]

Soit $\Fn{f}{\Rp}ℝ$ continue telle que
$∫_0^{+∞} f$ converge (pas nécessairement absolument).
On pose \[ φ(a) = ∫_0^{+∞} e^{-at}f(t) \D t \quad\text{et}\quad
F(x) = ∫_x^{+∞} f(t) \D t. \]
\begin{enumerate}
\item
  Montrer que $F$ est définie, continue et bornée sur $\Rp$.
\item
  Montrer que $φ$ est définie sur $\Rp$ et que $∀a∈\Rp$,
  \[ φ(a) = F(0) - a∫_0^{+∞} e^{-at} F(t) \D t. \]
\item
  Montrer que $φ$ est de classe $\CC∞$ sur $\Rps$.
\item
  Montrer que $φ$ est continue en $0$.
\end{enumerate}

\Exercice

On considère $f(x) = ∫_0^{+∞} \frac{\D t}{t^x (1+t)}$.
\begin{enumerate}
\item
  Domaine de définition, monotonie, convexité de $f$ (sans dériver $f$).
\item
  Continuité, dérivabilité, calcul de $f^{(k)}$.
\item
  Donner un équivalent de $f(x)$ en $0$ et en $1$.
\item
  Calculer $f(\frac1n)$ pour $n∈ℕ$, $n≥2$.
\end{enumerate}

\Exercice

Soit $f(x) = ∫_0^{+∞} \frac{e^{-tx}}{1+t^2} \D t$.
\begin{enumerate}
\item
  Montrer que $f$ est définie et continue sur $ℝ$.
\item
  Montrer que $f$ est de classe $\CC∞$ sur $\Rps$.
\item
  Montrer que $∀x > 0$, $f(x) + f''(x) = \frac{1}{x}$.
\end{enumerate}

\Exercice

Soit $f(x) = ∫_0^{+∞} \frac{\D t}{1+x^3+t^3}$.
\begin{enumerate}
\item
  Montrer que $f$ est définie sur $\Rp$.
\item
  À l'aide du changement de variable $u = \frac1t$, calculer $f(0)$.
\item
  Montrer que $f$ est continue et décroissante.
\item
  Déterminer $\lim\limits_{+∞} f$.
\end{enumerate}

\Exercice

Soit $f(x) = ∫_0^{π/2} \sin^x(t) \D t$.
\begin{enumerate}
\item
  Montrer que $f$ est définie et positive sur $\intO{-1,+∞}$.
\item
  Montrer que $f$ est de classe $\CC1$ et préciser sa monotonie.
\item
  Former une relation entre $f(x)$ et $f(x+2)$ pour tout $x > -1$.
\item
  On pose pour $x > 0$, $φ(x) = xf(x)f(x-1)$.
  Montrer que
  \[ ∀x > 0, \quadφ(x+1) =φ(x). \]
\item
  Déterminer un équivalent de $f$ en $-1^+$.
\end{enumerate}

\Exercice

On considère les fonctions $f$ et $g$ définies sur $\Rp$ par
\[ f(x) = ∫_0^{+∞} \frac{e^{-xt}}{1+t^2} \D t
  \quad\text{et}\quad
g(x) = ∫_0^{+∞} \frac{\sin t}{x+t} \D t. \]
\begin{enumerate}
\item
  Monter que $f$ et $g$ sont de classe $\CC2$ sur $\Rps$ et
  qu'elles sont solutions de l'équation différentielle
  $y'' + y = \frac1x$.
\item
  Montrer que $f$ et $g$ sont continues en $0$.
\item
  En déduire que \[ ∫_0^{+∞} \frac{\sin t}{t} \D t = \fracπ2. \]
\end{enumerate}

\Exercice

Existence et calcul de
\[ φ(x) = ∫_0^{+∞} e^{-t^2} \cos(xt) \D t. \]
On pourra montrer que $φ$ est solution d'une équation différentielle
que l'on résoudra.

\Exercice

Soit $f$ définie sur $\Rp$ par \[ f(x) = ∫_0^{+∞} \frac{\arctan(xt)}{1+t^2} \D t. \]
\begin{enumerate}
\item
  Étudier la continuité et la dérivabilité de $f$.
\item
  Pour $x > 0$, calculer $f'(x)$.
\item
  En déduire que \[ ∫_0^1 \frac{\ln t}{t^2-1} \D t = \frac{π^2}{8}. \]
\item
  Bonus: montrer que $f$ n'est pas dérivable en $0$.
  Pour cela, on pourra minorer $f(x)/x$.
\end{enumerate}

\Exercice

Soit \[ f(x) = ∫_0^{+∞} \frac{\cos(xt)}{1+t^2} \D t. \]
\begin{enumerate}
\item
  Étudier la continuité et la dérivabilité de $f$.

  \begin{enumerate}
  \item
    Montrer que pour tout $x > 0$, on a
    \[ f'(x) = \frac{1}{x}∫_0^{+∞} \cos(xt)⋅\frac{t^2 - 1}{(1+t^2)^2} \D t. \]
  \item
    Montrer que $f$ est de classe $\CC2$ sur $\Rps$ et que pour $x > 0$, on a
    \[ f''(x) = \frac{2}{x^2}∫_0^{+∞} \cos(xt)⋅\frac{1-3t^2}{(1+t^2)^3} \D t. \]
  \item
    En déduire que pour tout $x > 0$, on a $f''(x) = f(x)$.
  \end{enumerate}
\item
  Calculer $f(0)$ et $\lim\limits_{+∞} f$.
\item
  En déduire la valeur de $f$.
\item
  Calculer \[ ∫_0^{+∞} \frac{t\sin(xt)}{1+t^2}\D t. \]
\end{enumerate}

\Exercice

En utilisant une équation différentielle linéaire du premier ordre, calculer
\[ f(x) =∫_0^{+∞} \frac{e^{-t}}{√t}{e^{\I x t}} \D t. \]

\end{document}
