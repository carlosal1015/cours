\documentclass{yann}
\yann{layout=onecolumn,geometry={margin=2.5cm}}
\usepackage{booktabs}

\begin{document}
\title{Calcul de primitives}
\maketitle

Il y a longtemps, très longtemps, vous avez appris à dériver une fonction.
C'est utile, mais pas très amusant: il suffit de connaître les dérivées usuelles et d'appliquer les règles sur les sommes, produits, quotients et composées.
C'est très mécanique...

Primitiver une fonction, c'est une toute autre histoire.
Il faut toujours connaître les primitives usuelles (c'est juste le tableau des dérivées, vu à l'envers), et il y a encore une règle pour la somme.
En revanche, il n'y a pas de règle générale qui donne la primitive d'un produit, d'un quotient ou d'une composée.

Mais comment faire pour calculer une primitive?
Au fil du temps, on a trouvé un certain nombre de méthodes\footnote{Une méthode, c'est une astuce qui sert au moins deux fois.} qui s'appliquent dans certains cas particuliers.
C'est d'ailleurs l'objet de ce chapitre.
Après, vous êtes tout seuls.
Il faudra essayer, deviner, combiner les méthodes, trouver des astuces diaboliques, s'apercevoir que ça ne marche pas, recommencer, recommencer encore...
Bref, c'est comme un puzzle, il faudra réfléchir.

Existe-t-il une méthode ultime permettant de calculer tou\-tes les primitives?
Et Wolfram Alpha (ou Maple, Mathematica, Maxima, Sympy, Sage, etc.), au fait: la connaît-il?

À ces questions, la réponse est NON; et ce non ne signifie pas «on a beaucoup cherché, et on n'a rien trouvé», mais un bien plus fort «on a \emph{démontré} qu'il n'existe pas de méthode universelle».
Certaines primitives \emph{ne peuvent pas} s'exprimer à partir\footnote{Plus précisément, en partant des fonctions usuelles et en faisant un nombre fini de sommes, différences, produits, quotients ou compositions.} des fonctions usuelles.

Par exemple, si on demande $\DS ∫\frac{\D x}{\ln x}$ à Wolfram Alpha, il répondra \verb|li(x)|.
Or la fonction $\mathrm{li}$, dite logarithme intégral est par définition.. une primitive de $1/\ln(x)$!
En un sens, c'est de la triche, mais il n'existe pas de meilleure réponse.
Et si on lui demande une fonction assez compliquée comme
\[ ∫\frac{\sin^3\Pa{1+\frac{1}{\ln t}}}{\cos t + √{e^t + 1}} \D t, \]
il ne répondra rien.

\Para{Rappel}
Soit $I$ un intervalle de $ℝ$, et $f, F$ deux fonctions de $I$ dans $𝕂$.
On dit que $F$ est une primitive de $f$ sur $I$ si et seulement si
$F$ est dérivable sur $I$ et $F' = f$.
On note $∫ f = F + C^{\text{ste}}$.

\section{Fonctions usuelles}

\begingroup
\def\Icos{$\intO{kπ-π/2, kπ+π/2}$ où~$k∈ℤ$}
\def\Isin{$\intO{kπ, (k+1)π}$ où~$k∈ℤ$}
\def\IRs{$ℝ^*_-$, $ℝ^*_+$}
\renewcommand\arraystretch{1.5}
\begin{tabular}{ccc}
  \toprule
  Fonction $f$& Intervalle de définition& Primitive de $f$\cr
  \midrule
  $f(x)$& $I$& $∫f(x)\D x = F(x)$\cr
  $f(αx)$ où $α≠0$& $I$& $\frac1αF(αx)$\cr
  \midrule
  $e^x$ & $ℝ$& $e^x$\cr
  $\ln x$ & $\Rps$ & $x\ln(x)-x$\cr
  \midrule
  $x^α$ où $α∈ℂ∖\acco{-1}$& $\Rps$& $\frac{x^{α+1}}{α+1}$\cr
  $x^n$ où $n∈ℕ$& $ℝ$& $\frac{x^{n+1}}{n+1}$\cr
  $x^n$ où $n∈ℤ$, $n≤-2$& \IRs& $\frac{x^{n+1}}{n+1}$\cr
  $x^{-1} = \frac1x$& \IRs& $\ln\Abs{x}$\cr
  \midrule
  $\cos x$& $ℝ$& $\sin x$\cr
  $\sin x$& $ℝ$& $-\cos x$\cr
  $\ch x$& $ℝ$& $\sh x$\cr
  $\sh x$& $ℝ$& $\ch x$\cr
  % $\tan x$& \Icos& $-\ln\abs{\cos x}$\cr
  % $\cotan x$& \Isin& $\ln\abs{\sin x}$\cr
  % $\th x$& $ℝ$& $\ln\ch x$\cr
  % $\coth x$& \IRs& $\ln\abs{\sh x}$\cr
  % \midrule
  $\frac{1}{\cos^2 x} = 1 + \tan^2 x$& \Icos& $\tan x$\cr
  % $\frac{1}{\sin^2 x} = 1 + \cotan^2 x$& \Isin& $-\cotan x$\cr
  % $\frac{1}{\ch^2 x} = 1 - \th^2 x$& $ℝ$& $\th x$\cr
  % $\frac{1}{\sh^2 x} = \coth^2 x - 1$& \IRs& $-\coth x$\cr
  \midrule
  $\frac{1}{1+x^2}$& $ℝ$& $\arctan x$\cr
  % $\frac{1}{1-x^2}$& $]-1,1[$& $\argth x$\cr
  % $\frac{1}{1-x^2}$& $]-∞,-1[$, $]-1,1[$, $]1,+∞[$& $\frac12\ln\Abs{\frac{1+x}{1-x}}$\cr
  % $\frac{1}{√{1+x^2}}$& $ℝ$& $\argsh x = \ln\Pa{x+√{x^2+1}}$\cr
  % $\frac{1}{√{1-x^2}}$& $]-1,1[$& $\arcsin x$ ou  $-\arccos x$\cr
  % $\frac{1}{√{x^2-1}}$& $]1,+∞[$& $\argch x = \ln\Pa{x+√{x^2-1}}$\cr
  % $\frac{1}{√{x^2-1}}$& $]-∞,-1[$, $]1,+∞[$& $\ln\Abs{x+√{x^2-1}}$\cr
  % \midrule
  $\frac{1}{a^2+x^2}$ où $a ∈ℝ^*$& $ℝ$& $\frac1a \arctan\Pafrac xa$\cr
  % $\frac{1}{a^2-x^2}$ où $a > 0$& $]-∞,-a[$, $]-a,a[$, $]a,+∞[$& $\frac1{2a} \ln\Abs{\frac{a+x}{a-x}}$\cr
  % $\frac{1}{√{a^2+x^2}}$ où $a ∈\Rs$& $ℝ$& $\argsh\Pafrac xa$\cr
  % $\frac{1}{√{a^2-x^2}}$ où $a > 0$& $]-a,a[$& $\arcsin\Pafrac xa$ ou $-\arccos\Pafrac xa$\cr
  % $\frac{1}{√{x^2-a^2}}$ où $a > 0$& $]-∞,-a[$, $]a,+∞[$& $\argch\Pafrac xa$\cr
  \bottomrule
\end{tabular}
\endgroup

\yann{displaystyle}

\section{Méthodes générales}

\subsection{Changement de variables}

Soit $f$ une fonction de classe $\CC0$ et $ϕ$ de classe $\CC1$.
Le changement de variable $x = ϕ(y)$ s'écrit:
\[ ∫ f(x) \D x = ∫ f(ϕ(y)) ϕ'(y) \D y \]

Si l'on souhaite faire un changement de variables \enquote{dans l'autre sens}, £cad. de la forme $y = ψ(x)$,
on doit s'assurer que $ψ$ est bijective et que $ψ^{-1}$ est de classe $\CC1$;
cela est garanti si $ψ$ est une bijection de classe $\CC1$ dont la dérivée ne s'annule pas.

\Para{Exemples}
$∫\frac{\D x}{x^2 + a^2}$,
$∫\tan x \D x$,
$∫\frac{\arcsin x}{√{1-x^2}} \D x$,
$∫\frac{\D x}{√{1+√{1+x}}}$.

\subsection{Intégration par parties}

Si $f$ est de classe $\CC0$ et $g$ est de classe $\CC1$, on a:
\[ ∫fg = Fg - ∫Fg' \qquad \text{ où } F = ∫f \]

\Para{Exemples}
$∫\ln x \D x$,
$∫\arcsin x \D x$,
$∫\ln\Pa{1+x^2} \D x$,
$∫\frac{x \D x}{\cos^2 x}$.

\subsection{Généralisation: intégration par parties multiple}

Si $f$ et $g$ sont de classe $\CC n$:

\begin{align*}
  ∫ f^{(n)} g
  &= ∑_{k=0}^{n-1} (-1)^k f^{(n-1-k)} g^{(k)} + (-1)^n ∫fg^{(n)} \\
  &= f^{(n-1)} g - f^{(n-2)}g' + f^{(n-3)}g'' - f^{(n-4)}g^{(3)} + \cdots
  + (-1)^{n-1} f g^{(n-1)} + (-1)^n ∫fg^{(n)}
\end{align*}

\Para{Exemples}
$∫(x^3 + x^2 + 1) e^x \D x$,
$∫_{-1}^1 (1-x^2)^n \D x$.

\section{Fonctions polynôme-exponentielle}

\subsection{$∫P(x) e^{αx} \D x$}

On suppose $P ∈ℂ[X]$, $α∈ℂ^*$. Soit $d$ le degré de $P$.
On dispose de deux méthodes:
\begin{itemize}
\item
  intégration par parties: on effectue $d$ intégrations par parties (ou une intégration par parties multiple) en dérivant le polynôme et en primitivant l'exponentielle;
\item
  identification (préférable quand $d ≥3$): on cherche $Q ∈ℂ_d[X]$ tel que \[ ∫P(x) e^{αx} \D x = Q(X) e^{αx} + C, \] £cad. $αQ(X) + Q'(X) = P(X)$.
\end{itemize}

\Para{Exemple}
$∫(x^2-x+3) e^{2x} \D x$.

\subsection{$∫e^{αx} \cos(βx) \D x$}

On suppose $(α,β) ∈ℝ^2 ∖\Acco{(0,0)}$.
On calcule $∫e^{(α+ iβ) x} \D x$, et l'on sépare partie réelle et partie imaginaire.

\Para{Exemple}
$∫e^{2x} \sin{3x} \D x$.

\subsection{$∫P(x) \cos(βx) \D x$}

On suppose $P ∈ℝ_d[X]$, $β∈ℝ^*$.
On dispose de trois méthodes:
\begin{itemize}
\item
  intégration par parties;
\item
  identification: on cherche $∫P(x) \cos(βx) \D x$ sous la forme $A(x) \cos(βx) + B(x) \sin(βx)$ où $(A,B) ∈ℝ_d[X]^2$;
\item
  passage par l'exponentielle complexe: on commence par calculer $∫P(x) e^{iβx} \D x$.
\end{itemize}

\Para{Exemple}
$∫x^3 \sin x \D x$.

\subsection{$∫P(x) e^{αx} \cos(βx) \D x$}

On suppose $P ∈ℝ[X]$, $(α,β) ∈ℝ^2 ∖\Acco{(0,0)}$.
On dispose de deux méthodes:
\begin{itemize}
\item
  passage à l'exponentielle complexe puis intégration par parties;
\item
  identification.
\end{itemize}

\Para{Exemple}
$∫x^3 e^x \cos x \D x$.

\section{Fonctions rationnelles}

Pour toute fonction rationnelle $R ∈ℝ(X)$ dont on sait factoriser le dénominateur, on dispose d'un algorithme permettant de calculer $∫R(x) \D x$.

\Para{Astuce}
Si $R$ est impaire, on peut écrire \[ R(X) = X \frac{P(X^2)}{Q(X^2)}, \] de sorte que si l'on fait le changement de variables $y=x^2$, on a \[ ∫R(x) \D x = \frac12 ∫\frac{P(y)}{Q(y)} \D y. \]
Il reste à intégrer $S(Y) = \frac{P(Y)}{Q(Y)}$, qui est plus simple que $R$.
Ainsi, on a abaissé le degré et donc simplifié le calcul.

\subsection{Décomposition en éléments simples}

\Para{Exemples}[pôles simples réels]
\[ \frac{1}{X^2-1}, \quad
\frac{X+1}{X+2}, \quad
\frac{2X+8}{X^2+8X+15}, \]
\[ \frac{X^4+X+1}{X(X^2-1)}, \quad
\frac{X^5+6X^4+9X^3+6}{X(X+1)(X+2)(X+3)}. \]

\Para{Exemples}[pôles multiples réels]
\[ \frac{1}{X^2(X+1)}, \quad \frac{X^3+2}{(X^2-1)^2}. \]

\Para{Exemples}[pôles simples complexes]
\[ \frac{1}{X(X^2+1)}, \quad
\frac{-X^3+2X^2+1}{X^3+X}, \quad
\frac{2X^3-4X+2}{X^3+2X}. \]

\Para{Exemple}[pôles multiples complexes]
\[ \frac{X^3+3X+1}{X^5+2X^3+X}. \]

\subsection{Éléments simples de première espèce}

On cherche à calculer \[ ∫\frac{c}{(ax+b)^n} \D x. \]
Ce cas est très facile.
Attention toutefois au cas $n = 1$ si $(a,b) ∉ℝ^2$: il faut alors multiplier par la quantité conjuguée et se ramener à un élément simple de seconde espèce.

\subsection{Éléments simples de seconde espèce}

On cherche à calculer \[ ∫\frac{dx+e}{(ax^2+bx+c)^n} \D x, \]
où $(a,b,c)∈ℝ^3$, $a≠0$, $Δ=b^2-4ac<0$.

On procède par étapes.

\subsubsection{Première étape}

On effectue le changement de variables $y = x+\frac{b}{2a}$, de sorte que
\[ ∫\frac{dx+e}{(ax^2+bx+c)^n} \D x = ∫\frac{dy+e'}{(ay^2+c')^n} \D y. \]
Notons que l'on a nécessairement $c'>0$.

\subsubsection{Deuxième étape}

On sépare l'intégrale en deux, de sorte que l'on doit calculer:
\begin{itemize}
\item
  $I = ∫\frac{dy}{(ay^2+c')^n} \D y$, qui est immédiate à calculer, car l'intégrande est de la forme $\frac{u'}{u^n}$;
\item
  $J = ∫\frac{e'}{(ay^2+c')^n} \D y$, qui se ramène immédiatement à $K = ∫\frac{\D y}{(y^2+α^2)^n}$.
\end{itemize}

\subsubsection{Troisième étape}

Si $n=1$, c'est gagné, on connaît une primitive:
\[ \frac{1}{α}\arctan\Pafrac{y}{α}. \]
Sinon, on fait le changement de variables $y=tα$, de façon à se ramener à
\[ J_n = ∫\frac{\D t}{(t^2+1)^n}. \]
Avec une intégration par parties, on établit une relation de récurrence entre les $J_n$ permettant
de calculer $J_2$, $J_3$, etc...

\subsubsection{Conclusion} On exprime en fonction de la variable initiale, puis on regroupe le tout.

\Para{Exemples}
$∫\frac{\D x}{x(x^2+1)^3}$,
$∫\frac{x^3 \D x}{x^2+2x+2}$,
$∫\frac{\D x}{x^3+1}$.

% -----------------------------------------------------------------------------
\section{Fonctions de la forme $R(\cosθ,\sinθ)$}
On suppose $R ∈ℝ(X,Y)$.

\subsection{Cas particulier: $∫\cos^m θ\sin^n θ\D θ$}

On a une méthode qui marche toujours: linéariser.
Cependant, on peut parfois être plus malin.
Si $m$ est impair, on peut poser $x = \sin θ$;
si $n$ est impair, on peut poser $x = \cos θ$.

\Para{Exemple}
$∫\sin^7 θ\cos^4 θ\D θ$.

\subsection{Cas général}

On a une méthode qui marche toujours: la tangente de l'arc moitié, £cad. le changement de variable $x = \tan\frac{θ}{2}$.
Cela nous ramène à primitiver une fraction rationnelle, mais les calculs sont souvent horribles.
On peut être plus malin.

La \emph{règle de Bioche} répond au problème.
On forme l'élément différentiel $ω(θ) = R(\cosθ,\sinθ) \D θ$.
\begin{itemize}
\item
  si $ω(-θ)  = ω(θ)$, alors on pose $x = \cosθ$;
\item
  si $ω(π-θ) = ω(θ)$, alors on pose $x = \sinθ$;
\item
  si $ω(π+θ) = ω(θ)$, alors on pose $x = \tanθ$;
\item
  sinon, on pose $x = \tan\frac{θ}{2}$.
  Rappel: on a alors $\sinθ= \frac{2x}{1+x^2}$ et $\cosθ=\frac{1-x^2}{1+x^2}$.
\end{itemize}
Dans tous les cas, on se ramène alors à primitiver une fraction rationnelle.

\Para{Exemple}
$∫ \frac{\sin^2 x}{2+\cos x} \D x$,
$∫ \frac{\D x}{2+\cos x}$.

\subsection{Fonctions de la forme $R(\chθ, \shθ)$}
On suppose $R ∈ℝ(X,Y)$.

Il suffit d'appliquer la règle de Bioche à $R(\cosθ, \sinθ)$, puis:
\begin{itemize}
\item
  au lieu de poser $x = \cosθ$, on pose $x = \chθ$;
\item
  au lieu de poser $x = \sinθ$, on pose $x = \shθ$;
\item
  au lieu de poser $x = \tanθ$, on pose $x = \thθ$;
\item
  au lieu de poser $x = \tan\fracθ2$, on pose $x = \th\fracθ2$ ou bien $x = e^θ$.
\end{itemize}

\Para{Exemple}
$∫\frac{\sh^2 x}{\ch x(2\sh^3 x + 3\ch^3 x)} \D x$.

% -----------------------------------------------------------------------------
\section{Fonctions de la forme $R\Pa{e^{αx}}$}
On suppose $R ∈ℝ(X)$.
On pose $y = e^{αx}$.

\Para{Exemple}
$∫\frac{\D x}{(e^x + 2)^2}$.

Le même changement marche parfois même si $R$ n'est pas rationnelle, par exemple pour:
$∫√{e^x + 1} \D x$.

% -----------------------------------------------------------------------------
\section{Fonctions de la forme $R\Pa{x,√[n]{\frac{ax+b}{cx+d}}}$}
On suppose $R ∈ℝ(X,Y)$.
On pose $y = √[n]{\frac{ax+b}{cx+d}}$.

\Para{Exemples}
$∫√{\frac{x}{(1-x)^3}} \D x$,
$∫\frac{1}{√{x}+√{x+1}}$,
$∫{\frac{\D x}{√{x} + √[3]{x}}}$.

% -----------------------------------------------------------------------------
\section{Intégrales abéliennes $R\Pa{x, √{ax^2+bx+c}}$}
On suppose $R ∈ℝ(X,Y)$.

Notons $Δ= b^2 - 4ac$.
\begin{itemize}
\item
  $Δ= 0$: la racine carrée disparaît, il ne reste qu'une valeur absolue,
  qui disparait également si on restreint l'intervalle.
\item
  $a<0$, $Δ<0$: impossible car $∀x ∈ℝ$, $ax^2+bx+c < 0$.
\item
  $a>0$, $Δ<0$: par un changement de variable affine,
  on se ramène à $G(t,√{1+t^2})$, puis on pose $t = \shθ$.
\item
  $a<0$, $Δ>0$: par un changement de variable affine,
  on se ramène à $G(t,√{1-t^2})$, puis on pose $t = \sinθ$ ou $t = \cosθ$.
\item
  $a>0$, $Δ>0$: par un changement de variable affine,
  on se ramène à $G(t,√{t^2-1})$, puis on pose $t = \chθ$.
\end{itemize}

\Para{Exemple}
$∫\frac{x \D x}{√{x^2+x+1}}$,
$∫\frac{x \D x}{1 + √{x^2+x+1}}$.

%\end{multicols}
\newpage

\end{document}
