\documentclass{yann}

\newcommand\A{\mathbb{A}}
\newcommand\Epz{E\priveZE}
\newcommand\priveZE{∖\acco{0_E}}

\begin{document}
\title{Réduction des endomorphismes}
\maketitle

\Para{Introduction}

Réduire un endomorphisme $u∈\LE$,
c'est décomposer l'espace $E$ comme somme directe de sous-espaces stables par $u$,
$E = ⨁_{i∈I} E_i$ où les $E_i$ sont stables par $u$.

Ainsi, l'étude de $u$ se ramène à l'étude des endomorphismes $u_i$ induits par $u$ sur $E_i$.

L'idée directrice est que les $u_i$ sont plus \emph{simples} que $u$;
dans le cas favorable, les $u_i$ sont des homothéties, £cad. $u_i =λ_i \Id_{E_i}$.

\Para{Exemples}
\begin{itemize}
\item projecteurs;
\item symétries.
\end{itemize}

\Para{Notations}
\begin{itemize}
\item $𝕂$ désigne un corps (ici $ℝ$ ou $ℂ$);
\item $E$ désigne un $𝕂$-espace vectoriel;
\item $\LE$ désigne l'ensemble des endomorphismes de $E$,
  £cad. l'ensemble des applications linéaires de $E$ dans $E$.
\end{itemize}

% -----------------------------------------------------------------------------

\Para{Définitions}

Soit $u∈\LE$.
\begin{itemize}
\item On dit que $λ∈𝕂$ est une \emph{valeur propre} de $u$ si et seulement si il existe $x∈\Epz$ tel que $u(x) =λx$.
\item On dit que $x∈E$ est un \emph{vecteur propre} de $u$ associé à la valeur propre $λ$ si et seulement si $x∈\Epz$ et $u(x) =λx$.
\item On appelle \emph{spectre} de $u$, noté $\Sp(u)$, l'ensemble des valeurs propres de $u$.
\item On appelle \emph{sous-espace propre} de $u$ associé à la valeur propre $λ$ le sous-espace vectoriel $E_λ(u) = \Ker(u -λ\Id_E)$.
\end{itemize}

\Para{Proposition}

Soit $u∈\LE$. Les conditions suivantes sont équivalentes:
\begin{itemize}
\item $λ∈\Sp u$;
\item $∃x∈\Epz$ tel que $(u -λ\Id_E)(x) = 0_E$;
\item $E_λ(u) = \Ker(u -λ\Id_E)≠\Acco{0_E}$;
\item $u -λ\Id_E$ n'est pas injective.
\end{itemize}

\Para{Proposition}

Soit $u∈\LE$ et $x∈\Epz$.
Alors $x$ est un vecteur propre de $u$ si et seulement si la droite $𝕂x$ est stable par $u$.

\Para{Proposition}

Soit $u∈\LE$ et $λ∈\Sp u$.
Alors l'ensemble des vecteurs propres de $u$ associés à la valeur propre $λ$ est $\Ker(u-λ\Id_E) \priveZE$.

\Para{Proposition}

Soit $u∈\LE$, $λ$ une valeur propre de $u$ et $E_λ$ le sous-espace propre correspondant.
Alors $E_λ$ est stable par $u$.
De plus, l'endomorphisme induit par $u$ sur $E_λ$ est l'homothéthie de rapport $λ$.
Autrement dit, $u_{\vert E_λ} =λ\Id_{E_λ}$.

\Para{Lemme}

Soit $u$ et $v$ deux endomorphismes de $E$.
Si $u$ et $v$ commutent, les sous-espaces propres de $u$ sont stables par $v$.

\Para{Lemme}

Soit $u∈\LE$, $x∈E$, $λ∈𝕂$ et $P∈𝕂[X]$.
On suppose que $u(x) = λx$.
Alors $P(u)(x) = P(λ) x$.

\Para{Théorème}

Soit $u∈\LE$.
Les sous-espaces propres de $u$ sont en somme directe.
Autrement dit, si $(x_i)_{i∈I}$ est une famille de vecteurs propres
associés à des valeurs propres deux à deux distinctes,
alors la famille $(x_i)_{i∈I}$ est libre.

% -----------------------------------------------------------------------------
\section{Polynômes d'endomorphismes}

\subsection{Polynômes annulateurs}

\Para{Définition}

Soit $u∈\LE$.
Un polynôme $P∈𝕂[X]$ est dit \emph{polynôme annulateur de $u$} s'il est non nul et si $P(u) = 0 ∈\LE$.

\Para{Théorème}

Soit $u∈\LE$ et $P$ un polynôme annulateur de $u$.
Alors toutes les valeurs propres de $u$ sont des racines de $P$, £cad. $\Sp(u)⊂\Racines(P)$.

\Para{Notation}

À partir de maintenant et jusqu'à la fin du chapitre, on supposera que $E$ est un espace vectoriel de dimension finie.

\subsection{Polynôme caractéristique}

\Para{Lemme}

Soit $u∈\LE$.
L'application de $𝕂$ dans $𝕂$ définie par $λ \mapsto \det(λ\Id_E - u)$ est une fonction polynômiale.

\Para{Définition}

Soit $u∈\LE$.
On appelle \emph{polynôme caractéristique} de $u$ et on note $χ_u$ l'unique polynôme à coefficients dans $𝕂$ tel que $∀λ∈𝕂$, $χ_u(λ) = \det(λ\Id_E - u)$.

\Para{Proposition}

Soit $u∈\LE$.
Le spectre de $u$ est exactement l'ensemble des racines de $χ_u$, £cad. $\Sp(u) = \Racines(χ_u)$.

\Para{Théorème}

Soit $u∈\LE$, et $n = \dim E$. Alors:
\begin{itemize}
\item $χ_u$ est un polynôme unitaire de degré $n$;
\item le coefficient en $X^{n-1}$ de $χ_u$ vaut $- \Tr u$;
\item le coefficient constant de $χ_u$ vaut $(-1)^n \det u$.
\end{itemize}

\Para{Corollaire}

Soit $u∈\LE$. On suppose que $χ_u$ est scindé;
$χ_u$ peut alors s'écrire sous la forme $χ_u(X)=∏_{k=1}^n(X-α_k)$.

On a $\Tr u = ∑_{k=1}^n α_k$ et $\det u = ∏_{k=1}^n α_k$.

\Para{Proposition}

Soit $u∈\LE$.
Si $F$ est un sous-espace vectoriel stable par $u$,
et $v$ l'endomorphisme induit par $u$ sur $F$,
alors $χ_v$ divise $χ_u$.

\Para{Proposition}

Soit $u∈\LE$.
Si $F$ et $G$ sont des sous-espaces vectoriels $u$-stables tels que $E = F⊕G$,
$v$ et $w$ les endomorphismes induits par $u$ sur $F$ et $G$ respectivement,
alors $χ_u = χ_v ⋅χ_w$.

\Para{Définition}

On appelle \emph{multiplicité de la valeur propre $λ$} la multiplicité de $λ$ comme racine de $χ_u$.

\Para{Théorème}

Soit $u∈\LE$ et $λ$ une valeur propre de $u$.
Alors $1≤\dim E_λ≤m_λ$ où
\begin{itemize}
\item $E_λ$ est le sous-espace propre de $u$ associé à la valeur propre $λ$;
\item $m_λ$ est la multiplicité de la valeur propre $λ$.
\end{itemize}

\Para{Théorème}[Cayley-Hamilton]

Soit $u∈\LE$.
Alors le polynôme caractéristique de $u$ est un polynôme annulateur de $u$.

% -----------------------------------------------------------------------------
\section{Endomorphismes diagonalisables}

\Para{Définitions}
\begin{itemize}
\item Un endomorphisme $u∈\LE$ est dit \emph{diagonalisable} si et seulement si
  il existe une base $\B$ de $E$ telle que $\Mat_\B(u)$ est diagonale.
\item Une matrice $A$ de $\MnK$ est dite \emph{diagonalisable} si et seulement si
  elle est semblable à une matrice diagonale.
\end{itemize}

\Para{Proposition}

Soit $u∈\LE$ et $\B$ une base de $E$.
Alors $u$ est diagonalisable si et seulement si $\Mat_\B(u)$ est diagonalisable.

\Para{Remarque}

On dira que $P∈𝕂[X]$ est \emph{scindé simple} s'il est scindé et que toutes ses racines sont simples.

\vfil

\Para{Théorème fondamental}

Soit $u∈\LE$.
Pour toute valeur propre $λ$ de $u$, on note $m_λ$ sa multiplicité et $E_λ$ le sous-espace propre associé.
Les conditions suivantes sont équivalentes:
\begin{enumerate}
\item $u$ est diagonalisable, £cad. il existe une base $\B$ de $E$ telle que $\Mat_\B(u)$ est diagonale;
\item $u$ admet un polynôme annulateur scindé et $∀λ∈\Sp u$, $\dim E_λ= m_λ$;
\item $χ_u$ est scindé et $∀λ∈\Sp u$, $\dim E_λ= m_λ$;
\item $∑_{λ∈\Sp u} \dim E_λ = \dim E$;
\item $E$ est somme directe des espaces propres de $u$, £cad. $E = ⨁_{λ∈\Sp u} E_λ$;
\item Il existe une base $\B$ de $E$ formée de vecteurs propres de $u$;
\item Le polynôme scindé simple $∏_{λ∈\Sp u}(X-λ)$ annule $u$;
\item $u$ admet un polynôme annulateur scindé simple.
\end{enumerate}

\vfil

\Para{Corollaires}

Soit $u∈\LE$.
\begin{enumerate}
\item Si $χ_u$ est un polynôme scindé simple, alors $u$ est diagonalisable.
  La réciproque est fausse (contre-exemple: $u = \Id_E$).
\item Si $u$ est diagonalisable et que $F$ est stable par $u$, alors l'endomorphisme induit par $u$ sur $F$ est également diagonalisable.
\item Si $E = ⨁_{k=1}^p F_k$ et que pour tout $k∈\ccro{1,p}$, $u_{\vert F_k}$ est une homothétie, alors $u$ est diagonalisable.
  Pour la réciproque, cf. 2. du théorème précédent.
\end{enumerate}

\Para{Théorème}[cf. chapitre \emph{algèbre bilinéaire}]

Toute matrice symétrique réelle est diagonalisable.

\Para{Remarque}[en pratique]

Une méthode pour diagonaliser une matrice $A$ de $\MnK$:
\begin{itemize}
\item on calcule le polynôme caractéristique $χ_A$;
\item on factorise $χ_A$: s'il n'est pas scindé, c'est que $A$ n'est pas diagonalisable dans $𝕂$;
\item pour chaque racine $λ$, on détermine une base du sous-espace propre $E_λ$ en résolvant le système linéaire $AX =λX$. Si on trouve strictement moins de $m_λ$ vecteurs libres, c'est que $A$ n'est pas diagonalisable;
\item soit $P$ la matrice carrée dont les colonnes sont les $n$ vecteurs propres trouvés. Soit $D$ la matrice diagonale formée des valeurs propres correspondantes. On a alors $A = PDP^{-1}$ et $D = P^{-1}AP$.
\end{itemize}

\Para{Applications}
\begin{itemize}
\item calcul des puissances d'une matrice $A$;
\item suites linéaires récurrentes à coefficients constants;
\item système de suites linéaires récurrences à coefficients constants;
\item systèmes différentiels linéaires à coefficients constants.
\end{itemize}

% -----------------------------------------------------------------------------
\section{Endomorphismes trigonalisables}

\Para{Définitions}
\begin{itemize}
\item Un endomorphisme $u∈\LE$ est dit \emph{trigonalisable} si et seulement si il existe une base $\B$ de $E$ telle que $\Mat_\B(u)$ est triangulaire supérieure.
\item Une matrice $A$ de $\MnK$ est dite \emph{trigonalisable} si et seulement si elle est semblable à une matrice triangulaire supérieure.
\end{itemize}

\Para{Proposition}

Soit $u∈\LE$ et $\B$ une base de $E$.
Alors $u$ est trigonalisable si et seulement si $\Mat_\B(u)$ est trigonalisable.

\Para{Théorème}

Soit $u∈\LE$. Les conditions suivantes sont équivalentes:
\begin{enumerate}
\item $u$ est trigonalisable, £cad. il existe $\B$ base de $E$ telle que $\Mat_\B(u)$ est triangulaire supérieure;
\item $χ_u$ est scindé;
\item $u$ admet un polynôme annulateur scindé.
\end{enumerate}

\Para{Corollaire}
\begin{enumerate}
\item Tout endomorphisme d'un $ℂ$-espace vectoriel de dimension finie est trigonalisable.
\item Toute matrice carrée est trigonalisable dans $\MnC$.
\end{enumerate}

% -----------------------------------------------------------------------------
\section{Un peu de topologie sur $\LE$}

\Para{Lemme}

Soit $u∈\LE$.
Alors $∃n_0∈\Ns$, $∀n≥n_0$, $u + \frac1n \,\Id_E ∈\GLE$.

\Para{Corollaire}

Soit $u∈\LE$.
Il existe une suite numérique $(ε_n)_{n∈ℕ}$ convergeant vers $0$ telle que
pour tout $n∈ℕ$, l'endomorphisme $u +ε_n \Id_E$ est inversible.

\Para{Corollaire}

$\GLE$ est dense dans $\LE$.
De façon équivalente, $\GLnK$ est dense dans $\MnK$.

\Para{Proposition}

Soit $A$ et $B$ deux matrices de $\MnK$.
Les matrices $AB$ et $BA$ ont le même polynôme caractéristique.

\Para{Théorème}

L'ensemble des matrices diagonalisables complexes est dense dans $\MnC$.

\Para{Théorème}[Cayley-Hamilton]

Soit $A∈\MnK$. Alors $χ_A(A) = 0$.

% -----------------------------------------------------------------------------
\section{Exercices}

\subsection{Calculs explicites}

\Exercice

Les matrices suivantes sont-elles diagonalisables?
Si oui, les diagonaliser.
\begin{align*}
    A &= \begin{pmatrix} 2 & 0 & 1 \\ 1 & 1 & 1 \\ -2 & 0 & -1 \end{pmatrix}, &
    B &= \begin{pmatrix} 5 & -17 & 25 \\ 2 & -9 & 16 \\ 1 & -5 & 9 \end{pmatrix}, \\
    C &= \begin{pmatrix} 0 & 1 & 0 \\ 1 & 0 & 1 \\ 0 & 1 & 0 \end{pmatrix}, \\
    D &= \begin{pmatrix} 11 & -5 & 5 \\ -5 & 3 & -3 \\ 5 & -3 & 3 \end{pmatrix}.
\end{align*}

\Exercice

Soit $A = \begin{pmatrix} 1 & 0 & 0 \\ 0 & 1 & 1 \\ 1 & 0 & 1 \end{pmatrix}$.
En écrivant $A = I_3 + B$, calculer $A^n$ pour $n∈ℕ$.

\Exercice

Soit \[ A = \begin{pmatrix} 29 & 38 & -18 \\ -11 & -14 & 7 \\ 20 & 27 & -12 \end{pmatrix} \quad\text{et}\quad
B = \begin{pmatrix} 3 & 1 & 2 \\ 2 & 0 & 1 \\ 1 & 0 & 0 \end{pmatrix}. \]
Montrer que $A$ et $B$ ont même rang, même déterminant et même trace
mais ne sont pas semblables.

\Exercice

Soit $J$ la matrice de $\MnR$ définie par
\[ J = \begin{pmatrix} 1 & \cdots & 1 \\ \vdots & (1) & \vdots \\ 1 & \cdots & 1 \end{pmatrix}. \]
\begin{enumerate}
\item Déterminer un polynôme annulateur de $J$ de degré~2.
\item Montrer que $J$ est diagonalisable et déterminer ses valeurs propres (et leurs multiplicités).
\item En déduire la valeur du déterminant
  \[ \begin{vmatrix} a &  & (b) \\  & \ddots \\ (b) &  & a \end{vmatrix}. \]
\end{enumerate}

\Exercice

Soit $A = \begin{pmatrix} 0 & 1 &  & (0) \\ 1 & \ddots & \ddots \\  & \ddots & \ddots & 1 \\ (0) &  & 1 & 0 \end{pmatrix}∈\MnR$.
\begin{enumerate}
\item Calculer $D_n(θ) = \det\bigl(A + (2\cosθ) I_n\bigr)$ par récurrence.
\item En déduire les valeurs propres de $A$.
\end{enumerate}

\Exercice

Soit $A∈\mathrm{M}_3(ℝ)$ telle que $\Sp A = \Acco{1,-2,2}$.
\begin{enumerate}
\item Déterminer $χ_A$.
\item Montrer que $A^n$ peut s'écrire sous la forme $A^n =α_n A^2 +β_n A +γ_n I_3$ avec $(α_n,β_n,γ_n)∈ℝ^3$.
\item On considère le polynôme $P_n(X) =α_n X^2 +β_n X +γ_n$.
  Montrer que $P_n(1) = 1$, $P_n(2) = 2^n$ et $P_n(-2) = (-2)^n$.
\item En déduire les coefficients $α_n$, $β_n$ et $γ_n$.
\end{enumerate}

\Exercice

Calculer les puissances $p^{\text{ièmes}}$ ($p∈ℕ$ ou $p∈ℤ$) des matrices suivantes:

$A = \begin{pmatrix} 1 &  & (2) \\  & \ddots \\ (2) &  & 1 \end{pmatrix}$;
$B = \begin{pmatrix} 1 & 2 & 3 & 4 \\ 0 & 1 & 2 & 3 \\ 0 & 0 & 1 & 2 \\ 0 & 0 & 0 & 1 \end{pmatrix}$.

\Exercice

Soit $(u_n)_{n∈ℕ}$ une suite réelle vérifiant la relation de récurrence
$∀n∈ℕ$, $u_{n+3} = 6u_{n+2}-11u_{n+1}+6u_n$.
On pose $X_n = \begin{pmatrix} u_n \\ u_{n+1} \\ u_{n+2} \end{pmatrix}$.
\begin{enumerate}
\item Expliciter une matrice $A∈\mathrm{M}_3(ℝ)$ telle que $X_{n+1} = AX_n$.
\item Diagonaliser $A$.
\item En déduire $u_n$ en fonction de $n$.
\end{enumerate}

\Exercice

Soit $A = \begin{pmatrix} 5 & -4 & 1 \\ 8 & -7 & 2 \\ 12 & -12 & 4 \end{pmatrix}$.
\begin{enumerate}
\item Trouver une matrice $B∈\mathrm{M}_3(ℝ)$ différente de $A$ et de $-A$ telle que $B^2 = A$.
\item Montrer qu'il existe une infinité de matrices $B$ telles que $B^2 = A$.
\end{enumerate}

\Exercice

Soit $A = \begin{pmatrix} 9 & 0 & 0 \\ 1 & 4 & 0 \\ 1 & 1 & 1 \end{pmatrix}$.
Le but de cet exercice est de determiner \emph{toutes} les matrices $B∈\mathrm{M}_3(ℝ)$ telles que $B^2 = A$.
\begin{enumerate}
\item Diagonaliser $A$. On écrira $A = P D P^{-1}$ avec $D$ diagonale.
\item Soit $E$ un $ℝ$-espace vectoriel de dimension~$3$ et $\B$ une base de $E$.
  Soit $f∈\LE$ tel que $\Mat_\B(f) = D$.
  On suppose que $g∈\LE$ vérifie $g^2 = f$.
  Montrer que les sous-espace propre de $f$ sont stables par $g$.
\item En déduire que $\Mat_\B(g)$ est diagonale.
\item Soit $B∈\mathrm{M}_3(ℝ)$ une matrice telle que $B^2 = A$.
  Montrer que $P^{-1} B P$ est diagonale.
\item Déterminer l'ensemble des matrices $B∈\mathrm{M}_3(ℝ)$
  telles que $B^2 = A$.
\end{enumerate}

\Exercice

Soit $A = \begin{pmatrix} -1 & 2 & 1 \\ 2 & -1 & -1 \\ -4 & 4 & 3 \end{pmatrix}$.
\begin{enumerate}
\item Calculer $A^n$.
\item Soit $U_0 = \begin{pmatrix} -2 \\ 4 \\ 1 \end{pmatrix}$ et $(U_n)_{n∈ℕ}$ définie par la relation $U_{n+1} = AU_n$.
  Calculer $U_n$ en fonction de $n$.
\item Soit $X(t) = \begin{pmatrix} x(t) \\ y(t) \\ z(t) \end{pmatrix}$.
  Résoudre le système différentiel $\frac{\D X}{\D t} = AX$.
\end{enumerate}

\Exercice

Calculer les puissances $p^{\text{ièmes}}$ des matrices suivantes:

$A = \begin{pmatrix} -1 & a & -a \\ 1 & -1 & 0 \\ 1 & 0 & -1 \end{pmatrix}$;
$B = \begin{pmatrix} 1 & 1 & 1 \\ 2 & 2 & 2 \\ 0 & 0 & 1 \end{pmatrix}$;
$C = \begin{pmatrix} 1 & a & a^2 \\ a^{-1} & 1 & a \\ a^{-2} & a^{-1} & 1 \end{pmatrix}$.

\subsection{Polynôme caractéristique}

\Exercice

Soit $A∈\MnK$ inversible et $B = A^{-1}$.
Exprimer le polynôme caractéristique $χ_B$ en fonction de $χ_A$.

\Exercice

Soit $E$ un $ℂ$-espace vectoriel de dimension finie. Soit $(u,v)∈\LE^2$.
\begin{enumerate}
\item Montrer que $\Sp(u◦v) = \Sp(v◦u)$.
\item Si $u$ est inversible, montrer que $χ_{u◦v} =χ_{v◦u}$.
\item On rappelle qu'il existe une suite $(ε_k)_{k∈ℕ}$ telle que $ε_k \To{k∞} 0$ et $∀k∈ℕ$, $u +ε_k \Id_E$ est inversible.
  En déduire que $χ_{u◦v} =χ_{v◦u}$.
\end{enumerate}

\Exercice[magique]

Soit $(A,B)∈\MnK^2$,
$C = \begin{pmatrix} XI_n & A \\ B & I_n \end{pmatrix}$ et $D = \begin{pmatrix} I_n & -A \\ 0 & XI_n \end{pmatrix}$.
En écrivant que $\det(CD) = \det(DC)$, montrer que $χ_{AB} =χ_{BA}$.
On retrouve ainsi le résultat de l'exercice précédent.

\subsection{Endomorphismes nilpotents}

\emph{Rappel:} $u$ est dit nilpotent si et seulement si $∃n∈ℕ^*$ tel que $u^n = 0$.

\Exercice

Déterminer les matrices $A∈\MnK$ qui sont à la fois diagonalisables et nilpotentes.

\Exercice

Soit $u$ et $v$ deux endomorphismes nilpotents de $E$ qui commutent.
Montrer que $u+v$ et $u◦v$ sont également nilpotents.

\Exercice

Soit $u∈\GLE$ et $v∈\LE$ nilpotent tel que $u$ et $v$ commutent.
Montrer que $u+v$ est inversible et préciser son inverse;
on commencera par le cas $u = \Id_E$.

\Exercice

Soit $E$ un $𝕂$-espace vectoriel de dimension $n$ et $u∈\LE$.
\begin{enumerate}
\item On suppose $𝕂=ℂ$. Montrer que $u$ est nilpotent si et seulement si $\Sp u = \Acco{0}$.
\item Donner un contre-exemple dans le cas $𝕂=ℝ$.
\item On suppose $𝕂=ℂ$. Montrer que $u$ est nilpotent si et seulement si $χ_u = X^n$.
\item Montrer que cela est encore vrai si $𝕂=ℝ$ (on pourra se ramener à des matrices).
\end{enumerate}

\Exercice

Soit $A∈\MnK$.
Montrer que $A$ est nilpotente si et seulement si $∀k∈ℕ^*$, $\Tr(A^k) = 0$.

\Exercice

Soit $E$ un $𝕂$-espace vectoriel de dimension finie, $(u,v)∈\LE$ tels que $u◦v - v◦u = λv$ où $λ∈𝕂^*$.
\begin{enumerate}
\item Montrer que $∀P∈𝕂[X]$, $u◦P(v) - P(v)◦u =λv◦P'(v)$.
\item Soit $\Fonction{Φ}{\LE}{\LE}{f}{u◦f - f◦u.}$
  Interpréter le résultat précédent en termes de valeurs propres de $Φ$.
\item En déduire que $v$ est nilpotent.
\end{enumerate}

\subsection{Étude de quelques endomorphismes}

\Exercice

Soit $\Fonction{u}{\MnK}{\MnK}{M}{\frac23M - \frac13\T{M}}$
\begin{enumerate}
\item Déterminer un polynôme annulateur de $u$. En déduire que $u$ est diagonalisable.
\item Calculer $\Tr u$ et $\det u$.
\end{enumerate}

\Exercice

Soit $\Fonction{v}{\MnK}{\MnK}{M}{\T{M}}$
\begin{enumerate}
\item Déterminer un polynôme annulateur de $v$. En déduire que $v$ est diagonalisable.
\item Diagonaliser $v$; on ne cherchera pas à expliciter une base de vecteurs propres.
\item Montrer que l'endomorphisme $u$ de l'exercice précédent s'écrit $u = P(v)$
  pour un polynôme $P∈𝕂[X]$.
\item Retrouver les résultats de l'exercice précédent.
\end{enumerate}

\Exercice

Soit $\Fonction{u}{\MnK}{\MnK}{M}{-M+\Tr(M)I_n}$
\begin{enumerate}
\item Déterminer un polynôme annulateur de $u$ de degré 2.
\item Quels sont les éléments propres de $u$? $u$ est-elle diagonalisable? inversible?
  Calculer $\det(u)$.
\item À quelle condition, liant $\Tr M$ et $\Sp M$, la matrice $u(M)$ est-elle inversible?
\end{enumerate}

\Exercice

Soit $E$ un $𝕂$-espace vectoriel de dimension finie, et $a∈\GLE$.
\begin{enumerate}
\item Montrer que $\Fonction{Φ}{\LE}{\LE}{u}{aua^{-1}}$
  est un automorphisme de l'algèbre $\LE$.
\item Comparer les éléments propres de $u$ et de $aua^{-1}$.
\end{enumerate}

\Exercice

Soit $E = 𝕂_n[X]$ et $u$ l'endomorphisme de~$E$ défini par $u(P) = (X^2-1)P''+(2X+1)P'$.
\begin{enumerate}
\item Déterminer la matrice de $u$ dans la base canonique de $𝕂_n[X]$.
\item Montrer que $u$ est diagonalisable.
\end{enumerate}

\Exercice

Déterminer les éléments propres de l'endomorphisme $\Fonction{Φ}{𝕂[X]}{𝕂[X]}{P}{P(2-X)}$

\Exercice

Soit $E$ l'ensemble des fonctions continues de $ℝ$ dans $ℝ$ qui tendent vers~$0$ en $±∞$,
$\Fonction{φ}{ℝ}{ℝ}{x}{2x}$ et $\Fonction{u}{E}{E}{f}{f◦φ}$
\begin{enumerate}
\item Soit $f∈\Epz$ telle que $u(f) =λf$.
  Montrer que $∀x∈ℝ$, $∀n∈ℕ$, $f(2^n x) =λ^n f(x)$.
\item Montrer que $\Abs{λ} < 1$.
\item Montrer que $∀y∈ℝ$, $∀n∈ℕ$, $f(y) =λ^n f(y 2^{-n})$, et en déduire $\Abs{λ} > 1$.
\item Que vaut $\Sp u$?
\end{enumerate}

\subsection{Diagonalisabilité}

\Exercice

Soit $C∈\mathrm{M}_{n,1}(ℝ)$ et $A = C\T{C}$.
\begin{enumerate}
\item Quel est le rang de $A$?
\item En déduire le polynôme caractéristique de $A$.
\item $A$ est-elle diagonalisable?
\end{enumerate}

\Exercice

Trouver les matrices $A∈\MnR$ telles que $A^2 = A$ et $\Tr A = 0$.

\Exercice

Soit $A∈\MnC$ telle que $A = A^{-1}$.
\begin{enumerate}
\item $A$ est-elle diagonalisable?
\item Calculer $∑_{k=0}^{+∞} \frac{A^k}{k!}$.
\end{enumerate}

\Exercice

Soit $E$ un $ℂ$-espace vectoriel de dimension finie, $u∈\LE$.
\begin{enumerate}
\item Si $u$ est diagonalisable, montrer que $u^2$ l'est également.
\item On suppose $u^2$ diagonalisable et $u$ inversible.

  \begin{enumerate}
  \item Montrer que $u^2$ admet un polynôme annulateur de la forme $P = ∏_{i=1}^d (X-α_i)$ où tous les $α_i$ sont non nuls et deux à deux distincts.
  \item Soit $Q =∏_{i=1}^d (X^2-α_i)$. Montrer que $Q$ est scindé à racines simples.
  \item En déduire que $u$ est diagonalisable.
  \end{enumerate}
\item Trouver un endomorphisme $u$, non inversible, tel que $u^2$ soit diagonalisable sans que $u$ le soit.
\end{enumerate}

\Exercice

Soit $A∈\MnC$ inversible diagonalisable et $B∈\MnC$, $p∈\Ns$
tels que $A = B^p$.
\begin{enumerate}
\item Montrer que $B$ est diagonalisable.
  On pourra regarder l'exercice précédent.
\item Si $A$ n'est pas inversible, la conclusion est-elle encore valable?
\end{enumerate}

\Exercice

Soit $A∈\MnK$ une matrice de rang $1$.
Montrer que $A$ est diagonalisable si et seulement si $\Tr A≠0$.

\Exercice

Soit $A∈\MnK$ non nulle et $B = \begin{pmatrix} 0 & A \\ 0 & 0 \end{pmatrix}∈\mathrm{M}_{2n}(𝕂)$.
Montrer que $B$ n'est pas diagonalisable.

\Exercice

Soit $A∈\MnR$ une matrice diagonalisable
et $B = \begin{pmatrix} 2A & A \\ A & 2A \end{pmatrix}∈\mathrm{M}_{2n}(ℝ)$.
Montrer que $B$ est diagonalisable.

\Exercice

Déterminer une condition nécessaire et suffisante portant sur $(a,b)∈ℂ^2$ pour que la matrice
$M = \begin{pmatrix} 0 &  & (a) \\  & \ddots \\ (b) &  & 0 \end{pmatrix}$
soit diagonalisable.

\Exercice

Déterminer les matrice $M∈\GLnC$ telles que $M^2 + \T{M} = I_n$.
On cherchera, si $M$ vérifie cette relation, un polynôme annulateur de $M$.

\Exercice

Soit $A∈\MnR$ telle que $A^3 = A + I_n$.
Montrer que $\det A > 0$.

\Exercice

Soit $n∈\Ns$ et $A∈\MnR$ telle que $A^3+A^2+A = 0$.
Montrer que le rang de $A$ est pair.

\Exercice

Pour toute matrice carrée $M ∈\M{M}{p}{𝕂}$, on note
\[ \exp(M) = ∑_{n=0}^{+∞} \frac{1}{n!} M^n \]
sous réserve de convergence de la série.

\begin{enumerate}
\item
  Soit $D$ est une matrice diagonale.
  Montrer que $\exp(D)$ converge et l'expliciter.
\item
  Soit $M$ une matrice diagonalisable.
  Montrer que $\exp(M)$ converge.
\item
  Montrer que $M$ et $\exp(M)$ commutent.
\item
  On considère le système de Cauchy
  \[ \left\{ \begin{aligned}
    & X'(t) = M X(t) \\
    & X(0) = X_0
  \end{aligned} \right. \]
  où $X_0 ∈𝕂^p$ est donné et $\Fn{X}{t}{𝕂^p}$ est la fonction inconnue.
  Montrer que la solution est donnée par
  \[ X(t) = \exp(tM) X_0. \]
\end{enumerate}
\emph{Remarque:} ces résultats sont encore valable même si $M$ n'est pas diagonalisable.


\subsection{Divers}

\Exercice

Soit $(A,B)∈\MnR^2$ semblables en tant que matrices de $\MnC$.
\begin{enumerate}
\item Montrer qu'il existe $(P,Q)∈\MnR^2$ telles que $P + iQ∈\GLnC$ et $(P+iQ)A = B(P+iQ)$.
\item Montrer que $∀λ∈ℝ$, $(P+λQ)A = B(P+λQ)$.
\item Montrer que $∃λ∈ℝ$, $P + λQ∈\GLnR$.
\item En déduire que $A$ et $B$ sont semblables en tant que matrices de $\MnR$.
\end{enumerate}

\Exercice

Soit $E$ un $𝕂$-espace vectoriel de dimension finie, $u∈\LE$ et $P∈𝕂[X]$ un polynôme annulateur de $u$.
\begin{enumerate}
\item Si $P(0)≠0$, montrer que $u$ est inversible.
\item Si $P'(0)≠0$, montrer que $E = \Ker u ⊕\Ima u$.
\end{enumerate}

\end{document}
