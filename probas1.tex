\documentclass{yann}

\newcommand{\Part}{\mathcal{P}}
\newcommand{\Pro}{\bigl(Ω, \Part(Ω)\bigr)}
\newcommand{\Prob}{\bigl(Ω, \Part(Ω), ℙ\bigr)}

\begin{document}
\title{Probabilités finies}
\maketitle

% -----------------------------------------------------------------------------
\section{Généralités}

\Para{Définitions}

\begin{itemize}
\item
  Un \emph{espace probabilisable fini} est un couple $\Pro$
  où $Ω$ est un ensemble fini
  et $\Part(Ω)$ est l'ensemble des parties de $Ω$.
\item
  $Ω$ s'appelle l'\emph{univers}.
\item
  Un élément de $Ω$ s'appelle un \emph{résultat} ou une \emph{réalisation}.
\item
  Un élément de $\Part(Ω)$ s'appelle un \emph{événement}.
\end{itemize}

\Para{Définitions}

Soit $\Pro$ un espace probabilisable fini.
Soit $A$ et $B$ deux événements.
\begin{itemize}
\item
  L'événement $∅$ est l'événement impossible.
\item
  L'événement $Ω$ est l'événement certain.
\item
  Les événements $∅$ et $Ω$ sont les événements triviaux.
\item
  Un événement de la forme $\{x\}$ où $x∈Ω$ s'appelle un \emph{événement élémentaire}.
\item
  L'événement \og{}non $A$\fg{} est $\bar A =Ω∖A$.
\item
  L'événement \og{}$A$ ou $B$\fg{} est $A∪B$.
\item
  L'événement \og{}$A$ et $B$\fg{} est $A∩B$.
\item
  Les événements $A$ et $B$ sont \emph{incompatibles} s'ils sont disjoints, c.-à-d. si $A∩B=∅$.
\end{itemize}

\Para{Définitions}

Soit $\Pro$ un espace probabilisable fini.
Une \emph{probabilité} sur $\Pro$ est une application
$\Fnℙ{\Part(Ω)}{[0,1]}$ telle que
\begin{itemize}
\item
  $ℙ(Ω)=1$.
\item
  Si $A$ et $B$ sont des événements incompatibles,
  alors $ℙ(A∪B)=ℙ(A)+ℙ(B)$.
\end{itemize}

Le triplet $\Prob$ s'appelle un \emph{espace probabilisé fini}.

\Para{Proposition}

Soit $\Prob$ un espace probabilisé.
Si $A_1, \dots, A_n$ est une famille d'événements deux à deux incompatibles,
alors \[ ℙ\left( ⋃_{i=1}^n A_i \right) = ∑_{i=1}^n ℙ(A_i) \]

\Para{Proposition}

Soit $\Prob$ un espace probabilisé.
Pour tout événement $A$ on a
\[ ℙ(A) = ∑_{x∈A} ℙ\bigl(\{x\}\bigr) \]

\Para{Proposition}

Soit $\Prob$ un espace probabilisé.
\begin{itemize}
\item
  Si $A$ est un événement, alors $ℙ(\bar A) = 1 - ℙ(A)$.
\item
  Si $A$ et $B$ sont deux événements tels que $A⊂B$, alors $ℙ(A)≤ℙ(B)$.
\item
  Si $A$ et $B$ sont deux événements, alors $ℙ(A∪B)=ℙ(A)+ℙ(B)-ℙ(A∩B)$.
\end{itemize}

\Para{Définition}

Soit $\Pro$ un espace probabilisable fini.
Un \emph{système complet d'événements} est une famille $\nUplet A1n$
d'événements deux à deux incompatibles tels que
\[ ⋃_{i=1}^n A_i = Ω. \]

\Para{Proposition}

Soit $\Prob$ un espace probabilisé.
Soit $\nUplet A1n$ un système complet d'événements.
Pour tout événement $B$, on a
\[ ℙ(B) = ∑_{i=1}^n ℙ(B∩A_i). \]

\Para{Proposition-Définition}[probabilité uniforme]

Soit $\Pro$ un espace probabilisable fini.
Alors l'application
\[ \Fonctionℙ{\Part(Ω)}{[0,1]}{A}{\frac{\Card{A}}{\CardΩ}} \]
est une probabilité appelée \emph{probabilité uniforme sur $Ω$}.

% -----------------------------------------------------------------------------
\section{Indépendance}

\Para{Définition}

Soit $\Prob$ un espace probabilisé.
\begin{itemize}
\item
  Deux événements $A$ et $B$ sont \emph{indépendants} si $ℙ(A∩B)=ℙ(A)ℙ(B)$.
\item
  Une famille $\nUplet A1n$ d'évenements sont \emph{(mutuellement) indépendants}
  si pour toute partie $I⊂\Dcro{1,n}$, on a
  \[ ℙ\left( ⋂_{i∈I} A_i \right) = ∏_{i∈I} ℙ(A_i) \]
\end{itemize}

\Para{Attention}

Trois événements peuvent être indépendants deux à deux, sans pour autant être mutuellement indépendants.

\Para{Proposition}

Soit $\Prob$ un espace probabilisé.
Soit $\nUplet A1n$ des événements mutuellement indépendants.
\begin{itemize}
\item
  Si $I⊂\Dcro{1,n}$,
  alors $(A_i)_{i∈I}$ sont des événements mutuellement indépendants.
\item
  Si pour tout $i∈\Dcro{1,n}$, $B_i ∈ \bigl\{ ∅, A_i, \bar A_i,Ω \bigr\}$,
  alors $\nUplet B1n$ sont des événements mutuellement indépendants.
\end{itemize}

% -----------------------------------------------------------------------------
\section{Probabilité conditionnelle}

\Para{Proposition-Définition}

Soit $\Prob$ un espace probabilisé.
Soit $A$ un événement tel que $ℙ(A)>0$.
L'application
\[ \Fonction{ℙ_A}{\Part(Ω)}{[0,1]}{B}{\frac{ℙ(A∩B)}{ℙ(A)}} \]
est une probabilité sur $\Pro$
appelée \emph{probabilité conditionnellement à $A$},
ou \emph{probabilité sachant $A$}.

On note $ℙ(B|A) = ℙ_A(B)$.

\Para{Proposition}[lien avec l'indépendance]

Soit $\Prob$ un espace probabilisé.
Soit $A$ un événement de probabilité non nulle
et $B$ un événement quelconque.
Alors les événements $A$ et $B$ sont indépendants si et seulement si $ℙ(B|A)=ℙ(B)$.

\Para{Formule des probabilités totales}

Soit $\Prob$ un espace probabilisé.
Soit $\nUplet A1n$ un système complet d'événements de probabilités non nulles.
Pour tout événement $B$, on a
\[ ℙ(B) = ∑_{i=1}^n ℙ(B|A_i)ℙ(A_i). \]

\Para{Cas particulier}

Soit $\Prob$ un espace probabilisé.
Soit $A$ un événement de probabilité $ℙ(A) ∈ \intO{0,1}$.
Pour tout événement $B$, on a
\[ ℙ(B) = ℙ(B|A)ℙ(A) + ℙ(B|\bar A)ℙ(\bar A) \]

\Para{Formule de Bayes}

Soit $\Prob$ un espace probabilisé.
Soit $A$ et $B$ deux événements de probabilités non nulles.
Alors
\[ ℙ(A|B) = \frac{ ℙ(B|A)ℙ(A) }{ ℙ(B) } \]

\Para{Corollaire}[formule de Bayes usuelle]

Soit $\Prob$ un espace probabilisé.
Soit $\nUplet A1n$ un système complet d'événements de probabilités non nulles.
Pour tout événement $B$ de probabilité non nulle,
et pour tout $k∈\Dcro{1,n}$, on a
\[ ℙ(A_k|B) = \frac{ ℙ(B|A_k)ℙ(A_k) }{ \DS ∑_{i=1}^n ℙ(B|A_i)ℙ(A_i) }. \]

% -----------------------------------------------------------------------------
\section{Exercices}

\Exercice

Soit $A$ et $B$ deux événements tels que $ℙ(A)=ℙ(B)=\frac34$.
Donner un encadrement, le meilleur possible, de $ℙ(A∩B)$?

\Exercice

Soit $Ω = \{ 1,2,3,4 \}$ muni de la probabilité uniforme.
Soit $A = \{ 1,2 \}$, $B = \{ 1,3 \}$ et $C = \{ 1,4 \}$
Montrer que $A$, $B$ et $C$ sont deux à deux indépendants,
mais qu'ils ne sont pas mutuellement indépendants.

\Exercice

On jette deux dés (à 6 faces). Expliciter l'univers $Ω$.

Soit $A_0$ l'événement \og{}la somme des points est paire\fg{},
$A_1$ l'événement \og{}la somme des points est impaire\fg{}
et $B$ l'événement \og{}la valeur absolue de la différence des points est égale à 4\fg{}.
Combien comptez-vous d'événements élémentaires dans $A_0∖B$,
dans $A_1∖B$?

\Exercice

L'irradiation par les rayons $X$ de vers à soie induit certaines anomalies.
La probabilité d'une anomalie particulière est $p=\frac1{10}$.
\begin{itemize}
\item
  Quelle est la probabilité de trouver au moins un embryon présentant
  cette anomalie, sur dix disséqués?
\item
  Combien faut-il en disséquer pour trouver au moins une anomalie
  avec une probabilité supérieure à $50\%$? à $95\%$?
\end{itemize}

\Exercice

Un professeur décide de faire passer rapidement l'oral de \og{}probabilités\fg{}.
L'étudiant est autorisé à répartir quatre boules, deux blanches et deux noires,
entre deux urnes. Le professeur choisit au hasard une des urnes et en extrait
une boule. Si la boule est noire, l'étudiant est reçu.
Comment répartiriez-vous les boules?

\Exercice

On plombe un dé à 6 faces de sorte que la probabilité d'apparition d'une face
donnée est proportionnelle au nombre de points de cette face.
On lance le dé deux fois.
Quelle est la probabilité d'obtenir une somme des points égale à 4?

\Exercice

Un jeu consiste à lancer une pièce (diamètre $3$ cm) sur une table
quadrillée par des carrés de $4$ cm de côtés.
On gagne si la pièce tombe entièrement à l'intérieur d'un carré.
Quelle est la probabilité de gagner à ce jeu?

\Exercice

Un appareil contient $6$ transistors dont $2$ exactement sont défectueux.
On les identifie en testant les transistors l'un après l'autre.
Le test s'arrête quand les $2$ transistors défectueux sont trouvés.
Calculer la probabilité pour que le test:
\begin{itemize}
\item
  soit terminé au bout de $2$ opérations?
\item
  nécessite strictement plus de $3$ opérations?
\end{itemize}

\Exercice

Dans une course de $20$ chevaux, quelle est la probabilité, en jouant
$3$ chevaux, de gagner le tiercé dans l'ordre?
dans l'ordre ou le désordre?
dans le désordre?

\Exercice

Au Loto, on doit cocher 6 cases dans une grille comportant 49 numéros.
\begin{enumerate}
\item
  Quelle est la probabilité de gagner le gros lot (c'est-à-dire d'avoir les 6 bons numéros)?
\item
  On gagne quelque chose à partir du moment où l'on a au moins 3 bons numéros.
  Avec quelle probabilité cela arrive-t-il?
\item
  En fait, on peut aussi (en payant plus cher) cocher 7, 8, 9 ou même 10 numéros
  sur la grille. Dans chacun des cas, quelle est la probabilité de gagner le
  gros lot?
\end{enumerate}

\Exercice

On choisit au hasard un comité de quatre personnes parmi huit américains, cinq anglais et trois français. Quelle est la probabilité:
\begin{itemize}
\item
  qu'il ne se compose que d'américains?
\item
  qu'aucun américain ne figure dans ce comité?
\item
  qu'au moins un membre de chaque nation figure dans le comité?
\end{itemize}

\Exercice[\og{}paradoxe\fg{} des anniversaires]

Dans une classe de $n$ élèves, quelle est la probabilité pour que
deux étudiants au moins aient même anniversaire?

Quel est le nombre minimum de personnes dans le groupe pour que
cette probabilité soit d'au moins $50\%$? de $90\%$?

\Exercice

Alice, Bob, Charly et Denis jouent au bridge, et
reçoivent chacuns $13$ cartes d'un (même) jeu de $52$ cartes.

Sachant qu'Alice et Charly ont à eux deux $8$ Piques,
on en déduit que Bob et Denis ont $5$ Piques à eux deux.
Quelle est la probabilité pour que les Piques soient \og{}bien répartis\fg{},
c.-à-d. pour que la répartition des $5$ Piques soit $3-2$ ou $2-3$ entre Bob et Denis?

\Exercice[inclusion-exclusion]

Une école d'ingénieur propose à ses $320$ étudiants deux cours
de mathématiques, un en analyse, et un en probabilités.
On sait qu'il y a $140$ étudiants qui choisissent l'analyse,
$170$ qui n'assistent pas au cours de probabilités,
et $190$ qui suivent exactement un cours de mathématiques.

On choisit un étudiant au hasard. Quelle est la probabilité pour
qu'il ne suive pas les deux cours de mathématiques?
Qu'il suive au moins un cours de mathématiques?
Qu'il suive l'analyse mais pas les probabilités?

\Exercice

Un étudiant, Ulysse, sort habituellement le vendredi soir, avec une probabilité $2/3$.
Or ce jour-là, il y a justement un devoir de mathématiques le lendemain matin.
On suppose qu'Ulysse réussit à avoir la moyenne avec une probabilité
de $3/4$ s'il n'est pas sorti la veille, et de seulement $1/2$ dans le cas contraire.

Sachant qu'Ulysse n'a pas obtenu la moyenne, quelle est la probabilité pour qu'il soit sorti la veille?

\Exercice

Un nouveau test de dépistage d'une maladie rare et incurable, touchant
environ une personne sur $100000$, vient d'être mis au point.
Pour tester sa validité, on a effectué une étude statistique:
sur $534$ sujets sains, le test a été positif $1$ seule fois,
et, sur $17$ sujets malades, il a été positif $16$ fois.

Une personne effectue ce test, et le résultat est positif.
Quelle est la probabilité pour qu'elle soit atteinte de cette maladie?
Faut-il commercialiser ce test?

\Exercice

Loïc joue au Loto; s'il gagne, il part aux Seychelles pour un mois complet à
coup sûr. S'il perd, il ne partira probablement pas (seulement avec une
probabilité de $1/10000$).
\begin{enumerate}
\item
  Quelle est la probabilité pour que Loïc parte aux Seychelles demain?
\item
  Sachant qu'il est parti aux Seychelles, quelle est la probabilité qu'il ait gagné au Loto?
\end{enumerate}

\Exercice

Trois personnes (Alduire, Basilis et Cléophie)
jouent à la roulette russe
de la façon suivante: on fait tourner une fois le barillet au début,
puis chacun appuie sur la détente à son tour
(Alduire, puis Basilis, puis Cléophie).
Préféreriez-vous être à la place d'Alduire, de Basilis ou de Cléophie?

\Exercice

Trois machines $A$, $B$ et $C$ fournissent respectivement $50\%$, $30\%$
et $20\%$ de la production d'une usine. Les pourcentages de pièces
défectueuses sont respectivement de $3\%$, $4\%$ et $5\%$.
\begin{enumerate}
\item
  Quelle est la probabilité qu'une pièce, prise au hasard dans la production, soit défecteuse?
\item
  Quelle est la probabilité pour qu'une pièce défectueuse prise au hasard
  provienne de $A$? de $B$? de $C$?
\end{enumerate}

\Exercice

Alice et Bob jouent aux fléchettes.
À chaque manche, Alice gagne avec une probabilité $p$.
La partie se déroule en deux manches gagnantes.
Quelle est la probabilité $p'$ pour que Alice gagne?

Quand a-t-on $p' < p$, $p' = p$, $p' > p$? Commenter le résultat.

\Exercice[Monty Hall]

Hildebert joue à un jeu télévisé.
Il a face à lui, trois portes ($A$, $B$ et $C$) identiques;
derrière l'une d'entre elle se trouve le gros lot, mais derrière
les deux autres, rien du tout.

Hildebert choisit une des portes (disons, la $A$), et alors le
présentateur (qui connaît la porte gagnante) ouvre une autre porte
(disons, la $C$) et montre à tous qu'il n'y a rien derrière
celle-ci (la porte $C$).
Il demande ensuite à Hildebert s'il préfère rester sur son choix
ou s'il veut changer.

Hildebert, se disant que cela ne fait aucune différence,
reste sur son choix (la porte $A$).
A-t-il raison d'agir ainsi?

\Exercice

Soit $\Prob$ un espace probabilisé fini tel que $\CardΩ$ est un nombre premier $p$ et $ℙ$ est la probabilité uniforme.
Soit $A$ et $B$ deux événements non triviaux.
Montrer que $A$ et $B$ ne sont pas indépendants.

\Exercice

Un couple a deux enfants dont une fille.
Quelle est la probabilité que l'autre enfant soit un garçon?

\Exercice

On considère un jeu de $52$ cartes.
Après avoir mélangé les cartes, quelle est la probabilité que
\begin{enumerate}
\item
  l'as de coeur soit placé avant l'as de pique?
\item
  les as de coeur et de pique soient voisins?
\end{enumerate}

\Exercice[la chaîne des menteurs]

On suppose qu'un message binaire ($0$ ou $1$) est transmis depuis un emetteur $M_0$ à travers une chaîne $M_1, M_2, \dots, M_n$ de messagers menteurs, qui transmettent correctement le message avec une probabilité $p$, mais qui changent sa valeur avec une probabilité $1-p$.

Si l'on note $a_n$ la probabilité que l'information transmise par $M_n$ soit identique à celle envoyée par $M_0$ (avec comme convention que $a_0=1$), déterminer $a_{n+1}$ en fonction de $a_n$, puis une expression explicite de $a_n$ en fonction de $n$, ainsi que la valeur limite de la suite $(a_n)_{n∈ℕ}$. Le résultat est-il conforme à ce à quoi l'on pouvait s'attendre?

\Exercice[urne de Polya]

Soit $(a,b,c) ∈ ℕ^3$ avec $a+b>0$.
On considère une urne contenant $a$ boules blanches et $b$ boules noires.
Après chaque tirage, on remet la boule choisie dans l'urne avec $c$ nouvelles boules de la même couleur.
Déterminer la probabilité que la $n$-ième boule tirée soit blanche.

\end{document}
