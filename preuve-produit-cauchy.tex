\documentclass{yann}
\usepackage{tikz}
\yann{layout=onecolumn,displaystyle}

\begin{document}
\title{Preuve: produit de Cauchy}
\maketitle

Soit $(a_n)_{n∈ℕ}$ et $(b_n)_{n∈ℕ}$ deux suites numériques telles que
les séries $∑_n a_n$ et $∑_n b_n$ sont absolument convergentes.
On note $(c_n)_{n∈ℕ}$ la suite définie par
$∀n∈ℕ$, $c_n = ∑_{k=0}^n a_k b_{n-k}$.
On cherche à montrer que $∑_n c_n$ est absolument convergente et que
\[ ∑_{n=0}^{+∞} c_n = \Pa{ ∑_{n=0}^{+∞} a_n } \Pa{ ∑_{n=0}^{+∞} b_n }. \]

\section{Cas positif}

On commence par traiter le cas où $(a_n)$ et $(b_n)$ sont à valeurs positives.
Introduisons quelques notations:
\begin{itemize}
\item
$A_n = ∑_{k=0}^n a_k$, $B_n = ∑_{k=0}^n b_k$ et $C_n = ∑_{k=0}^n c_k$;
\item
$A = ∑_{n=0}^{+∞} a_n$ et $B = ∑_{n=0}^{+∞} b_n$;
\item
$Δ_n = \Ensemble{(i,j)∈ℕ^2}{i+j≤n}$.
\item
$□_n = \Ensemble{(i,j)∈ℕ^2}{i≤n \text{ et } j≤n}$
\end{itemize}

On vérifie que, pour tout entier $n∈ℕ$, on a:
\begin{gather}
  Δ_n ⊂ □_n ⊂ Δ_{2n} \label{eq:subset} \\
  A_n B_n = ∑_{(i,j)∈□_n}{a_i b_j} \label{eq:square} \\
  C_n = ∑_{(i,j)∈Δ_n}{a_i b_j} \label{eq:triangle}
\end{gather}

\begin{center}
  \begin{tikzpicture}
    \def\fig{
      \draw[->,thick,>=stealth] (0,0) -- (5.5,0) ;
      \draw[->,thick,>=stealth] (0,0) -- (0,5.5) ;
      \foreach \i in {0,...,5} {
        \foreach \j in {0,...,5} {
          \filldraw (\i, \j) circle (1pt) ;
        }
      }
    }
    \begin{scope}
      \fill[gray!50] (0,0) -- (4.05,0) -- (0,4.05);
      \fig
      \node[above] at (2,0) {$Δ_4$} ;
    \end{scope}
    \begin{scope}[xshift=8cm]
      \fill[gray!50] (0,0) rectangle (4.05,4.05) ;
      \fig
      \node[above] at (2,0) {$□_4$} ;
    \end{scope}
  \end{tikzpicture}
\end{center}

Si $(i,j)∈Δ_n$, alors $i+j≤n$, donc $i≤i+j≤n$ et de même $j≤i+j≤n$, donc $(i,j)∈□_n$.
De plus, si $(i,j)∈□_n$, alors $i+j≤n+n=2n$, donc $(i,j)∈Δ_{2n}$.
Cela prouve~(\ref{eq:subset}).

La relation~(\ref{eq:square}) vient du fait que
\[ A_n B_n = \Pa{ ∑_{i=0}^n a_i } \Pa{ ∑_{j=0}^n b_j } = ∑_{0≤i,j≤n} a_i b_j. \]

Enfin, commme \( c_k = ∑_{\substack{(i,j)∈ℕ^2 \\ i+j=k}} a_i b_j \),
on a la relation~(\ref{eq:triangle}) car
\[ C_n = ∑_{k=0}^n c_k = ∑_{k≤n} \, ∑_{i+j=k} a_i b_j
= ∑_{i+j≤n} a_i b_j
= ∑_{(i,j)∈Δ_n} a_i b_j. \]

Pour $n∈ℕ$, notons $p=\floor{n/2}$. Comme $□_p ⊂ Δ_n ⊂ □_n$, il vient
\[ ∑_{(i,j)∈□_p} a_i b_j ≤ ∑_{(i,j)∈Δ_n} a_i b_j ≤ ∑_{(i,j)∈□_n} a_i b_j \]
car les $a_i b_j ≥ 0$.
Autrement dit, on a $A_p B_p ≤ C_n ≤ A_n B_n$.
En faisant tendre $n\to+∞$, le théorème des gendarmes assure que $C_n \to AB$, ce qui est exactement le résultat souhaité.
L'absolue convergence de $∑ c_n$ est immédiate car $c_n≥0$ pour tout $n∈ℕ$.

\section{Cas général}

On garde les mêmes notations, en ajoutant:
\begin{itemize}
\item
$\tilde a_n = \Abs{a_n}$, $\tilde b_n = \Abs{b_n}$
  et $\tilde c_n = ∑_{k=0}^n \tilde a_k \tilde b_{n-k}$.
  Notons que $\tilde c_n ≠ \Abs{c_n}$ en général;
\item
$\tilde A_n = ∑_{k=0}^n \tilde a_k$ et de même pour $\tilde B_n$ et $\tilde C_n$.
\end{itemize}

Comme les séries $∑ \Abs{a_n}$ et $∑ \Abs{b_n}$ sont absolument convergentes et positives,
le cas positif montre que la série de terme général
$\tilde c_n$ est également absolument convergente.
Or d'après l'inégalité triangulaire, $\Abs{c_n} ≤ \tilde c_n$,
donc la série $∑ c_n$ est également absolument convergente.
De plus,
\[ A_n B_n - C_n
= ∑_{(i,j)∈□_n} a_i b_j - ∑_{(i,j)∈Δ_n} a_i b_j
= ∑_{(i,j)∈□_n∖Δ_n} a_i b_j \]
donc
\[ \Abs{A_n B_n - C_n}
≤ ∑_{(i,j)∈□_n∖Δ_n} \Abs{a_i} ⋅ \Abs{b_j}
= ∑_{(i,j)∈□_n} \Abs{a_i} ⋅ \Abs{b_j} - ∑_{(i,j)∈Δ_n} \Abs{a_i} ⋅ \Abs{b_j}
= \tilde A_n \tilde B_n - \tilde C_n \]

Or d'après le cas positif, $\tilde A_n \tilde B_n - \tilde C_n \to 0$,
d'où $A_n B_n - C_n \to 0$, ce qui est exactement le résultat cherché.

CQFD

\end{document}
