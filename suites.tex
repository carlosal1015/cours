\documentclass{yann}

\undef\U \undef𝕍
\newcommand\I{i}
\newcommand\U{(u_n)_{n∈ℕ}}
\newcommand𝕍{(v_n)_{n∈ℕ}}

\begin{document}
\title{Suites numériques}
\maketitle

% -----------------------------------------------------------------------------
\section{Généralités}

\Para{Définitions}
\begin{itemize}
\item
  Une \emph{suite} à valeurs dans un ensemble $X$ est une application $\Fn{u}ℕ{X}$.
  On note fréquemment $u_n$ au lieu de $u(n)$ et $\U$ au lieu de $u$.
\item
  On dit que $u_n$ est le \emph{terme général} de la suite $\U$.
\item
  Une \emph{suite numérique} est une suite à valeurs dans $ℝ$ ou $ℂ$.
\item
  Une \emph{suite réelle} est une suite à valeurs dans $ℝ$.
\item
  Une suite réelle $\U$ est dite \emph{croissante} si et seulement si
  $∀n∈ℕ$, $u_n≤u_{n+1}$.
\item
  Une suite réelle $\U$ est dite \emph{décroissante} si et seulement si
  $∀n∈ℕ$, $u_n≥u_{n+1}$.
\item
  Une suite numérique $\U$ est dite \emph{stationnaire} £ssil. existe
  $c ∈ 𝕂$ et $n_0 ∈ ℕ$ tels que $∀n ≥n_0$, $u_n = c$.
\end{itemize}

\Para{Définitions}
\begin{itemize}
\item
  Soit $\U$ une suite numérique et $ℓ∈𝕂$.
  On dit que la suite $\U$ \emph{converge} vers $ℓ$ si et seulement si
  \[ ∀ε> 0 \+∃N∈ℕ\+∀n≥N\+
  \Abs{u_n -ℓ}≤ε. \]
  $ℓ$ est nécessairement unique et s'appelle \emph{la limite} de la suite $\U$.
  On écrit $\lim\limits_{\ninf} u_n = ℓ$ ou encore $u_n \Toninfℓ$.
\item
  Une suite numérique $\U$ est dite \emph{convergente} si et seulement s'il
  existe $ℓ∈𝕂$ tel que $\U$ converge vers $ℓ$.
  Elle est dite \emph{divergente} dans le cas contraire.
\end{itemize}

\Para{Définitions}

Soit $\U$ une suite réelle.
\begin{itemize}
\item
  On dit que \emph{$u_n \Toninf +∞$} si et seulement si
  \[ ∀M∈ℝ\+∃N∈ℕ\+∀n≥N\+ u_n≥M. \]
\item
  On dit que \emph{$u_n \Toninf -∞$} si et seulement si
  \[ ∀M∈ℝ\+∃N∈ℕ\+∀n≥N\+ u_n≤M. \]
\end{itemize}

\Para{Remarque}

Soit $\U$ une suite réelle telle que $u_n \Toninf±∞$.
Alors $\U$ est une suite \emph{divergente}.

\Para{Définitions}
\begin{itemize}
\item
  Une suite réelle $\U$ est dite \emph{majorée} si et seulement si
  \[ ∃A∈ℝ\+∀n∈ℕ\+ u_n≤A. \]
\item
  Une suite réelle $\U$ est dite \emph{minorée} si et seulement si
  \[ ∃B∈ℝ\+∀n∈ℕ\+ u_n≥B. \]
\item
  Une suite numérique $\U$ est dite \emph{bornée} si et seulement si
  \[ ∃M∈ℝ\+∀n∈ℕ\+ \Abs{u_n}≤M. \]
\end{itemize}

\Para{Définitions}
\begin{itemize}
\item
  Une \emph{extractrice} est une fonction $ℕ\toℕ$ strictement croissante.
\item
  Une \emph{sous-suite} ou \emph{suite extraite} de la suite $\U$
  est une suite de la forme $(u_{σ(n)})_{n∈ℕ}$ où $σ$
  est une extractrice.
\end{itemize}

\Para{Proposition}

Une sous-suite d'une suite convergente est également convergente.

Plus précisément:
Soit $\U$ une suite numérique et $ℓ∈𝕂$ tels que $u_n \Toninfℓ$.
Soit $σ$ une extractice.
Alors $u_{σ(n)} \Toninfℓ$.

\subsection{Théorèmes}

\Para{Théorème}

Toute suite réelle croissante majorée converge.

Autrement dit:
Soit $\U$ une suite réelle croissante et majorée.
Alors $\U$ est convergente.

\Para{Corollaire}

Toute suite réelle décroissante minorée converge.

\Para{Proposition}

Toute suite réelle croissante non majorée tend vers $+∞$.

\Para{Théorème des gendarmes}

Soit $(u_n)_{n∈ℕ}$, $(v_n)_{n∈ℕ}$ et $(w_n)_{n∈ℕ}$ trois suites réelles et $ℓ∈ℝ$.
Si:
\begin{itemize}
\item
  $u_n \Toninfℓ$;
\item
  $w_n \Toninfℓ$;
\item
  $∃N∈ℕ \+ ∀n≥N \+ u_n≤v_n≤w_n$.
\end{itemize}

Alors $(v_n)_{n∈ℕ}$ est une suite convergente et $v_n \Toninfℓ$.

\Para{Théorème}

Soit $\U$ une suite réelle.
Soit $I$ un intervalle \emph{ouvert} de $ℝ$ et $\Fn fIℝ$.
Si:
\begin{itemize}
\item
  $u_n \Toninfℓ$;
\item
  $ℓ∈I$;
\item
  $f$ continue en $ℓ$.
\end{itemize}

Alors la suite $\bigl( f(u_n) \bigr)_n$ est définie à partir d'un certain rang et
$f(u_n) \Toninf f(ℓ)$.

\Para{Théorème de Bolzano-Weierstraß}[hors-programme]

Toute suite réelle bornée admet une sous-suite convergente.

Autrement dit:

Soit $\U$ une suite réelle \emph{bornée}.
Alors il existe une extractrice $σ$
et un réel $ℓ$ tels que $u_{σ(n)} \Toninfℓ$.

\Para{Définition}

Soit $\U$ et $𝕍$ deux suites réelles.
On dit qu'elles sont \emph{adjacentes} si et seulement si:
\begin{itemize}
\item
  $\U$ est croissante;
\item
  $𝕍$ est décroissante;
\item
  $v_n - u_n \Toninf 0$.
\end{itemize}

\Para{Théorème}

Soit $\U$ et $𝕍$ deux suites adjacentes.
Alors elles convergent toutes les deux vers la même limite $ℓ∈ℝ$.
De plus:
\begin{multline*}
  u_0≤\cdots≤u_n≤u_{n+1}≤\cdots≤ℓ \\
  ℓ≤\cdots≤v_{n+1}≤v_n≤\cdots≤v_0
\end{multline*}

\Para{Théorème de Cesàro}

Soit $\U$ une suite numérique telle que $u_n \Toninfℓ∈𝕂$.
On pose $m_n = \frac{1}{n}∑_{k=0}^{n-1} u_k$;
il s'agit de la moyenne arithmétique des $n$ premiers termes de la suite $\U$.

Alors $m_n \Toninfℓ$.

\Para{Remarque}

Si $\U$ est une suite réelle qui tend vers $±∞$, le résultat est encore vrai.

\subsection{Comparaison}

\Para{Définitions}

Soit $\U$ et $𝕍$ deux suites numériques.
\begin{itemize}
\item
  On dit que \emph{$u_n = \GrandO_\ninf(v_n)$} si et seulement s'il existe
  une suite numérique $(e_n)_{n∈ℕ}$ \emph{bornée} telle que
  $∃N∈ℕ,∀n≥N$, $u_n = e_n v_n$.
\item
  On dit que \emph{$u_n = \PetitO_\ninf(v_n)$} si et seulement s'il existe
  une suite numérique $(e_n)_{n∈ℕ}$ \emph{qui tend vers 0} telle que
  $∃N∈ℕ,∀n≥N$, $u_n = e_n v_n$.
\item
  On dit que \emph{$u_n \Sim_\ninf v_n$} si et seulement s'il existe
  une suite numérique $(e_n)_{n∈ℕ}$ \emph{qui tend vers 1} telle que
  $∃N∈ℕ,∀n≥N$, $u_n = e_n v_n$.
\end{itemize}

\Para{Proposition}

Soit $\U$ et $𝕍$ deux suites numériques.
Alors $u_n \Sim_\ninf v_n$ si et seulement si $u_n = v_n + \PetitO_\ninf(v_n)$.

\Para{Proposition}

Soit $\U$ et $𝕍$ deux suites numériques.
On suppose de plus que:
\[ ∃N∈ℕ\+∀n≥N\+ v_n≠0. \]
Alors:
\begin{itemize}
\item
  $u_n = \GrandO_\ninf(v_n)$ si et seulement si la suite $\left(\frac{u_n}{v_n}\right)_{n≥N}$ est bornée.
\item
  $u_n = \PetitO_\ninf(v_n)$ si et seulement si $\frac{u_n}{v_n} \Toninf 0$.
\item
  $u_n \Sim_\ninf v_n$ si et seulement si $\frac{u_n}{v_n} \Toninf 1$.
\end{itemize}

% -----------------------------------------------------------------------------
\section{Suites récurrentes du premier ordre}

\Para{Définition}

Soit $\U$ une suite à valeurs dans un ensemble $X$
et $\Fn fXX$.

On dit que $\U$ est une \emph{suite récurrente (du premier ordre)}
associée à la fonction $f$
si et seulement si \[ ∀n∈ℕ, u_{n+1} = f(u_n). \]

\Para{Proposition}

Soit $\U$ une suite récurrente associée à $f$.
Si:
\begin{itemize}
\item
  $u_n \Toninfℓ$;
\item
  $ℓ∈X$;
\item
  $f$ continue en $ℓ$.
\end{itemize}

Alors $f(ℓ) =ℓ$.

\Para{Remarque}

Cela sert à trouver les limites \emph{potentielles} de $\U$.

\Para{Définition}

Soit $\Fn fXX$ et $Y⊂X$.
On dit que $Y$ est \emph{stable} par $f$ si et seulement si $f(Y)⊂Y$.

\Para{Proposition}

Soit $\U$ une suite récurrente réelle associée à $f$
et $I$ un intervalle stable par $f$ tel que $u_{n_0}∈I$.
Alors $∀n≥n_0$, $u_n∈I$.

\Para{Proposition}

Soit $\Fn{f}{I}{I}$ où $I$ est un intervalle de $ℝ$
et $\U$ une suite récurrente associée à $f$.
\begin{enumerate}
\item
  Si $f$ est croissante sur $I$, alors $\U$ est monotone;
\item
  Si $f$ est décroissante sur $I$, alors les suites extraites
  $(u_{2n})_{n∈ℕ}$ et $(u_{2n+1})_{n∈ℕ}$ sont monotones;
\item
  Dans le cas général, on ne peut rien dire: $\U$ peut avoir un comportement très complexe.
\end{enumerate}

\Para{Exemples}
\begin{itemize}
\item
  \emph{Suites arithmétiques:}
  $∀n∈ℕ\+ u_{n+1} = u_n + b$.
  On a alors $∀n∈ℕ\+ u_n = u_0 + nb$.
\item
  \emph{Suites géométriques:}
  $∀n∈ℕ\+ u_{n+1} = a u_n$.
  On a alors $∀n∈ℕ\+ u_n = a^n u_0$.
\item
  \emph{Suites arithmético-géométriques:}
  $∀n∈ℕ\+ u_{n+1} = au_n + b$ où $a≠1$.
  On pose $ℓ$ tel que $ℓ= aℓ+ b$, puis $v_n = u_n -ℓ$.
  On vérifie alors que $𝕍$ est une suite géométrique.
\item
  \emph{Récurrence homographique:}
  $∀n∈ℕ\+ u_{n+1} = \frac{a u_n + b}{c u_n + d}$.
  Soit $A = \begin{pmatrix} a & b \\ c & d \end{pmatrix}$.
  On calcule $A^n = \begin{pmatrix} α_n & β_n \\ γ_n & δ_n \end{pmatrix}$
  et l'on a:
  \[ ∀n∈ℕ\+
  u_n = \frac{α_n u_0 +β_n}{γ_n u_0 +δ_n}. \]
\end{itemize}

\subsection{Suites récurrentes du second ordre}

\Para{Définition}

Soit $\U$ une suite à valeurs dans un ensemble $X$,
et $\Fn{f}{X^2}{X}$.
On dit que $\U$ est une \emph{suite récurrente du second ordre} associée à la fonction $f$
si et seulement si \[ ∀n∈ℕ, u_{n+2} = f(u_n, u_{n+1}). \]

\Para{Proposition}

Soit $\U$ une suite récurrente du second ordre associée à $f$.
Si:
\begin{itemize}
\item
  $u_n \Toninfℓ$;
\item
  $ℓ∈X$;
\item
  $f$ continue en $(ℓ,ℓ)$.
\end{itemize}

Alors $f(ℓ,ℓ) =ℓ$.

\subsection{Cas des suites récurrentes linéaires du second ordre à coefficients constants sans second membre}

\Para{Proposition}

Soit $\U$ une suite numérique et $(a,b,c)∈𝕂^3$ avec
$a≠0$ et $(b,c)≠(0,0)$
tels que \[ ∀n∈ℕ\+ au_{n+2} + bu_{n+1} + c u_n = 0. \]

On forme l'\emph{équation caractéristique}:
\[ ar^2 + br + c = 0. \tag{EC} \]

Si \emph{$𝕂=ℂ$}, il y a deux cas:
\begin{enumerate}
\item
  Soit $Δ≠0$:
  (EC) admet deux racines distinctes $r_1$ et $r_2$, et
  \[ ∃(A,B)∈ℂ^2\+∀n∈ℕ\+ u_n = A r_1^n + B r_2^n \]
\item
  Soit $Δ= 0$:
  (EC) admet une racine double $r_0≠0$, et
  \[ ∃(A,B)∈ℂ^2\+∀n∈ℕ\+ u_n = (An + B) r_0^n \]
\end{enumerate}

Si \emph{$𝕂=ℝ$}, il y a trois cas:
\begin{enumerate}
\item
  Soit $Δ> 0$:
  (EC) admet deux racines distinctes $r_1$ et $r_2$, et
  \[ ∃(A,B)∈ℝ^2\+∀n∈ℕ\+ u_n = A r_1^n + B r_2^n \]
\item
  Soit $Δ= 0$:
  (EC) admet une racine double $r_0≠0$, et
  \[ ∃(A,B)∈ℝ^2\+∀n∈ℕ\+ u_n = (An + B) r_0^n \]
\item
  Soit $Δ< 0$:
  (EC) admet deux racines complexes conjuguées $ρe^{±\Iθ}$, et
  \[ ∃(A,B)∈ℝ^2\+∀n∈ℕ\+ u_n =ρ^n \Big( A \cos(nθ) + B \sin(nθ) \Big) \]
\end{enumerate}

% -----------------------------------------------------------------------------
\section{Exercices}

\Exercice

Déterminer la limite, ou montrer la divergence des suites $\U$ définies par
\begin{enumerate}
\item
  $u_n = \frac{3^n-(-2)^n}{3^n+(-2)^n}$
\item
  $u_n = √{n^2+n+1} - √{n^2-n+1}$
\item
  $u_n = \frac{n-√{n^2+1}}{n+√{n^2+1}}$
\item
  $u_n = \frac{1}{n^2} ∑_{k=1}^n k$
\item
  $u_n = \bigPa{1+\frac1n}^n$
\item
  $u_n = √[n]{n^2}$
\item
  $u_n = \bigPa{\sin\frac1n}^{1/n}$
\item
  $u_n = \bigPa{\frac{n-1}{n+1}}^n$
\item
  $u_n = \frac{\sin n}{n+(-1)^{n+1}}$
\item
  $u_n = \frac{n!}{n^n}$
\item
  $u_n = \frac{n-(-1)^n}{n+(-1)^n}$
\item
  $u_n = \frac{e^n}{n^n}$
\item
  $u_n = √[n]{2+(-1)^n}$
\item
  $u_n = √[n]{n}$
\item
  $u_n = \bigPa{1+\frac xn}^n$
\item
  $u_n = \bigPa{\frac{n-1}{n+1}}^{n+2}$
\item
  $u_n = n^2 \bigPa{\cos\frac{1}{n} - \cos\frac{1}{n+1}}$
\item
  $u_n = \bigPa{\tan\pa{\fracπ4 + \fracα{n}}}^n$
\item
  $u_n = \bigPa{\frac{\ln(n+1)}{\ln(n)}}^{n\ln n}$
\item
  $u_n = \bigPa{\frac{\arctan(n+1)}{\arctan(n)}}^{n^2}$
\item
  $u_n = \cos\bigPa{πn^2 \ln(1-1/n)}$
\end{enumerate}

\Exercice

Montrer que l'ensemble des entiers naturels $n$ tels que $2^{n^2} < (4n)!$ est fini.

\Exercice

Soit $\U$ une suite réelle convergeant vers $ℓ∈ℝ$.
Si $\floor⋅$ désigne la fonction partie entière,
la suite $\bigPa{\floor{u_n}}_{n∈ℕ}$ est-elle convergente?

\Exercice

Trouver un exemple de suite qui diverge mais dont la moyenne de Cesàro converge.

\Exercice

Étudier la convergence des suites réelles $\U$ définies par:
\begin{enumerate}
\item
  $u_0 = a > 0, u_{n+1} = \frac12\left(u_n + \frac{a}{u_n}\right)$
\item
  $u_0 = 1$, $u_{n+1} = \frac{u_n}{u_n^2+1}$
\item
  $u_0≥0$, $u_{n+1} = \frac16(u_n^2+8)$
\item
  $u_0 = 1$, $u_{n+1} = \frac{1}{2+u_n}$
\item
  $u_{n+1} = \cos u_n$
\item
  $0 < u_0 < \frac{√5-1}{2}, u_{n+1} = 1 - u_n^2$
\item
  $u_{n+1} = u_n - u_n^2$
\item
  $u_{n+1} = u_n + \frac{1+u_n}{1+2u_n}$
\item
  $0≤u_0≤1$, $u_{n+1} = \frac{√{u_n}}{√{u_n}+√{1-u_n}}$
\item
  $u_{n+1} =√{2-u_n}$
\end{enumerate}

\Exercice
\begin{enumerate}
\item
  \begin{enumerate}
  \item
    Résoudre $u_0 = a$, $u_1 = b$ et $u_{n+2} = \frac{u_{n+1} + u_n}{2}$.
  \item
    Déterminer $\lim_\ninf u_n$.
  \end{enumerate}
\item
  \begin{enumerate}
  \item
    Résoudre $v_0 =α> 0$, $v_1 =β> 0$ et $v_{n+2} =√{ v_{n+1} v_n }$
  \item
    Déterminer $\lim_\ninf v_n$.
  \end{enumerate}
\end{enumerate}

\Exercice

Soit $\U$ et $𝕍$ deux suites réelles vérifiant
\[ \left\{ \begin{array}{l}
    0≤u_n≤1, \\
    0≤u_n≤1, \\
    u_n v_n \Toninf 1.
\end{array} \right. \]
Que peut-on dire de ces deux suites?

\Exercice

Pour $n∈ℕ$, on pose
$u_n =∑_{k=0}^n \frac{1}{k!}$ et
$v_n = u_n + \frac{1}{n⋅n!}$.
\begin{enumerate}
\item
  Montrer que $\U$ et $𝕍$ sont deux suites adjacentes;
  on note $ℓ$ leur limite commune.
\item
  Déterminer un $n$ tel que $v_n - u_n≤10^{-7}$.
\item
  En déduire une valeur approchée de $ℓ$ à $10^{-7}$ près.
\end{enumerate}

\Exercice

Trouver $\lim_\ninf√{1+√{1+ \cdots +√{1}}}$ (avec $n$ radicaux).

\Exercice

Étudier la convergence des suites $\U$ et $𝕍$ définies par
$0 < u_0 < v_0$,
$u_{n+1} = \frac{2u_n+v_n}{3}$ et
$v_{n+1} = \frac{2v_n+u_n}{3}$.

\Exercice

Soit $\U$ et $𝕍$ les suites réelles définies par
$u_0 = a > 0$,
$v_0 = b > 0$,
$u_{n+1} =√{u_n v_n}$ et
$v_{n+1} = \frac{u_n+v_n}{2}$.

Montrer que $\U$ et $𝕍$ sont des suites adjacentes.

\emph{Remarque:}
leur limite commune $M(a,b)$ s'appelle la \emph{moyenne arithmético-géométrique} de $a$ et $b$.
On peut prouver que:
\[ \fracπ{M(a,b)} =∫_{-∞}^{+∞} \frac{\D x}{√{(x^2+a^2)(x^2+b^2)}} \]

\Exercice

Soit $\U$ et $𝕍$ deux suites réelles telles que $u_n^2 + u_n v_n + v_n^2 \to 0$.
Que dire de $\U$ et $𝕍$?

\Exercice

Soit $\U$ une suite à valeurs positives et $λ ∈ \intO{0,1}$.
\begin{enumerate}
\item
  On suppose que $u_{n+1} ≤ λu_n$ pour tout entier~$n$.
  Montrer que $u_n \to 0$.
\item
  On suppose qu'il existe une suite réelle $𝕍$ tendant vers~0 telle que $u_{n+1} ≤ λ (u_n + v_n)$ pour tout entier~$n$.
  A-t-on $u_n \to 0$?
\end{enumerate}

\Exercice

Soit $\U$ une suite à valeurs strictement positives
telle que $\frac{u_{n+1}}{u_n} \to ℓ$.
\begin{enumerate}
\item
  Si $ℓ < 1$, montrer que $u_n \to 0$.
\item
  Si $ℓ > 1$, montrer que $u_n \to +∞$.
\item
  Montrer qu'on ne peut pas conclure dans le cas $ℓ = 1$.
\end{enumerate}

\Exercice

Soit $\U$ une suite injective à valeurs dans $ℕ$.
Montrer que $u_n \to +∞$.

\Exercice[lemme des petits pas]

Soit $\U$ une suite numérique telle que $u_{n+1} - u_n \to ℓ$.
Montrer que $u_n = n ℓ + o(n)$ quand $n \to +∞$.

\end{document}
