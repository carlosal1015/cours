\documentclass{yann}
\yann{layout=onecolumn}

\begin{document}
\title{Les quantificateurs}
\maketitle

\Para*{Notations et rappels}

Soit $P$ et $Q$ deux propositions.
\begin{itemize}
\item
$\neg P$ est la négation de $P$ c.-à-d. «non $P$».
\item
$P \vee Q$ est la disjonction de $P$ et de $Q$ £cad. «$P$ ou $Q$».
\item
$P \wedge Q$ est la conjonction de $P$ et de $Q$ £cad. «$P$ et $Q$».
\item
$P \implies Q$ est équivalent à $(\neg P) \vee Q$.
\item
$∀x∈∅, P(x)$ est toujours vraie.
\item
$∃x∈∅, P(x)$ est toujours fausse.
\item
$∀x∈A$, $P(x)$ est équivalent à $∀x \bigPa{x∈A \implies P(x)}$
\end{itemize}

\Exercice

Soit $X$ et $Y$ deux ensembles (non vides).
Déterminer le sens des phrases suivantes.
\begin{enumerate}
\item
$∀x∈X$, $∀y∈Y$, $f(x) = y$
\item
$∀x∈X$, $∃y∈Y$, $f(x) = y$
\item
$∃x∈X$, $∀y∈Y$, $f(x) = y$
\item
$∃x∈X$, $∃y∈Y$, $f(x) = y$
\item
$∀y∈Y$, $∀x∈X$, $f(x) = y$
\item
$∃y∈Y$, $∀x∈X$, $f(x) = y$
\item
$∀y∈Y$, $∃x∈X$, $f(x) = y$
\item
$∃y∈Y$, $∃x∈X$, $f(x) = y$
\item
$∀x∈X$, $∀x'∈X$, $x = x' \vee f(x)≠f(x')$
\item
$∃x∈X$, $∀x'∈X$, $f(x) = f(x') \implies x = x'$
\item
$∀x∈X$, $∃x'∈X$, $f(x) = f(x') \implies x = x'$
\end{enumerate}

\Exercice

Soit $\Fn fℝℝ$.
Déterminer le sens des phrases suivantes.
\begin{enumerate}
\item
$∀x∈ℝ$, $∀y∈ℝ$, $x≤y \implies f(x)≤f(y)$
\item
$∀x∈ℝ$, $∃y∈ℝ$, $f(x)≤y$
\item
$∃y∈ℝ$, $∀x∈ℝ$, $f(x)≤y$
\item
$∀x∈ℝ$, $∀y∈ℝ$, $x>y \vee f(x)≥f(y)$
\end{enumerate}

\Exercice

Soit $\Fn fℝℝ$.
Déterminer le sens des phrases suivantes.
\begin{enumerate}
\item
$∀x∈ℝ$, $∀y>0$, $∃z>0$, $∀t∈ℝ$, $\Abs{x-t}<z \implies \Abs{f(x)-f(t)}<y$
\item
$∀x∈ℝ$, $∀y≥0$, $∃z>0$, $∀t∈ℝ$, $\Abs{x-t}<z \implies \Abs{f(x)-f(t)}<y$
\item
$∀x∈ℝ$, $∀y>0$, $∃z≥0$, $∀t∈ℝ$, $\Abs{x-t}<z \implies \Abs{f(x)-f(t)}<y$
\item
$∀x∈ℝ$, $∀y>0$, $∃z>0$, $∀t∈ℝ$, $\Abs{x-t}≤z \implies \Abs{f(x)-f(t)}<y$
\item
$∀x∈ℝ$, $∀y>0$, $∃z>0$, $∀t∈ℝ$, $\Abs{x-t}<z \implies \Abs{f(x)-f(t)}≤y$
\item
$∀x∈ℝ$, $∀y≥0$, $∃z>0$, $∀t∈ℝ$, $\Abs{x-t}<z \implies \Abs{f(x)-f(t)}≤y$
\end{enumerate}

\end{document}
