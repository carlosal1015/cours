\documentclass{yann}

\newcommand{\Part}{\mathcal{P}}
\newcommand{\Pro}{\bigl(Ω, \Part(Ω)\bigr)}
\newcommand{\Prob}{\bigl(Ω, \Part(Ω), ℙ\bigr)}

\begin{document}
\title{Variables aléatoires dans un univers fini}
\maketitle

% -----------------------------------------------------------------------------
\section{Généralités}

\Para{Définition}[variable aléatoire]

Soit $\Prob$ un espace probabilisé fini.
Une \emph{variable aléatoire} est une application $\Fn{X}{Ω}{E}$.
Lorsque $E⊂ℝ$, on parle de \emph{variable aléatoire réelle}.

\Para{Définition}[fonction d'une variable aléatoire]

Soit $\Prob$ un espace probabilisé.
Soit $X$ une variable aléatoire à valeurs dans $E$.
Soit $\Fn{f}{E}{F}$ une application quelconque.
L'application
\[ \Fonction{f◦X}{Ω}{F}{ω}{f(X(ω))} \]
est une variable aléatoire à valeurs dans $F$.
L'usage veut qu'on la note abusivement $f(X)$ au lieu de $f◦X$.

\Para{Définition}[événements associés à une variable aléatoire]

Soit $\Prob$ un espace probabilisé.
Soit $X$ une variable aléatoire à valeurs dans $E$.
Pour tout $A⊂E$, on définit l'événement $\{ X ∈A \}$ comme étant
\[ X^{-1}(A) = \Ensemble{ω∈Ω}{X(w) ∈A}. \]
\begin{itemize}
\item
  On notera plus simplement \og{}$X∈A$\fg{} pour \og{}$\{ X∈A \}$\fg{}.
\item
  Si $A = \Acco{a}$, on notera \og{}$X=a$\fg{} pour \og{}$\{ X∈A \}$\fg{}.
\item
  Si $E = ℝ$ et $A = \intOF{-∞,a}$, on notera \og{}$X≤a$\fg{} pour \og{}$\{ X∈A \}$\fg{}, etc.
\end{itemize}

\Para{Définition}[loi d'une variable aléatoire]

Soit $\Prob$ un espace probabilisé fini.
Soit $X$ une variable aléatoire à valeurs dans l'ensemble $E$ fini.
L'application
\[ \Fonction{ℙ_X}{\Part(E)}{[0,1]}{A}{ℙ(X∈A)} \]
est une probabilité sur $E$, appelée \emph{loi de la variable $X$}.

\Para{Proposition}

Soit $\Prob$ un espace probabilisé fini.
Soit $X$ une variable aléatoire à valeurs dans $E$.
La loi de $X$ est uniquement déterminée par les valeurs $ℙ(X=x)$ pour $x∈X(Ω)$.

% -----------------------------------------------------------------------------
\section{Lois usuelles}

\Para{Définition}[loi uniforme]

Une variable $X$ à valeurs dans $E$ fini suit la loi uniforme sur $E$
si pour tout $x∈E$, \[ ℙ(X=x) = \frac{1}{\Card{E}}. \]
On note $X \sim \mathscr{U}(E)$.

\Para{Définition}[loi de Bernoulli]

Une variable $X$ à valeurs dans $\{0,1\}$ suit la loi de Bernoulli de paramètre $p$
si \[ ℙ(X=1) = p \quad \text{et} \quad ℙ(X=0) = 1-p. \]
On note $X \sim \mathscr{B}(p)$.

\Para{Définition}[loi binomiale]

Une variable $X$ à valeurs dans $\Dcro{0,n}$ suit la loi de binomiales de paramètres $n$ et $p$
si pour tout $k∈\Dcro{0,n}$, \[ ℙ(X=k) = \binom{n}{k} p^k (1-p)^{n-k}. \]
On note $X \sim \mathscr{B}(n,p)$.

% -----------------------------------------------------------------------------
\section{Espérance}

\Para{Définition}[espérance d'une variable aléatoire]

Soit $X$ une valeur aléatoire à valeurs dans $E⊂ℝ$.
L'\emph{espérance de $X$} est
\[ 𝔼(X) = ∑_{ω∈Ω} X(ω) \, ℙ\bigl(\{ω\}\bigr). \]

\Para{Théorème}[formule de transfert]

Si $X$ est une variable aléatoire réelle, alors
\[ 𝔼(X) = ∑_{x∈E} x \, ℙ(X=x). \]

Plus généralement,
si $X$ est une variable aléatoire à valeurs dans $E$
et $\Fn{f}{E}{ℝ}$, alors
\[ 𝔼\bigl(f(X)\bigr) = ∑_{x∈E} f(x) \, ℙ(X=x). \]

\Para{Définition}

Une variable aléatoire est dite \emph{centrée} si son espérance est nulle.

\Para{Proposition}[propriétés de l'espérance]

Soit $X$ et $Y$ deux variables aléatoires réelles et $(a,b) ∈ℝ^2$.
\begin{itemize}
\item
  $𝔼(aX+bY) = a\,𝔼(X)+b\,𝔼(Y)$.
\item
  Si $X$ est à valeurs positives, alors $𝔼(X)≥0$.
\item
  Si $ℙ(X≤Y)=1$, alors $𝔼(X)≤𝔼(Y)$.
\end{itemize}

\Para{Proposition}[inégalité de Markov]

Soit $X$ une variable aléatoire réelle.
Pour tout $a > 0$, on a
\[ ℙ(X≥a) ≤\frac{𝔼(X)}{a} \]

% -----------------------------------------------------------------------------
\section{Indépendance}

\Para{Définition}

Soit $\Prob$ un espace probabilisé fini.
Soit $X$ une variable aléatoire à valeurs dans $E$ et $Y$ une variable aléatoire à valeurs dans $F$.
On dit que $X$ et $Y$ sont \emph{indépendantes} si et seulement si
\begin{multline*}
  ∀x∈E \+ ∀y∈F \+ \\
  ℙ\bigl( (X=x)∩(Y=y) \bigr) = ℙ(X=x) \, ℙ(Y=y).
\end{multline*}

\Para{Théorème}

Soit $\Prob$ un espace probabilisé fini.
Soit $X$ une variable aléatoire à valeurs dans $E$ et $Y$ une variable aléatoire à valeurs dans $F$.
Les variables aléatoires $X$ et $Y$ sont indépendantes si et seulement si
\begin{multline*}
  ∀A⊂E \+∀B⊂F \+ \\
  ℙ\bigl( (X∈A)∩(Y∈B) \bigr) = ℙ(X∈A) \, ℙ(Y∈B).
\end{multline*}

\Para{Théorème}

Soit $\Prob$ un espace probabilisé fini.
Soit $X$ une variable aléatoire à valeurs dans $E$ et $Y$ une variable aléatoire à valeurs dans $F$.
Si $X$ et $Y$ sont indépendantes, il en va de même pour les variables aléatoires $f(X)$ et $g(Y)$
où $\Fn{f}{E}{E'}$ et $\Fn{g}{F}{F'}$ sont deux applications quelconques.

\Para{Théorème}

Soit $X$ et $Y$ deux variables aléatoires indépendantes.
Alors \[ 𝔼(XY) = 𝔼(X) \, 𝔼(Y). \]

% -----------------------------------------------------------------------------
\section{Variance}

\Para{Définition}[variance d'une variable aléatoire]

Soit $X$ une variable aléatoire réelle.
Sa \emph{variance} est $𝔼\bigl( (X-𝔼(X))^2 \bigr)$ et on la note $𝕍(X)$.
Son \emph{écart-type} est $σ(X) =√{𝕍(X)}$.

\Para{Proposition}[inégalité de Bienaymé-Tchebychev]

Soit $X$ une variable aléatoire réelle.
Pour tout $α>0$, on a
\[ ℙ \bigPa{ |X-𝔼(X)| ≥ α} ≤ \frac{𝕍(X)}{α^2} \]

\Para{Définition}[covariance de deux variables aléatoires]

Soit $X$ et $Y$ deux variables aléatoires réelles.
Leur \emph{covariance} est $𝔼\bigPa{ (X-𝔼(X)) (Y-𝔼(Y)) }$ et on la note $\mathrm{Cov}(X,Y)$.

\Para{Propriétés}

Soit $X$ et $Y$ deux variables aléatoires réelles.
\begin{itemize}
\item
  $𝕍(aX+b) = a^2𝕍(X)$.
\item
  Si $X$ et $Y$ sont indépendantes, alors $𝕍(X+Y) =𝕍(X)+𝕍(Y)$.
\item
  $𝕍(X+Y) = 𝕍(X) + 𝕍(Y) + 2\mathrm{Cov}(X,Y)$.
\end{itemize}

\Para{Définition}[corrélation]

Soit $X$ et $Y$ deux variables aléatoires réelles de variance non nulles.
Leur \emph{coefficient de corrélation} est
\[ ρ(X,Y) = \frac{\mathrm{Cov}(X,Y)}{σ(X)σ(Y)} \]

\Para{Propriétés}

Soit $X$ et $Y$ deux variables aléatoires réelles de variance non nulles.
\begin{itemize}
\item
  $ρ(X,Y) ∈ \intF{-1,1}$
\item
  $ρ(X,Y) = ±1$ si et seulement si il existe deux réels $a$ et $b$ tels que l'événement $(Y=aX+b)$ soit certain.
\item
  Si $X$ et $Y$ sont indépendantes, alors $ρ(X,Y) = 0$. On dit qu'elles sont décorrélées.
  La réciproque est fausse.
\end{itemize}

% -----------------------------------------------------------------------------
\section{Exercices}

\Exercice[lois usuelles]
\begin{enumerate}
\item
  Soit $X$ une variable aléatoire suivant une loi uniforme sur $\Dcro{a,b}$.
  Déterminer son espérance et sa variance.
\item
  Même questions pour la loi de Bernoulli.
\item
  Même questions pour la loi binomiale.
\end{enumerate}

\Exercice

Soit $X_1, \dots, X_n$ des variables aléatoires indépendantes de loi de Bernoulli de paramètre $p$.
On pose $X = ∑_{k=1}^n X_k$.
\begin{enumerate}
\item
  Déterminer la loi de $X$
\item
  Retrouver l'espérance et la variance de la loi binomiale.
\end{enumerate}

\Exercice

D'après les statistiques, un piéton marchant sur la bande d'arrêt
d'urgence d'une autoroute a une probabilité de $8\%$ d'être renversé
par une voiture chaque minute.
Monsieur X, quelque peu atteint, a décidé de faire avec un ami
le pari stupide suivant:
s'il réussit à marcher un quart d'heure sur la bande d'arrêt d'urgence
sans être victime d'un accident, son ami lui devra $200$ Euros.
À combien monsieur $X$ estime-t-il sa vie?

\Exercice

Pour déterminer la note de fin d'année, un professeur procède ainsi:
il lance deux dés, et considère la plus petite valeur obtenue.
Il définit alors la variable aléatoire $N$, valant $3$ fois la
plus petite valeur obtenue.
Décrire la loi de $N$, puis calculer son espérance et son écart-type.

\Exercice

Monsieur Duchmol affirme que, grâce à son ordinateur, il peut prédire
le sexe des enfants à naître. Pour cette prédiction, il ne demande que
5 Euros, destinés à couvrir les frais de gestion; de plus, pour
\og{}prouver\fg{} sa bonne foi, il s'engage à rembourser intégralement
en cas de prédiction erronée.
\begin{enumerate}
\item
  Soit $X$ le gain de monsieur Duchmol; écrire la loi de probabilité de $X$.
\item
  Si monsieur Duchmol trouve 1000 naïfs, combien peut-il espérer gagner?
\end{enumerate}

\Exercice

Casimir a une technique bien particulière pour choisir quel nouveau CD il va acheter.
Il commence par choisir un CD au hasard, et l'achète s'il lui plaît,
et le repose dans le cas contraire.
Or Casimir est difficile: il n'aime que $1\%$ des CDs.
Bien sûr, tant qu'il n'a pas trouvé de CD à sa convenance, il recommence l'opération.
\begin{enumerate}
\item
  Soit $N$ le nombre de CDs que Casimir regarde avant de se décider.
  Calculer $ℙ(N=k)$.
\item
  Déterminer l'espérance et l'écart-type de $N$.
\item
  En fait, Casimir, toujours curieux, lance deux dés quand le CD ne lui plaît pas.
  S'il obtient deux as, il prend le CD quand même, se disant qu'il y a là
  un signe. Que vaut maintenant l'espérance de $N$?
\end{enumerate}

\Exercice

Monica et Chandler jouent au baby-foot.
Monica, plus douée que Chandler au baby-foot,
gagne chaque manche avec une probabilité de $80\%$.

Quelle est la probabilité pour que Monica gagne la partie, i.e.
gagne $5$ manches avant Chandler?

\Exercice

Alice et Bob jouent aux dés.
Normalement, Alice utilise un dé équilibré, mais il lui arrive de tricher,
environ une fois sur cent, en utilisant un dé pipé qui sort un as deux fois
sur trois. Une fois que la partie est commencée, Alice ne peut plus
changer de dé sans que Bob s'en aperçoive.

Sachant qu'Alice a obtenu 9 as sur les 20 derniers lancers, quelle est la
probabilité pour que le prochain lancer donne un as?

\Exercice[lois usuelles]

\begin{enumerate}
\item
  Un automobiliste rencontre successivement 5 feux de circulation indépendants sur le boulevard de Strasbourg.
  La probabilité qu'un feu soit vert est de $1/2$.
  On note $X$ le nombre de feux verts pour le cycliste.
  Déterminer la loi de $X$, son espérance et sa variance.
\item
  Un parking souterrain contient 20 scooters à trois roues, 20 motos et 20 voitures.
  On choisit un véhicule au hasard, et on note $X$ le nombre de roues de ce véhicule.
  Déterminer la loi de $X$, son espérance, et sa variance.
\item
  Une étude statistique a permis de déterminer que 10\% de la population est gauchère.
  Quelle est la probabilité qu'un groupe de 8 personnes contienne un seul gaucher?
  Au plus deux gauchers?
\item
  Le stock d'un fournisseur de lasagnes contient une proportion $p = 49/1000$
  de barquettes de lasagnes à base de viande de cheval.
  Un contrôleur examine des barquettes de lasagnes chez ce fournisseur.
  Combien doit-il contrôler de barquettes en moyenne pour qu'il trouve au moins une barquette à base de viande de cheval?
\item
  Une jarre contient 12 scorpions, 27 araignées et 56 blattes.
  On choisit une araignée au hasard parmi les 27 araignées dans la jarre.
  $X$ est le nombre de pattes de l'animal choisi.
  Déterminer la loi de $X$.
\end{enumerate}

\Exercice

Harry P., apprenti-sorcier de son état, sort en moyenne deux soirs par semaine.
Comme les lendemains matins sont plutôt difficile, il soulage alors ses maux de tête par un sortilège.
Cependant, ainsi que Hermione G. l'en avait averti, ce sortilège possède un effet secondaire parfois gênant:
une fois sur cent, aléatoirement, le sorcier se retrouve transformé pour la journée en une icône disco, dans son cas un \emph{Village People}.

Quelle est la probabilité que Harry P., sur le cours d'une année entière, se retrouve au moins 3 jours sous cette forme?

\Exercice[loi hypergéométrique]

Une urne contient $a$ boules blanches et $b$ boules noires.
On tire une poignée de $n$ boules dans l'urne, avec $(a,b) ∈(ℕ^*)^2$ et $n ∈\Dcro{1,a+b}$.
On appelle $X$ le nombre de boules blanches dans la poignée.
\begin{enumerate}
\item
  Déterminer le support de $X$.
\item
  Déterminer la loi de $X$.
\item
  Calculer l'espérance de $X$.
\item
  Calculer l'espérance de $X(X-1)$ puis la variance de $X$.
\item
  Comparer l'espérance et la variance de $X$ à celle d'une loi binomiale de paramètres
  $n$ et $a/(a+b)$. Commentaire?
\end{enumerate}

\Exercice[marche aléatoire]

Un mobile se déplace de façon aléatoire sur un axe gradué.
À l'instant $0$, il est à l'origine.
À chaque instant entier, il se déplace d'une unité vers la droite avec la probabilité $p∈\intO{0,1}$
ou d'un pas vers la gauche avec la probabilité $q=1-p$,
et de ce façon indépendante.
On note $X_n$ son abscisse après $n$ pas.
\begin{enumerate}
\item
  Soit $D_n$ la variable aléatoire égale au nombre de pas vers la droite.
  Quelle est la loi de $D_n$? Exprimer $X_n$ en fonction de $D_n$.
\item
  En déduire l'espérance et la variance de $X_n$.
  Pour quelle valeur de $p$ la variable $X_n$ est-elle centrée?
  Interpréter.
\item
  Reprendre l'exercice avec une autre méthode:
  on note, pour $n≥1$, $Y_n = X_n - X_{n-1}$.

  \begin{enumerate}
  \item
    Déterminer la loi de $Y_n$.
  \item
    Justifier l'indépendance de $Y_1, \dots, Y_n$.
  \item
    En déduire l'espérance et la variance de $X_n$.
  \end{enumerate}
\end{enumerate}

\Exercice

Un microorganisme se reproduit par divisions à intervalles réguliers.
Après une division, la division suivante se produit soit une heure après
(division courte, avec la probabilité $p$)
soit deux heures après (division lente, avec la probabilité $q=1-p$),
indépendamment de l'histoire du microorganisme.

On isole un microorganisme,
et on note $X_n$ le nombre de microorganismes présents au bout de $n$ heures.
On cherche à calculer $𝔼(X_n)$.
\begin{enumerate}
\item
  Pour cela, notons $B_n$ le nombre de microorganismes présents au temps $n$
  qui sont au milieu d'une division longue, et $A_n = X_n - B_n$.
  Exprimer $𝔼(A_{n+1})$ et $𝔼(B_{n+1})$ en fonction de $𝔼(A_n)$ et de $𝔼(B_n)$.
\item
  Déterminer $𝔼(X_n)$.
\end{enumerate}

\Exercice

Deux avions $A_1$ et $A_2$ possèdent respectivement deux et quatre moteurs.
Chaque moteur a la probabilité $p$ (où $p∈\intO{0,1}$) de tomber en panne
et les moteurs sont indépendants les uns des autres.
Les deux avions partent pour un même trajet.
Chacun des avions arrivent à destination si strictement plus de la moitié de ses moteurs reste en état de marche.
Vous partez pour cette destination.
Quel avion choisissez vous?

\Exercice

Soit $X$ une variable aléatoire suivant une loi binomiale $\mathscr{B}(n,p)$.
Calculer $𝔼(2^X)$ et $𝔼(\frac{1}{1+X})$.

\Exercice

Une urne contient deux boules blanches et $n-2$ boules noires.
On tire les boules successivement, sans remise.
On appelle $X$ le rang de sortie de la première boule blanche,
$Y$ le nombre de boules noires restantes à ce moment dans l'urne
et $Z$ le rang de sortie de la seconde boule blanche.
\begin{enumerate}
\item
  Déterminer la loi de $X$ et son espérance.
\item
  Exprimer $Y$ en fonction de $X$ et calculer $𝔼(Y)$.
\item
  Trouver un lien entre $Z$ et $X$ et en déduire la loi de~$Z$.
\end{enumerate}

\end{document}
