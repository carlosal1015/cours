\documentclass{yann}
\usepackage[most]{tcolorbox}

\undef\U
\undef\V
\newcommand\U{(u_n)_{n∈ℕ}}
\newcommand\V{(v_n)_{n∈ℕ}}
\newcommand\SU{∑_n u_n}
\newcommand\SV{∑_n v_n}

\begin{document}
\title{Séries numériques}
\maketitle

% -----------------------------------------------------------------------------
\section{Généralités}

\Para{Définitions}
\begin{itemize}
\item
Soit $\U$ une suite numérique.
  La \emph{série numérique} de \emph{terme général} $u_n$ est la suite $(S_n)_{n∈ℕ}$ où $∀n∈ℕ$, $S_n = ∑_{k=0}^n u_k$.
  On note $\SU$ au lieu de $(S_n)_{n∈ℕ}$.
\item
Soit $\SU$ une série numérique.
  Pour $n∈ℕ$, on pose $S_n = ∑_{k=0}^n u_k$.
  $S_n$ s'appelle la \emph{somme partielle} d'ordre $n$ de la série $\SU$.
\end{itemize}

\Para{Remarque}

Techniquement, une série est donc égale à la suite de ses sommes partielles; toutefois, en pratique, on fera la distinction.

\Para{Définition}

Soit $\SU$ une série numérique et $(S_n)_{n∈ℕ}$ ses sommes partielles.

On dit que la série $\SU$ est convergente (respectivement divergente, majorée, minorée, bornée, croissante ou décroissante) si et seulement si la suite $(S_n)_{n∈ℕ}$ l'est.

\Para{Définition}

Soit $\SU$ une série numérique \emph{convergente}.

On appelle \emph{somme de la série} la limite $S$ de la suite de ses sommes partielles $(S_n)_{n∈ℕ}$, et on la note $∑_{n=0}^{+∞} u_n$.
Ainsi, $S = ∑_{n=0}^{+∞} u_n = \lim_\ninf S_n = \lim_\ninf ∑_{k=0}^n u_k$

\Para{Définition}

Soit $\SU$ une série numérique \emph{convergente}.
Notons $(S_n)_{n∈ℕ}$ ses sommes partielles et $S$ sa somme.
On appelle \emph{reste} d'ordre $n$ de la série la quantité $R_n = S - S_n = ∑_{k=0}^{+∞} u_k -∑_{k=0}^n u_k = ∑_{k=n+1}^{+∞} u_k = ∑_{k>n} u_k$.

\Para{Proposition}

Soit $\SU$ une série numérique convergente.
Notons $S$ sa somme, $(S_n)_{n∈ℕ}$ ses sommes partielles et $(R_n)_{n∈ℕ}$ ses restes.
Alors on a $∀n∈ℕ, S_n + R_n = S$.
De plus, $R_n \Toninf 0$.

\Para{Exemples}
\begin{itemize}
\item
\emph{Séries géométriques:}

  Soit $(u_n)_{n∈ℕ}$ une suite géométrique non nulle de raison $q∈ℂ$.
  Alors la série $\SU$ converge si et seulement si $\Abs{q}<1$.
\item
\emph{Séries télescopiques:}

  Soit $(a_n)_{n∈ℕ}$ une suite numérique et $u_n = a_{n+1} - a_n$.
  Alors la série $\SU$ converge si et seulement si la suite $(a_n)_{n∈ℕ}$ converge.
  On a alors $∑_{n=0}^{+∞} u_n = \lim_\ninf a_n - a_0$.
\end{itemize}

\Para{Proposition}

Soit $\SU$ une série numérique.

Si la série converge, alors $u_n \Toninf 0$.

\Para{Attention}

La réciproque est \emph{FAUSSE}.
Le contre-exemple classique est la \emph{série harmonique}
$∑_{n≥1} \frac1n$ qui est \emph{divergente},
bien que $\frac1n \Toninf 0$.

\Para{Définition}

On dit que la série numérique $\SU$ est \emph{grossièrement divergente} si et seulement si $u_n$ ne tend pas vers 0.

\subsection{Absolue convergence}

\Para{Définition}

Soit $\SU$ une série numérique.
On dit que la série est \emph{absolument convergente} si et seulement si la série $∑_n \Abs{u_n}$ converge.

\Para{Théorème}

Toute série absolument convergente est convergente.

\emph{Autrement dit:}
Soit $\U$ une suite numérique.
Si $∑_n \Abs{u_n}$ converge, alors $\SU$ converge.

\Para{Remarque}

L'utilité pratique de ce théorème réside dans le fait qu'il est généralement plus simple d'étudier la série à termes positifs $∑_n \Abs{u_n}$ que la série numérique générale $\SU$.

% -----------------------------------------------------------------------------
\section{Séries à termes positifs (SATP)}

\Para{Définition}

Une \emph{série à termes positifs} est une série numérique $\SU$ telle que $∀n∈ℕ, u_n∈\Rp$.

\Para{Proposition}

Une SATP converge si et seulement si la suite des ses sommes partielles est majorée.
De plus, si elle converge, sa somme est égale à la borne supérieure
des sommes partielles.

\Para{Lemme}

Soit $\U$ et $\V$ deux suites réelles.

On suppose que $∀n∈ℕ, 0≤u_n≤v_n$.
\begin{itemize}
\item
Si $\SV$ converge, alors $\SU$ converge également.
\item
Si $\SU$ diverge, alors $\SV$ diverge également.
\end{itemize}

\Para{Proposition}

Soit $\SU$ et $\SV$ deux SATP.
On suppose que $u_n = \GrandO_\ninf(v_n)$.
Alors:
\begin{itemize}
\item
Si $\SV$ converge, alors $\SU$ converge également.
\item
Si $\SU$ diverge, alors $\SV$ diverge également.
\end{itemize}

\Para{Théorème}

Soit $\SU$ et $\SV$ deux SATP telles que \emph{$u_n \Sim_\ninf v_n$}.
Alors $\SU$ et $\SV$ ont \emph{même nature}.

\Para{Remarque}

On a en fait un résultat plus précis (mais hors-programme), cf. exercice~\ref{exo:equiv_reste}.

\subsection{Comparaison série-intégrale}

\Para{Lemme}

Soit $I$ est un intervalle de $ℝ$
et $\Fn{f}{I}{ℝ}$ continue par morceaux \emph{décroissante}.
Alors si $[n-1,n+1]⊂I$, on a:
\[ f(n+1) ≤ ∫_n^{n+1} f(x) \D x ≤ f(n), \]
\[ ∫_n^{n+1} f(x) \D x ≤ f(n) ≤ ∫_{n-1}^n f(x) \D x. \]

\Para{Lemme}

Soit $\Fn{f}{\Rp}{ℝ}$ continue par morceaux, \emph{positive} et \emph{décroissante}.
Alors la série de terme général $u_n =∫_{n-1}^n f(x) \D x - f(n)$
est convergente.

\Para{Théorème}

Soit $n_0∈ℕ$ et $\Fn{f}{\intFO{n_0, +∞}}{ℝ}$ continue par morceaux, \emph{positive} et \emph{décroissante}.
Alors la série $∑_{n} f(n)$ converge si et seulement si la suite $\left( ∫_{n_0}^n f(x) \D x \right)_n$ converge.

\Para{Corollaire}[pour les $5/2$]

La série $∑_{n} f(n)$ converge si et seulement si l'intégrale $∫_{n_0}^{+∞} f(x) \D x$ converge.

\Para{Proposition}[séries de Riemann]

Soit $α∈ℝ$.
La série $∑_n \frac{1}{n^α}$ converge si et seulement si $α> 1$.

% -----------------------------------------------------------------------------
\section{Séries numériques générales}

\Para{Définition}

Une série \emph{semi-convergente} est une série convergente mais non absolument convergente.
Ainsi, une série numérique est:
\begin{itemize}
\item
soit absolument convergente,
\item
soit semi-convergente,
\item
soit divergente.
\end{itemize}

\Para{Définition}

Soit $\SU$ une série réelle.
On dit que cette série est \emph{alternée} si et seulement si $(-1)^n u_n$ est de signe constant.

\Para{Théorème}[critère spécial]

On considère la série réelle $\SU$ où:
\begin{itemize}
\item
$u_n = (-1)^n a_n$
\item
$(a_n)_{n∈ℕ}$ décroissante
\item
$a_n \Toninf 0$
\end{itemize}

Alors la série $\SU$ converge.

De plus, on a la majoration suivante, valable pour tout $n∈ℕ∪\Acco{-1}$:
\[ \Abs{R_n} = \left| ∑_{k=n+1}^{+∞} u_k \right| ≤\Abs{u_{n+1}} \]
et $R_n$ est du signe de $u_{n+1}$.

% -----------------------------------------------------------------------------
\section{Règles pratiques}

\Para{Proposition}

Soit $\SU$ et $\SV$ deux séries numériques. On suppose que:
\begin{itemize}
\item
$\SV$ converge \emph{absolument},
\item
$u_n = \GrandO_\ninf(v_n)$, ou a fortiori $u_n = \PetitO_\ninf(v_n)$.
\end{itemize}

Alors $\SU$ converge absolument.

\Para{Proposition}

Soit $\SU$ et $\SV$ deux séries réelles.
On suppose que:
\begin{itemize}
\item
$u_n \Sim_\ninf v_n$,
\item
à partir d'un certain rang $v_n$ est de \emph{signe constant}.
\end{itemize}

Alors $\SU$ et $\SV$ ont même nature.

\Para{Proposition}[règle $n^αu_n$]

Soit $\SU$ une série numérique.
On suppose qu'il existe \emph{$α>1$} tel que $n^αu_n \Toninf 0$.

Alors la série $\SU$ est absolument convergente.

\Para{Proposition}[règle de d'Alembert]

Soit $\SU$ une série numérique.
On suppose que:
\begin{itemize}
\item
$∃N∈ℕ,∀n≥N, u_n≠0$
\item
$\left|\frac{u_{n+1}}{u_n}\right| \Toninf λ∈\Rp ∪\Acco{+∞}$
\end{itemize}

Alors:
\begin{itemize}
\item
si $λ>1$, la série est grossièrement divergente.
\item
si $λ<1$, la série est absolument convergente.
\item
si $λ=1$, on ne peut pas conclure.
\end{itemize}

\Para{Remarque}

La règle de Raabe-Duhamel (hors-programme) permet de préciser le cas $\lambda=1$, cf. exercice~\ref{exo:raabe-duhamel}.

% -----------------------------------------------------------------------------
\section{Opérations}

\Para{Proposition}

L'ensemble des séries numériques est un espace vectoriel pour les lois usuelles.
L'ensemble des séries numériques convergentes est un sous-espace vectoriel de cet espace vectoriel.

\Para{Définition}

Soit $∑_n a_n$ et $∑_n b_n$ deux séries numériques.

On appelle \emph{produit de Cauchy} de ces deux séries la série
$∑_n c_n$ où $∀n∈ℕ$,
\[ c_n =∑_{k=0}^n a_k b_{n-k} =∑_{l=0}^n a_{n-l} b_l \]

\Para{Théorème}

Soit $∑_n a_n$ et $∑_n b_n$ deux séries numériques \emph{absolument convergentes}.

Alors leur produit de Cauchy $∑_n c_n$ est absolument convergent, et:
\[ ∑_{n=0}^{+∞} c_n = \left(∑_{n=0}^{+∞} a_n \right) \left(∑_{n=0}^{+∞} b_n \right) \]

% -----------------------------------------------------------------------------
\section{Exercices}

\subsection{Séries à termes positifs}

\Exercice

Étudier la nature des séries $∑_n u_n$ dans chacun des cas suivants:
\begin{enumerate}
\item
$u_n =√{\frac{n+1}{n}}$
\item
$u_n = \frac{\ln n}{2^n}$
\item
$u_n = n^{1/n}-(n+1)^{1/n}$
\item
$u_n = \frac{n√n}{2^n+√n}$
\item
$u_n = \ln(\frac{1}{√n}) - \ln\bigl(\sin(\frac{1}{√n})\bigr)$
\item
$u_n = \frac{1}{n√[n]{n}}$
\item
$u_n = \arccos\left(\frac{n^3+1}{n^3+2}\right)$
\item
$u_n = \frac{\ln(1 + a^n n^α)}{n^β}$ où $a > 0$
\item
$u_n = n^αe^{β√n}$
\item
$u_n^{-1} = 1^α+ 2^α+ \cdots + n^α$ où $α> 0$
\item
$u_n^{-1} = 1 + √2 + √[3] 3 + \cdots + √[n] n$
\item
$u_n = \cos\left(\arctan(n) + \frac{1}{n^α}\right)$ où $α> 0$
\item
$u_n = \bigl(\cos \frac1n\bigr)^{n^α}$ où $α> 0$
\item
$u_n = \frac{(√n)^{\ln n}}{(\ln n)^{√n}}$
\item
$u_n = \ln\left(3\tan^2\bigl(\frac{πn^α}{6n^α+1}\bigr)\right)$ où $α> 0$
\item
$u_n = \sin\bigl(π(2+√3)^n\bigr)$
\end{enumerate}

\Exercice

Soit $u_n = 1/n$ si $n$ est un carré parfait (0, 1, 4, 9, 16, etc.) et $u_n = 0$ sinon.
Déterminer la nature de la série de terme général $u_n$.

\Exercice

Soit $\SU$ et $\SV$ deux SATP convergentes.
Peut-on affirmer que la série de terme général $\max(u_n, v_n)$ est également convergente?

\Exercice

Soit $\SU$ une SATP convergente.
Peut-on affirmer que la série de terme général $v_n = \prod_{k=0}^n u_k$ est également convergente?

\Exercice\label{exo:equiv_reste}

Soit $\U$ et $\V$ deux séries numériques à termes positifs.
On suppose que $u_n \sim v_n$ quand $n \to +\infty$.
\begin{enumerate}
\item
On suppose que $\SU$ converge;
  on sait qu'alors $\SV$ converge également.
  Montrer que les restes des deux séries sont équivalents, £cad.:
  \[ ∑_{k=n+1}^{+∞} u_k \,\,\Sim_\ninf\,\,∑_{k=n+1}^{+∞} v_k \]
\item
On suppose que $\SU$ diverge;
  on sait qu'alors $\SV$ diverge également.
  Montrer que les sommes partielles des deux séries sont équivalentes, £cad.:
  \[ ∑_{k=0}^n u_k \,\,\Sim_\ninf\,\,∑_{k=0}^n v_k \]
\end{enumerate}

\Exercice

Démontrer que chacune des séries $∑_n u_n$ converge,
et calculer leur somme:
\begin{enumerate}
\item
$u_n =√{n}+a√{n+1}+b√{n+2}$
\item
$u_n = \arctan\left(\frac{1}{n^2+n+1}\right)$
\item
$u_n = \ln\left(1-\frac{1}{n^2}\right)$
\item
$u_n = \ln\left(\cos\frac{θ}{2^n}\right)$ où $θ∈\intFO{0,\fracπ2}$
\item
$u_n = \frac{(-1)^n}{3^n} \cos^3(3^nθ)$
\item
$u_n = \frac{√{n+1}-2√{n}+1}{2^{n+1}}$
\item
$u_n = \frac{4n}{n^4+2n^2+9}$
\item
$u_n = \frac{4n-3}{n(n^2-4)}$ pour $n≥3$
\item
$u_n = \frac{1}{(2n-1)(2n)(2n+1)(2n+2)}$
\item
$u_n = \frac{2n^3-3n^2+1}{(n+2)!}$
\item
$u_n = \frac{1}{(n+1)^2(n+2)^2}$
\item
$u_n =∫_{\frac{1}{n+1}}^{\frac{1}{n}} \frac{e^{√x}}{√x} \D x$ pour $n≥1$
\end{enumerate}

\Exercice

\begin{enumerate}
\item
Soit $\alpha < 1$.

  Déterminer un équivalent de $S_n = \sum_{k=1}^n \frac{1}{k^\alpha}$.
\item
Soit $\alpha > 1$.

  Déterminer un équivalent de $R_n = \sum_{k=n+1}^{+\infty} \frac{1}{k^\alpha}$.
\end{enumerate}

\Exercice[développement asymptotique de la série harmonique]

La \emph{série harmonique} est la série $∑_n \frac1n$.
\begin{enumerate}
\item
Montrer de plusieurs façons qu'il s'agit d'une série divergente:
  \begin{itemize}
  \item
en comparant la série à une intégrale
  \item
en minorant $∑_{k=2^n}^{2^{n+1} - 1} \frac1k$
  \item
en remarquant que $\frac1n \sim \ln n - \ln (n-1)$, et que cette expression est télescopique
  \item
une autre méthode?
  \end{itemize}
\item
On note $H_n = ∑_{k=1}^n \frac1k$ les sommes partielles de la série harmonique.
  En reprenant la comparaison série intégrale de la question 1a,
  montrer que $H_n \sim \ln n$.
  \emph{Remarque:} on pourrait aussi utiliser la question 1c.
\item
Montrer qu'il existe une unique suite $(u_n)_{n∈ℕ^*}$
  telle que $∀n≥1$, $∑_{k=1}^n u_k = H_n - \ln n$ et la déterminer.
\item
Déterminer un équivalent de $u_n$, puis montrer que la série $∑u_n$ converge.
\item
En déduire que la suite $H_n - \ln n$ converge également.
  On note $γ$ sa limite; il s'agit de la \emph{constante d'Euler}, dont une valeur approchée est $γ\simeq 0.577$.
  Ainsi \[ \tcboxmath{ H_n = \ln n + γ + \PetitO\limits_\ninf(1) } \]
\item
Pour $α> 1$, montrer que $∑_{k=n+1}^{+∞} \frac{1}{k^α} \sim \frac{1}{(α-1) n^{α-1}}$.
  On pourra utiliser une comparaison série-intégrale.
\item
Déterminer un équivalent du reste $R_n = ∑_{k = n+1}^{+∞} u_k$ de la série $∑u_n$.
\item
Exprimer $R_n$ en fonction de $H_n$, $\ln n$ et $γ$.
  En déduire que $H_n = \ln n + γ+ \frac{1}{2n} + \PetitO(\frac{1}{n})$.
\item
Montrer qu'il existe une unique suite $(v_n)_{n∈ℕ^*}$ telle que $∀n≥1$, $∑_{k=1}^n v_k = H_n - \ln n - \frac{1}{2n}$ et la déterminer.
\item
Déterminer un équivalent de $v_n$ quand $n \to +∞$.
\item
En déduire un équivalent du reste $T_n = ∑_{k=n+1}^{+∞} v_k$.
\item
Montrer que:
    \[ H_n = \ln n +γ+ \frac{1}{2n} - \frac{1}{12n^2} + \PetitO_\ninf\BigPa{\frac{1}{n^2}}. \]
\end{enumerate}

\Exercice

Soit $u_n = \frac{n-a\floor{n/a}}{n(n+1)}$ où $a\in\Ns$.
Déterminer la nature et la somme de la série de terme général $u_n$.

\Exercice[séries de Bertrand]

Les \emph{séries de Bertrand} sont les séries de la forme
$∑_n \frac{1}{n^α\ln^βn}$ où $(α,β)∈ℝ^2$.

Le but de cet exercice est de montrer le résultat suivant:
\begin{itemize}
\item
si $α> 1$, la série converge
\item
si $α< 1$, la série diverge
\item
si $α= 1$:
  \begin{itemize}
  \item
si $β> 1$, la série converge
  \item
si $β≤1$, la série diverge
  \end{itemize}
\end{itemize}

Notons $u_n = \frac{1}{n^α\ln^βn}$ pour $n≥2$.
\begin{enumerate}
\item
Vérifier que dans le cas particulier $β= 0$, on retrouve une règle connue.
\item
Si $α> 1$, montrer qu'il existe $γ> 1$ tel que $u_n = \PetitO(\frac{1}{n^γ})$. Conclure.
\item
Si $α< 1$, montrer qu'il existe $γ< 1$ tel que $\frac{1}{n^γ} = \PetitO(u_n)$. Conclure.
\item
On suppose désormais $α= 1$. Calculer $∫\frac{\D x}{x \ln^βx}$, en faisant bien attention de distinguer le cas $β= 1$.
\item
Si $β≤0$, montrer que $u_n≥\frac1n$. Conclure.
\item
Si $β> 0$, montrer que $∫_n^{n+1} \frac{\D x}{x \ln^βx}≤u_n≤∫_{n-1}^n \frac{\D x}{x \ln^βx}$ pour tout $n≥3$.
\item
Conclure en distinguant les cas $0<β<1$, $β=1$ et $β>1$.
\end{enumerate}

% \Exercice[règle de Cauchy]
%
% La \emph{règle de Cauchy} est la suivante:
%
% Soit $∑u_n$ une série à termes positifs. On suppose que $√[n]{u_n} \Toninf ℓ∈ℝ∪\Acco{+∞}$. Alors:
% \begin{itemize}
% \item
si $ℓ< 1$, la série converge;
% \item
si $ℓ> 1$, la série diverge;
% \item
si $ℓ= 1$, on ne peut pas conclure.
% \end{itemize}
%
% Questions:
% \begin{enumerate}
% \item
Si $ℓ< 1$, montrer qu'il existe $a∈\intFO{0,1}$ tel que $u_n = \PetitO(a^n)$. Conclure.
% \item
Si $ℓ> 1$, montrer qu'il existe $a > 1$ tel que $a^n = \PetitO(u_n)$. Conclure.
% \item
Trouver un exemple de suite $(u_n)$ convergente telle que $ℓ= 1$.
%   Trouver également un exemple de suite $(u_n)$ divergente telle que $ℓ= 1$.
%   Conclure.
% \item
Démontrer la règle de Cauchy.
% \item
\emph{Application:} Déterminer la nature des séries $∑u_n$ dans les cas suivants.
%   \begin{enumerate}
%   \item
$u_n = \left(\frac{n-1}{2n+1}\right)^n$
%   \item
$u_n = \left({a+\frac1n}\right)^n$
%   \end{enumerate}
% \item
Soit $(u_n)$ une suite de réels strictements positifs telle que $\frac{u_{n+1}}{u_n} \Toninfℓ∈ℝ∪\Acco{+∞}$.
%   Montrer que $√[n]{u_n} \Toninfℓ$.
%   En déduire que la règle de Cauchy est \emph{plus forte} que la règle de d'Alembert.
% \end{enumerate}

\Exercice[comparaison logarithmique]

Soit $∑_n u_n$ et $∑_n v_n$ deux séries réelles.
On suppose qu'il existe $N∈ℕ$ tel que pour tout $n≥N$:
\begin{itemize}
\item
$u_n > 0$
\item
$v_n > 0$
\item
$\frac{u_{n+1}}{u_n}≤\frac{v_{n+1}}{v_n}$
\end{itemize}

Montrer que:
\begin{itemize}
\item
Si $∑_n v_n$ converge, alors $∑_n u_n$ converge également.
\item
Si $∑_n u_n$ diverge, alors $∑_n v_n$ diverge également.
\end{itemize}

\Exercice[règle de Raabe-Duhamel]\label{exo:raabe-duhamel}

La \emph{règle de Raabe-Duhamel} est la suivante:

Soit $∑u_n$ une série à termes strictement positifs.
On suppose que $∃α∈ℝ$ tel que
\[ \frac{u_{n+1}}{u_n} = 1 - \frac{α}{n} + O\Pafrac{1}{n^2}. \]
Alors la série $∑u_n$ converge si et seulement si $α > 1$.
Plus précisément, il existe une constante $K > 0$ telle que $u_n \sim \frac{K}{n^α}$.

Cette règle précise le cas douteux de la règle de d'Alembert.
\begin{enumerate}
\item
Soit $(u_n)$ une suite de réels strictements positifs telle que $\frac{u_{n+1}}{u_n} = 1 - \frac{α}{n} + \GrandO(\frac{1}{n^2})$.
  On pose $v_n = n^αu_n$ et $w_n = \ln\left(\frac{v_{n+1}}{v_n}\right)$.
  Montrer que $w_n = \GrandO(\frac{1}{n^2})$.
\item
En déduire que la suite de terme général $\ln v_n$ converge.
\item
En déduire l'existence d'une constante $K > 0$ telle que $u_n \sim \frac{K}{n^α}$.
\item
Démontrer la règle de Raabe-Duhamel.
\item
Appliquer le résultat précédent à:
  \begin{enumerate}
  \item
$u_n = \frac{n⋅n!}{(a+1)(a+2)\cdots(a+n)}$ où $a>0$
  \item
$u_n = n! e^n n^{-n}$
  \end{enumerate}
\item
Montrer que le règle s'applique encore si l'on suppose seulement que
  $\frac{u_{n+1}}{u_n} = 1 - \frac{α}{n} + ε_n$, où
  $∑\Abs{ε_n}$ est une série convergente,
  ce qui est bien le cas si $ε_n = \GrandO(\frac{1}{n^2})$.
\end{enumerate}

\Exercice

Soit $(a_n)_{n∈ℕ}$ une suite réelle positive.
On définit la suite $(u_n)$ par $u_0 > 0$ et $∀n∈ℕ$, $u_{n+1} = u_n + \frac{a_n}{u_n}$.
Montrer que la suite $(u_n)$ converge si et seulement si la série $∑a_n$ converge.

\Exercice[critère de condensation de Cauchy]
\begin{enumerate}
\item
Soit $(a_n)_{n∈ℕ^*}$ une suite réelle positive décroissante.
  On note $b_n = 2^n a_{2^n}$ pour $n∈ℕ$.
  \begin{enumerate}
  \item
Montrer que $∀n∈ℕ$, $∑_{k = 1}^{2^{n+1}-1} a_k≤∑_{k=0}^n b_k$
  \item
Montrer que $∀n∈ℕ$, $∑_{k = 0}^n b_k ≤2\left( ∑_{k=1}^{2^n} a_k \right) - a_{2^n}$
  \item
En déduire que les séries $∑_n a_n$ et $∑_n b_n$ sont de même nature.
  \end{enumerate}
\item
Montrer le \emph{critère de condensation de Cauchy}:

  Soit $(a_n)_{n∈ℕ}$ une suite réelle positive, décroissante à partir d'un certain rang.
  Alors les séries $∑_n a_n$ et $∑_n 2^n a_{2^n}$ sont de même nature.
\item
\emph{Application:} retrouver les résultats sur les séries de Riemann, puis sur les séries de Bertrand.
\end{enumerate}

\Exercice[archi-classique, formule de Stirling]
\begin{enumerate}
\item
Soit $a_n = \ln(n!)$.
  \begin{enumerate}
  \item
Montrer que $a_n =∑_{k=2}^n \ln k$.
  \item
Montrer que $∀k≥2$, $∫_{k-1}^k \ln x \D x≤\ln k≤∫_k^{k+1} \ln x \D x$.
  \item
En déduire un encadrement de $a_n$, puis que $a_n \sim n\ln n$.
  \end{enumerate}
\item
Soit $b_n = a_n - n\ln n = \ln(n!) - n\ln n$.
  \begin{enumerate}
  \item
Montrer que $b_{n+1} - b_n \Toninf -1$.
  \item
En utilisant le théorème de Cesàro, en déduire que $b_n \sim -n$.
  \end{enumerate}
\item
Soit $c_n = b_n - (-n) = \ln(n!) - n\ln n + n$.
  \begin{enumerate}
  \item
Montrer que $c_{n+1} - c_n \sim \frac1{2n}$.
  \item
En déduire que $c_n \sim \frac12 \ln n$.
  \end{enumerate}
\item
Soit $d_n = c_n - \frac12 \ln n = \ln(n!) - n\ln n + n - \frac12 \ln n$.
  \begin{enumerate}
  \item
Montrer que $d_{n+1} - d_n = \GrandO_\ninf (\frac{1}{n^2})$.
  \item
En déduire que la série de terme général $∑_n (d_{n+1}-d_n)$ converge.
  \item
En déduire que la suite $(d_n)$ converge. On note $ℓ$ sa limite.
    Montrer que:
    \[ \ln(n!) = n\ln n - n + \frac12 \ln n +ℓ+ \PetitO_\ninf(1) \]
  \end{enumerate}
\item
En déduire qu'il existe une constante $K > 0$, que l'on exprimera en fonction de $ℓ$, telle que:
  \[ n! \sim K √n \Pafrac{n}{e}^n \]
\item
On pose $W_n =∫_0^{\fracπ2} \sin^n t \D t$.
  Il s'agit bien évidemment des \emph{intégrales de Wallis}.
  \begin{enumerate}
  \item
Montrer que pour tout $n∈ℕ$, on a: $W_{n+2} = \frac{n+1}{n+2} W_n$.
    \emph{Indication}: on pourra partir de $W_{n+2}$ et faire un intégration par parties
    bien choisie.
  \item
En déduire que pour tout $n∈ℕ$, on a:
    $W_{2n} = \frac{(2n)!π}{(n!)^2 2^{2n+1}}$.
  \item
En déduire que $W_{2n} \sim \frac{π}{K√{2n}}$.
  \item
Établir, pour tout entier naturel $n$, l'encadrement $W_{n+2}≤W_{n+1}≤W_n$.
    En déduire que $W_{n+1} \sim W_n$.
  \item
Soit $I_n = (n+1) W_{n+1} W_n$.
    Montrer que la suite $(I_n)$ est constante; calculer $I_0$.
    En déduire que $W_n \sim√{\frac{π}{2n}}$.
  \item
Déduire de c) et e) la valeur de $K$.
  \end{enumerate}
\item
En déduire que
  \[ \tcboxmath{ n! \Sim_\ninf \sqrt{2\pi n} \; \BigPa{\frac{n}{e}}^n } \]
  Ce résultat est connu sous le nom de \emph{formule de Stirling}.
\end{enumerate}

\Exercice

Soit $u_0 \in \intO{0,\pi}$ et pour tout $n\in\N$, $u_{n+1} = \sin(u_n)$.
On cherche à déterminer la nature de la série de terme général $\frac{u_n}{\sqrt n}$.
\begin{enumerate}
    \item
Montrer que $u_n \to 0$.
    \item
Déterminer un réel $\alpha$ tel que $u_{n+1} - u_n$ ait une limite finie non nulle.
    \item
En déduire un équivalent de $u_n$; on utilisera le théorème de Cesàro (ou celui des petits pas).
    \item
Conclure.
\end{enumerate}

\subsection{Séries numériques}

\Exercice

Étudier la nature des séries $∑_n u_n$ dans chacun des cas suivants:
\begin{enumerate}
\item
$u_n = \frac{(-1)^n}{\tan(\frac1n)}$
\item
$u_n = \cos\left(π√{n^2+n+1}\right)$
\item
$u_n = \frac{(-1)^n}{n+\sin n}$
\item
$u_n = \ln\left(1+\frac{(-1)^n}{n^α}\right)$ où $α> 0$
\item
$u_n = \frac{(-1)^n}{n^α+ (-1)^n n^β}$ où $α> 0$, $β> 0$ et $α≠β$
\item
$u_n = \frac{\sin(\ln n)}{n}$
\item
$u_n = \cos\left( πn^2 \ln(\frac{n}{n-1}) \right)$
\item
$u_n = \frac{n(-1)^n}{(2n+1)(3n+1)}$, et calculer la somme
\item
$u_n = \frac{(-1)^{\floor{√n}}}{n^α}$
\end{enumerate}

\Exercice

On considère la série harmonique alternée $∑_n u_n$ où $u_n = \frac{(-1)^n}{n+1}$
\begin{enumerate}
\item
Montrer qu'il s'agit d'une série convergente.
\item
Montrer que les sommes partielles de cette série peuvent s'exprimer
  en fonction des sommes partielles de la série harmonique.
  Plus précisément, montrer que:
  \[ ∑_{n = 0}^{2N-1} u_n = H_{2N} - H_N \]
\item
En déduire la valeur de la somme $∑_{n=0}^{+∞} u_n$.

  On s'intéresse désormais à la série $∑_n v_n$,
  où la suite $(v_n)$ est une permutation de la suite $(u_n)$
  obtenue en prenant dans l'ordre un terme positif puis deux termes négatifs, et ainsi de suite:
  $v_0 = 1$, $v_1 = - \frac12$, $v_2 = -\frac14$, $v_3 = \frac13$, $v_4 = -\frac16$, $v_5 = -\frac18$,
  $v_6 = \frac15$, $v_7 = -\frac{1}{10}$, $v_8 = -\frac{1}{12}$, etc...
\item
Vérifiez que vous avez compris la définition de la suite $(v_n)$: calculer $v_{3n}, v_{3n+1}, v_{3n+2}$.
\item
Notons $S_n =∑_{k=0}^n v_k$.
  Exprimer $S_{3n-1}$ sous forme d'une série et en déduire que la suite $(S_{3n-1})$ converge.
\item
En déduire que la série $∑_n v_n$ converge.
\item
Exprimer $S_{3n-1}$ en fonction de $H_p$.
\item
En déduire la valeur de la somme $∑_{n=0}^{+∞} v_n$.
\item
Commenter.
\end{enumerate}

\Exercice[dur]

On pose $u_n =∑_{k=n}^{+∞} \frac{(-1)^k}{√{k+1}}$.
Étudier la convergence de la série $∑ u_n$.

\Exercice[transformation d'Abel]

Soit $(a_n)$ et $(b_n)$ deux suites numériques.
On pose $B_n =∑_{k=0}^n b_k$, et $S_n =∑_{k=0}^n a_k b_k$.
\begin{enumerate}
\item
Montrer que $S_n = a_{n+1} B_n +∑_{k=0}^n (a_k - a_{k+1}) B_k$.
  Cette égalité porte le nom de \emph{transformation d'Abel}.
\item
En déduire que si
  \begin{itemize}
  \item
la suite $(B_n)$ est bornée, et
  \item
la suite $(a_n)$ est une suite réelle positive décroissante de limite nulle,
  \end{itemize}
  alors la série $\sum a_n b_n$ converge.
\item
Montrer que le critère spécial est un cas particulier de le la règle précédente.
\item
\emph{Applications:}
  \begin{enumerate}
  \item
Si $a_n$ est une suite réelle positive décroissante de limite nulle, et $t∈ℝ∖2πℤ$, montrer que la série $∑_n a_n e^{int}$ converge.
  \item
Étudier pour $α∈\intOF{0,1}$, $β\not∈2πℤ$, $γ∈ℝ$ la série $∑_n \frac{\sin(nβ+γ)}{n^α}$.
  \item
Étudier pour $α>0$ la série $∑_n \frac{\sin n}{n^α + \sin n}$.
  \end{enumerate}
\end{enumerate}

\subsection{Amusant}

\Exercice

Expliquer comment il est possible d'empiler~$n$ pièces identiques de sorte que celle du haut soit en \emph{porte-à-faux} avec celle du bas, c.-à-d. que la projection la pièce du haut sur le plan horizontal ne touche pas la pièce du bas.

\end{document}
