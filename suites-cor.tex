\documentclass{yann}

\newcommand\Exo[1]{\paragraph{Exercice #1}}

\begin{document}
\title{Suites numériques: corrigés}
\maketitle

\Exo{1}
\begin{enumerate}
\setcounter{enumi}{17}
\item
% 18
  C'est une forme indéterminée du type $1^\infty$.
  On applique la formule de Taylor-Young à la fonction tangente en $\pi/4$:
  $\tan(\pi/4+x) = \tan(\pi/4) + x \tan'(\pi/4) + o(x)$ quand $x \to 0$.
  Or $\tan(\pi/4) = 1$ et $\tan'(\pi/4) = 1+\tan^2(\pi/4) = 2$, ainsi
  $\tan(\pi/4+\alpha/n) = 1+2\alpha/n + o(1/n)$ quand $n\to\infty$.

  Ainsi, $\ln u_n = n \ln(1+2\alpha/n + o(1/n)) = n(2\alpha/n + o(1/n)) \to 2\alpha$,
  donc $u_n \to e^{2\alpha}$ quand $n\to\infty$.

  Remarquons également que $1+2\alpha/n+o(1/n)$ est positif à partir d'un certain rang, ce qui justifie l'utilisation du logarithme.

\item
% 19
  On a $\ln(n+1) = \ln(n) + \ln(1+1/n) = \ln(n) + 1/n + o(1/n)$,
  d'où
  \begin{align*}
    \ln(u_n) &= n\ln(n) \ln \biggPa{\frac{\ln(n+1)}{\ln(n)}} \\
             &= n\ln(n) \ln \biggPa{ 1 + \frac{1}{n\ln(n)} + o\BigPa{\frac{1}{n\ln(n)}} } \\
             &= n\ln(n) \biggPa{ \frac{1}{n\ln(n)} + o\BigPa{\frac{1}{n\ln(n)}} } \\
             &= 1 + o(1)
  \end{align*}
  Ainsi, $\ln u_n \to 1$, donc $u_n \to e$ quand $n \to +\infty$

\item
% 20
  On a besoin d'un équivalent de $\ln\bigPa{\frac{\arctan(n+1)}{\arctan(n)}}$.

  On a une formule de trigonométrie bien pratique:
  \[ \forall x > 0, \arctan(x) + \arctan(1/x) = \frac\pi2, \]
  d'où $\arctan(n) = \pi/2 - \arctan(1/n) = \pi/2 - 1/n + o(1/n^2)$
  puis $\ln(\arctan(n)) = \ln(\pi/2) + \ln(1-2/(\pi n) + o(1/n^2)
  = \ln(\pi/2) - 2/(\pi n) + o(1/n^2)$.
  Ainsi,
  \begin{align*}
  \ln\Pa{\frac{\arctan(n+1)}{\arctan(n)}}
     &= \ln(\arctan(n+1)) - \ln(\arctan(n)) \\
     &= \frac{2}{\pi} \pa{ \frac{1}{n} - \frac{1}{n+1} } + o\BigPa{\frac{1}{n^2}} \\
     &= \frac{2}{\pi} \cdot \frac{1}{n(n+1)} + o\BigPa{\frac{1}{n^2}} \\
     &\sim \frac{2}{\pi n^2}
  \end{align*}

  Ceci nous fournit immédiatement un équivalent de $\ln(u_n)$, d'où $\ln(u_n) \to 2/\pi$,
  et $u_n \to e^{2/\pi}$ quand $n\to\infty$.

\item
% 21
  Le cosinus étant pair, on a
  \begin{align*}
  u_n &= \cos\bigPa{\pi n^2 \cdot \bigCro{ -\ln(1-1/n) }} \\
      &= \cos\bigPa{\pi n^2 \cdot \bigCro{ 1/n + 1/(2n^2) + o(1/n^2)}} \\
      &= \cos\bigPa{\pi n + \pi/2 + o(1)} \\
      &= (-1)^n \cos(\pi/2 + o(1)) \\
      &\to 0
  \end{align*}
  car $\cos(x+n\pi) = (-1)^n \cos(x)$ et $\cos(\pi/2) = 0$.
\end{enumerate}

\end{document}
