\documentclass{yann}

\undef\U
\undef\V
\newcommand{\En}{(E,\Norme)}
\newcommand{\U}{(u_n)_{n∈ℕ}}
\newcommand{\V}{(v_n)_{n∈ℕ}}

\begin{document}
\title{Espaces vectoriels normés en dimension finie}
\maketitle

% -----------------------------------------------------------------------------
\section{Normes}

\subsection{Généralités}

\Para{Définition}

Soit $E$ un $𝕂$-espace vectoriel.
Une \emph{norme} sur $E$ est une application $\Fn{N}{E}{ℝ}$ vérifiant les axiomes suivants.
Pour tous $(x,y)∈E^2$ et pour tout $λ∈𝕂$:\begin{enumerate}
\item \emph{positivité:} $N(x)≥0$
\item \emph{séparation:} $N(x) = 0 \implies x = 0_E$
\item \emph{homogénéité:} $N(λx) = \Abs{λ} N(x)$
\item \emph{inégalité triangulaire:} $N(x+y) ≤ N(x) + N(y)$
\end{enumerate}

\Para{Remarque}

Le nom d'inégalité triangulaire vient du fait que dans un triangle $ABC$ on a nécessairement $AC ≤AB + BC$.

\Para{Remarque}

On note fréquemment $\Norm{x}$ au lieu de $N(x)$.

\Para{Exemples}

Sur $E = 𝕂^n$.

Pour $x = \nUplet x1n∈E$:\begin{enumerate}
\item $\Norm{x}_1 = ∑_{k=1}^n \Abs{x_k}$
\item $\Norm{x}_2 = √{∑_{k=1}^n \Abs{x_k}^2}$
\item $\Norm{x}_∞ = \max\Big(\Uplet{\Abs{x_1}}{\Abs{x_n}}\Big)$
\end{enumerate}

\Para{Proposition}[inégalité triangulaire inversée]

Soit $E$ un espace vectoriel, $N$ une norme sur $E$ et $(x,y)∈E^2$.
On a $\Abs{N(x) - N(y)}≤N(x+y)$.

\Para{Définition}

Un \emph{espace vectoriel normé} est un couple $(E,N)$ où $E$ est un espace vectoriel et $N$ un norme sur $E$.
On note parfois $E$ au lieu de $(E,N)$ si le choix de la norme est clair d'après le contexte.

\subsection{Distance}

\Para{Définition}

Soit $\En$ un espace vectoriel normé.
La \emph{distance associée} à $\Norme$ est l'application $\Fn{d}{E×E}{ℝ}$ définie par:
$∀(x,y)∈E^2$, $d(x,y) = \Norm{x-y}$.

\Para{Remarque}[HP]

De façon générale, on appelle \emph{distance} sur un ensemble $E$
une application $\Fn{d}{E^2}{ℝ}$ vérifiant les axiomes suivants:
pour tous $(x,y,z)∈E^3$:
\begin{enumerate}
\item \emph{positivité:} $d(x,y) ≥ 0$
\item \emph{symétrie:} $d(x,y) = d(y,x)$
\item \emph{séparation:} $d(x,y) = 0 \implies x = y$
\item \emph{inégalité triangulaire:} $d(x,y) ≤ d(x,z) + d(z,y)$
\end{enumerate}

Si $d$ est une distance sur $E$, on dit que $(E,d)$ est un \emph{espace métrique}.

\subsection{Boules, sphères}

\Para{Définitions}

Soit $\En$ un espace vectoriel normé.\begin{itemize}
\item La \emph{boule ouverte} de centre $a∈E$ et de rayon $r∈ℝ$ est $B(a,r) = \Ensemble{x∈E}{d(x,a)<r}$.
\item La \emph{boule fermée} de centre $a∈E$ et de rayon $r∈ℝ$ est $B_f(a,r) = \Ensemble{x∈E}{d(x,a)≤r}$.
\item La \emph{sphère} de centre $a∈E$ et de rayon $r∈ℝ$ est $S(a,r) = \Ensemble{x∈E}{d(x,a)=r} = B_f(a,r) ∖B(a,r)$.
\end{itemize}

\Para{Exemple}

On prend $E = ℝ^2$.

Tracer les sphères unité pour les normes $\Norme_1$, $\Norme_2$ et $\Norme_∞$.

\subsection{Parties convexes}

\Para{Définition}

Soit $\En$ un espace vectoriel et $A$ une partie de $E$.
On dit que $A$ est une \emph{partie convexe}, ou plus simplement que $A$ est \emph{convexe}, si et seulement si $∀(x,y)∈A^2$, $∀λ∈[0,1]$, $λx + (1-λ)y∈A$.

\Para{Proposition}

Soit $\En$ un espace vectoriel normé, $a∈E$ et $r∈ℝ$.
Les boules $B(a,r)$ et $B_f(a,r)$ sont convexes.

\subsection{Parties bornées, applications bornées}

\Para{Définition}

Soit $\En$ un espace vectoriel normé et $A$ une partie de $E$.
On dit que $A$ est une partie \emph{bornée} de $E$ si et seulement si $∃M≥0$, $∀x∈A$, $\Norm{x}≤M$.

\Para{Définition}

Soit $(E,\Norme_E)$ et $(F,\Norme_F)$ deux espaces vectoriels normés et $A⊂E$.
Soit $\Fn{f}{A}{F}$.
On dit que $f$ est une application \emph{bornée} si et seulement si la partie $f(A)$ est bornée, c.-à-d. si et seulement si $∃M≥0$, $∀x∈A$, $\Norm{f(x)}_F≤M$.

% -----------------------------------------------------------------------------
\section{Suites}

\Para{Définitions}

Soit $\En$ un espace vectoriel normé.\begin{itemize}
\item Une \emph{suite à valeurs dans $E$} est une application $\Fn{u}{ℕ}{E}$.
  On note fréquemment $u_n$ au lieu de $u(n)$ et $\U$ au lieu de $u$.
\item Soit $\U$ une suite à valeurs dans $E$ et $ℓ∈E$.
  On dit que la suite $\U$ \emph{converge} vers $ℓ$ si et seulement si
  \[∀ε> 0\+∃n_0∈ℕ\+∀n≥n_0\+ \Norm{u_n -ℓ}≤ε\]
  $ℓ$ est nécessairement unique et s'appelle \emph{la limite} de la suite $\U$.
  On note $\lim\limits_{\ninf}^{\Norme} u_n = ℓ$, ou encore $u_n \xrightarrow[\ninf]{\Norme}ℓ$.
\end{itemize}

\Para{Remarque}

Soit $\En$ un espace vectoriel normé, $\U$ une suite à valeurs dans $E$ et $ℓ∈E$.
Alors $u_n \xrightarrow{\Norme}ℓ$ si et seulement si $\Norm{u_n - ℓ} \to 0$.

\Para{Définitions}
\begin{enumerate}
\item Une suite $\U$ à valeurs dans un espace vectoriel normé $\En$ est dite \emph{convergente} si et seulement si il existe $ℓ∈E$
  tel que $\U$ converge vers $ℓ$.
\item Elle est dite \emph{divergente} dans le cas contraire.
\end{enumerate}

\Para{Définition (HP)}

Soit $E$ un £ev., $N$ et $N'$ deux normes sur $E$.
On dit que $N$ et $N'$ sont \emph{équivalentes} £ssil. existe deux constantes strictement positives $α$ et $β$
telles que \[ ∀x∈E \+ αN(x) ≤ N'(x) ≤ βN(x). \]

\Para{Propriétés (HP)}

\begin{itemize}
\item Il s'agit d'une relation d'équivalence, £cad.
  \begin{itemize}
  \item toute norme est équivalente à elle-même;
  \item si $N$ est équivalente à $N'$, alors $N'$ est équivalente à $N$;
  \item deux normes équivalentes à une même troisième sont équivalentes entre elles.
  \end{itemize}
\item Si $N$ et $N'$ sont deux normes équivalentes,
  alors pour toute suite $(u_n)$ à valeurs dans $E$ et pour tout vecteur $ℓ∈E$, on a
  \[ u_n \xrightarrow{N} ℓ  \iff u_n \xrightarrow{N'} ℓ. \]
\end{itemize}

\Para{Théorème (HP)}

Si $E$ est de dimension finie, toutes les normes sur $E$ sont équivalentes.

\Para{Corollaire fondamental}

Si $E$ est de dimension finie, le choix de la norme n'influe pas sur la convergence d'une suite, ni sur sa limite.
On pourra donc écrire $u_n \to ℓ$, sans préciser la norme choisie.

\Para{Proposition}

En dimension finie, la convergence d'une suite de vecteurs est équivalente à la convergence des suites coordonnées.
Plus précisément, soit:
\begin{enumerate}
\item $\En$ un $𝕂$-espace vectoriel normé de dimension finie
\item $\B = \nUplet e1p$ une base de $E$
\item $(u_n)_{n∈ℕ}$ une suite à valeurs dans $E$
\item $\Uplet{(u^1_n)_{n∈ℕ}}{(u^p_n)_{n∈ℕ}}$ des suites numériques telles que:
  $∀n∈ℕ$, $u_n = u^1_n e_1 + u^2_n e_2 + \cdots + u^p_n e_p$
\item $ℓ = ℓ^1 e_1 + ℓ^2 e_2 + \cdots + ℓ^p e_p$ où $ℓ∈E$ et $(\Uplet{ℓ^1}{ℓ^p})∈𝕂^p$.
\end{enumerate}
Alors \[ u_n \to ℓ \iff ∀k∈\Dcro{1,p} \+ u^k_n \toℓ^k. \]

\Para{Proposition}[opérations algébriques sur les limites]

Soit $\En$ un $𝕂$-espace vectoriel normé.
Soit $\U$ et $\V$ deux suites à valeurs dans $E$
et $(λ_n)_{n∈ℕ}$ une suite à valeurs dans $𝕂$.\begin{itemize}
\item Si $u_n \to ℓ$ et $v_n \to ℓ'$, alors $u_n + v_n \to ℓ+ℓ'$.
\item Si $λ_n \to α$ et $u_n \to ℓ$, alors $λ_n u_n \to αℓ$.
\end{itemize}

\Para{Définitions}[relations de comparaison entre suites]

Soit $\En$ un $𝕂$-espace vectoriel normé.
Soit $\U$ une suite à valeurs dans $E$ et $(α_n)_{n∈ℕ}$ une suite numérique à valeurs dans $𝕂$.\begin{enumerate}
\item On dit que $u_n = O(α_n)$ si et seulement si $\Norm{u_n} = O(α_n)$ si et seulement si $∃N∈ℕ$, $∃K∈\Rp$, $∀n≥N$, $\Norm{u_n} ≤ K\Abs{α_n}$.
  On dit alors que $\U$ est \emph{dominée} par $(α_n)_{n∈ℕ}$.
\item On dit que $u_n = o(α_n)$ si et seulement si $\Norm{u_n} = o(α_n)$ si et seulement si $∀ε>0$, $∃N∈ℕ$, $∀n≥N$, $\Norm{u_n} ≤ ε\Abs{α_n}$.
  On dit alors que $\U$ est \emph{négligeable} devant $(α_n)_{n∈ℕ}$.
\end{enumerate}

% -----------------------------------------------------------------------------
\section{Topologie}

\subsection{Ouverts et fermés}

\Para{Définitions}

Soit $\En$ un espace vectoriel normé.\begin{itemize}
\item Une \emph{partie ouverte} de $E$, ou plus simplement un \emph{ouvert} de $E$, est une partie $U$ de $E$ telle que $∀x∈U$, $∃r>0$, $B(x,r)⊂U$.
\item Une \emph{partie fermée} de $E$, ou plus simplement un \emph{fermé} de $E$, est une partie $F$ de $E$ telle que $E∖F$ est un ouvert de $E$.
\end{itemize}

\Para{Proposition}

Soit $\En$ un espace vectoriel normé. Soit $a∈E$ et $r∈ℝ$. Alors:\begin{itemize}
\item la boule ouverte $B(a,r)$ est une partie ouverte de $E$;
\item la boule fermée $B_f(a,r)$ est une partie fermée de $E$.
\end{itemize}

\Para{Proposition}[propriétés des ouverts]

Soit $\En$ un espace vectoriel normé.\begin{itemize}
\item $∅$ et $E$ sont des ouverts.
\item \emph{stabilité par union quelconque}:
  Si $∀i∈I$, $U_i$ est un ouvert de $E$, alors $⋃_{i∈I} U_i$ est un ouvert de $E$.
\item \emph{stabilité par intersection finie}:
  Si $U_1, \dots, U_n$ sont des ouverts de $E$, alors $⋂_{i=1}^n U_i$ est un ouvert de $E$.
\end{itemize}

\Para{Proposition}[propriétés des fermés]

Soit $\En$ un espace vectoriel normé.\begin{itemize}
\item $∅$ et $E$ sont des fermés.
\item \emph{stabilité par intersection quelconque}:
  Si $∀i∈I$, $F_i$ est un fermé de $E$, alors $⋂_{i∈I} F_i$ est un fermé de $E$.
\item \emph{stabilité par union finie}:
  Si $F_1, \cdots, F_n$ sont des fermés de $E$, alors $⋃_{i=1}^n F_i$ est un fermé de $E$.
\end{itemize}

\Para{Proposition}

Soit $E$ un espace vectoriel \emph{de dimension finie}.
Alors la topologie de $E$ ne dépend pas du choix de la norme sur $E$.

Autrement dit, si $A⊂E$, le caractère ouvert (ou fermé) de $A$ ne dépend pas du choix de la norme sur $E$.

\subsection{Point intérieur, point adhérent}

\Para{Proposition-Définition}

Soit $\En$ un espace vectoriel normé, $A$ une partie de $E$ et $x∈E$.
Les conditions suivantes sont équivalentes:\begin{enumerate}
\item il existe un ouvert $U$ de $E$ tel que $x∈U$ et $U⊂A$;
\item $∃r > 0$, $B(x,r)⊂A$;
\item pour toute suite $\U$ à valeurs dans $E$ convergent vers $x$, on a $u_n∈A$ à partir d'un certain rang.
\end{enumerate}

On dit que $x$ est un \emph{point intérieur} à $A$ lorsque ces conditions sont vérifiées.

\Para{Proposition-Définition}

Soit $\En$ un espace vectoriel normé, $A$ une partie de $E$ et $x∈E$.
Les conditions suivantes sont équivalentes:\begin{enumerate}
\item pour tout ouvert $U$ de $E$ tel que $x∈U$, on a $U∩A≠∅$;
\item $∀r>0$, $B(x,r)∩A ≠∅$;
\item $∀ε>0$, $∃a∈A$, $\Norm{x-a}≤ε$;
\item il existe une suite $\U$ à valeurs \emph{dans $A$} telle que $u_n \to x$.
\end{enumerate}

On dit que $x$ est un \emph{point adhérent} à $A$ lorsque ces conditions sont vérifiées.

\subsection{Intérieur, adhérence, frontière}

\Para{Définitions}

Soit $\En$ un espace vectoriel normé, $A$ une partie de $E$.\begin{itemize}
\item L'ensemble des points intérieurs de $A$ s'appelle l'\emph{intérieur} de $A$,
  et on le note $\Int{A}$
\item L'ensemble des points adhérents à $A$ s'appelle l'\emph{adhérence} de $A$,
  et on le note $\Adh{A}$.
\item L'adhérence de $A$ privée de l'intérieur de $A$ s'appelle la \emph{frontière} de $A$ et on le note $\mathrm{Fr}(A) = ∂A = \Adh{A}∖\Int{A}$.
\end{itemize}

\Para{Proposition}
\begin{itemize}
\item si $A ⊂ B$, alors $\Int A ⊂ \Int B$ et $\Adh A ⊂ \Adh B$;
\item on a toujours $\Int{A} ⊂ A ⊂ \Adh{A}$;
\item $\Int{A}$ est un ouvert, $\Adh{A}$ est un fermé;
\item $A = \Int{A}$ si et seulement si $A$ est un ouvert;
\item $A = \Adh{A}$ si et seulement si $A$ est un fermé.
\end{itemize}

\subsection{Densité}

\Para{Proposition-Définition}

Soit $\En$ un espace vectoriel normé et $A$ une partie de $E$.
Les conditions suivantes sont équivalentes:\begin{enumerate}
\item pour tout ouvert $U$ non vide de $E$, on a $U∩A≠∅$;
\item $∀x∈E$, $∀r > 0$, $B(x,r)∩A≠∅$;
\item tout point de $E$ est adhérent à $A$;
\item $\Adh{A} = E$;
\item pour tout $x∈E$, il existe une suite $\U$ à valeurs \emph{dans $A$} telle que $u_n \to x$.
\end{enumerate}

On dit que $A$ est une partie \emph{dense} de $E$ lorsque ces conditions sont vérifiées.

% -----------------------------------------------------------------------------
\section{Applications dans un espace vectoriel normé}

\subsection{Limite d'une application}

\Para{Définition}

Soit $(E,\Norme_E)$ et $(F,\Norme_F)$ deux espaces vectoriels normés, $A$ une partie de $E$, $\Fn{f}{A}{F}$, $a∈E$ un point adhérent à $A$ et $ℓ∈F$.
On dit que $f$ admet $ℓ$ pour \emph{limite} en $a$ si et seulement si
$∀ε> 0$, $∃δ>0$, $∀x∈A$, $\Norm{x-a}_E≤δ\implies \Norm{f(x)-ℓ}_F≤ε$.
On écrit alors $\lim_{x \to a} f(x) = \lim_a f = ℓ$.

\Para{Remarque}

On peut comme d'habitude étendre la définition de la limite dans les cas suivants:\begin{itemize}
\item si $E = ℝ$ et $a = ±∞$;
\item si $F = ℝ$ et $ℓ= ±∞$.
\end{itemize}

\Para{Théorème}[critère séquentiel des limites]

Soit $E$ et $F$ deux espaces vectoriels normés et $A⊂E$.
Soit $\Fn{f}{A}{F}$, $a$ un point adhérent à $A$ et $b∈F$.
Les conditions suivantes sont équivalentes:\begin{enumerate}
\item $\lim_a f = b$
\item Toute suite à valeur dans $A$ telle que $u_n \to a$ vérifie $f(u_n) \to b$
\end{enumerate}

\Para{Proposition}[limite et coordonnées]

Soit $(E,\Norme_E)$ et $(F,\Norme_F)$ deux espaces vectoriels normés, $A$ une partie de $E$, $\Fn{f}{A}{F}$, $a∈E$ un point adhérent à $A$ et $ℓ∈F$.
On suppose que:\begin{itemize}
\item \emph{$F$ est de dimension finie} et $\B = \nUplet e1p$ une base de $F$.
\item $∀x∈A$, $f(x) = ∑_{k=1}^p f_k(x) e_k$ où $\Fn{f_k}{A}{𝕂}$.
\item $ℓ= ∑_{k=1}^p ℓ_k e_k$ où $\nUpletℓ1p ∈𝕂^p$.
\end{itemize}

Alors $\lim_{x \to a} f(x) = ℓ$ si et seulement si $∀k∈\Dcro{1,p}\+ \lim_{x \to a} f_k(x) =ℓ_k$.

\Para{Proposition}[limite d'une composée]

Soit $E$, $F$, $G$ trois espaces vectoriels normés, $A⊂E$ et $B⊂F$.
Soit $\Fn{f}{A}{F}$ et $\Fn{g}{B}{G}$.
On suppose:\begin{itemize}
\item $f(A)⊂B$ de sorte que la composée $\Fn{g◦f}{A}{G}$ existe;
\item $a$ est un point adhérent à $A$ et $\lim_a f = b$;
\item $b$ est un point adhérent à $B$ et $\lim_b g = ℓ$.
\end{itemize}

Alors $\lim_a g◦f = ℓ$.

\Para{Proposition}[opérations algébriques sur les limites]

Soit $E$ et $F$ deux espaces vectoriels normés et $A⊂E$.
Soit $\Fn{f}{A}{F}$, $g \colon A \to F$ et $k \colon A \to𝕂$.
Soit $a$ un point adhérent à $A$.\begin{itemize}
\item Si $\lim_a f =ℓ$ et $\lim_a g =ℓ'$, alors $\lim_a (f+g) = ℓ+ℓ'$.
\item Si $\lim_a k =α$ et $\lim_a f =ℓ$, alors $\lim_a (kf) = αℓ$.
\end{itemize}

\subsection{Continuité}

\Para{Définition}

Soit $E$ et $F$ deux espaces vectoriels normés et $A⊂E$.
Soit $\Fn{f}{A}{F}$.
On dit que $f$ est \emph{continue en $a∈A$} si et seulement si $\lim_a f = f(a)$, c.-à-d. si et seulement si
\begin{multline*}
  ∀ε>0 \+ ∃δ>0 \+ ∀x∈A \+ \\
  \Norm{x-a}_E≤δ\implies \Norm{f(x)-f(a)}_F≤ε.
\end{multline*}
On dit que $f$ est \emph{continue} si et seulement si elle est continue en tout point de $A$.

\Para{Proposition}[continuité et coordonnées]

Soit $E$ et $F$ deux espaces vectoriels normés et $A⊂E$.
Soit $\Fn{f}{A}{F}$ et $a∈A$.
On suppose que:\begin{itemize}
\item $F$ est de dimension finie et $\mathcal{B} = \nUplet e1p$ une base de $F$;
\item $∀x∈A$, $f(x) =∑_{k=1}^p f_k(x) e_k$ où $f_k \colon A \to𝕂$.
\end{itemize}

Alors $f$ est continue en $a$ si et seulement si les fonctions $\Uplet{f_1}{f_p}$ sont toutes continues en $a$.

\Para{Proposition}[propriétés des fonctions continues]
\begin{itemize}
\item La composée de deux fonctions continues est continue.
\item La restriction d'une fonction continue est continue.
\end{itemize}

\Para{Théorème}

Soit $E$ et $F$ deux espaces vectoriels normés et $\Fn{f}{E}{F}$ une fonction continue.
Soit $A⊂F$ une partie de $F$\begin{itemize}
\item Si $A$ est un ouvert de $F$, alors $f^{-1}(A)$ est un ouvert de $E$.
\item Si $A$ est un fermé de $F$, alors $f^{-1}(A)$ est un fermé de $E$.
\end{itemize}

\Para{Corollaire}

Soit $E$ un espaces vectoriels normés et $\Fn{f}{E}{ℝ}$ une fonction continue.\begin{itemize}
\item $\Ensemble{x∈E}{f(x)>0}$ est un ouvert de $E$.
\item $\Ensemble{x∈E}{f(x)=0}$ est un fermé de $E$.
\item $\Ensemble{x∈E}{f(x)≥0}$ est un fermé de $E$.
\end{itemize}

\subsection{Applications lipschitziennes}

\Para{Définition}

Soit $(E, \Norme_E)$ et $(F, \Norme_F)$ deux $𝕂$-espaces vectoriels normés et $A⊂E$.
Soit $\Fn{f}{A}{F}$.\begin{itemize}
\item Soit $k∈\Rp$. L'application $f$ est dite \emph{$k$-lipschitzienne} si et seulement si $∀(x,y)∈A^2$, $\Norm{f(x)-f(y)}_F≤k \Norm{x-y}_E$.
\item L'application $f$ est dite \emph{lipschitzienne}, ou vérifie la \emph{propriété de Lipschitz}, si et seulement si il existe $k∈\Rp$ telle que $f$ soit $k$-lipschitzienne.
\end{itemize}

\Para{Proposition}

La composée de deux applications lipschitzienne l'est également.

\Para{Proposition}

Toute application lipschitzienne est continue.
La réciproque est fausse.

\subsection{Applications linéaires}

\Para{Proposition}[continuité des applications linéaires]

Soit $E$ un espace vectoriel normé de dimension finie, $F$ un espace vectoriel normé et $\Fn{f}{E}{F}$ \emph{linéaire}.
Alors $f$ est lipschitzienne (et donc continue).

\Para{Proposition}[continuité des applications multilinéaire]

Toute application multilinéaire dont l'ensemble de départ est un espace vectoriel normé de dimension finie est continue.

\subsection{Compacité}

\Para{Définition}

Soit $E$ un espace vectoriel normé de dimension finie.
On dit qu'une partie $A$ de $E$ est \emph{compacte} si et seulement si elle est fermée et bornée

\Para{Théorème}

Soit $E$ et $F$ deux espaces vectoriels normés de dimensions finies, $A⊂E$ et $\Fn{f}{A}{F}$ une fonction \emph{continue}.
Si $K⊂A$ est une partie compacte de $E$,
alors $f(K)$ est une partie compacte de $F$.

\Para{Corollaire important}

Soit $E$ un espace vectoriel normé de dimension finie, $A$ une partie fermée et bornée de $E$ et $\Fn{f}{A}{ℝ}$ une fonction continue.
Alors $f$ est bornée et ses bornes sont atteintes.

% -----------------------------------------------------------------------------
\section{Exercices}

\subsection{Normes}

\Exercice

Soit $E=𝕂^n$ et $\Norme_1$, $\Norme_2$, $\Norme_∞$ les normes usuelles\begin{enumerate}
\item Montrer que pour tout $x∈𝕂^n$, on a
  $\Norm{x}_1 ≤√n \Norm{x}_2 ≤n \Norm{x}_∞≤n \Norm{x}_1$.
\item Montrer que les constantes sont optimales.
\item En déduire que $u_n \toℓ$ a la même signification pour les trois normes.
\end{enumerate}

\Exercice[normes matricielles]

Soit $E = \MnK$.
Pour $A∈E$, $A = \Big( a_{i,j} \Big)_{1≤i,j≤n}$, on pose:
\begin{itemize}
\item $\DS\Norm{A}_1 = ∑_{i=1}^n∑_{j=1}^n \Abs{a_{i,j}}$;
\item $\DS\Norm{A}_2 = √{\Tr(\T{\bar A}A)} = √{∑_{i=1}^n∑_{j=1}^n \Abs{a_{i,j}}^2}$;
\item $\DS\Norm{A}_∞= \max_{1≤i,j≤n} \Abs{a_{i,j}}$;
\item $\DS N(A) = \max_{1≤i≤n} ∑_{j=1}^n \Abs{a_{i,j}}$.
\end{itemize}
\begin{enumerate}
\item Montrer qu'il s'agit de normes sur $E$.
\item Montrer que la norme $N$ est une \emph{norme d'algèbre}, c.-à-d. que $∀(A,B)∈E^2$, $N(AB) ≤N(A) N(B)$.
\end{enumerate}

\Exercice[normes de polynômes]

Soit $E = 𝕂[X]$.
Pour $P∈E$, $\DS P =∑_{k=0}^d a_k X^k$, on pose:
\begin{itemize}
\item $\DS N_1(P) = ∑_{k=0}^d \Abs{a_k}$;
\item $\DS N_2(P) = √{∑_{k=0}^d \Abs{a_k}^2}$;
\item $\DS N_∞(P) = \max_{0≤k≤d} \Abs{a_k}$;
\item $\DS N(P) = \sup_{x∈[0,1]} \Abs{P(x)}$.
\end{itemize}
\begin{enumerate}
\item Montrer qu'il s'agit de normes sur $E$.
\item On pose $P_n = n^{-3/4} ∑_{k=0}^{n-1} X^k$.
  Déterminer $\lim_\ninf N_1(P_n)$ et $\lim_\ninf N_2(P_n)$.
  Que remarquez-vous?
\end{enumerate}

\Exercice[normes de fonctions]

Soit $I$ un intervalle de $ℝ$.
\begin{enumerate}
\item
  On note $\mathscr{L}^1$ l'ensemble des fonctions $f$ continues et intégrables de $I$ dans $𝕂$.
  Pour $f ∈ \mathscr{L}^1$, on note
  \[ \Norm{f}_1 = ∫_I \abs{f(x)} \D x. \]
  Montrer qu'il s'agit d'une norme sur $\mathscr{L}^1$.

\item
  On note $\mathscr{L}^2$ l'ensemble des fonctions $f$ continues de $I$ dans $𝕂$ telles que $f^2$ est intégrable sur $I$.
  Pour $f ∈ \mathscr{L}^2$, on note
  \[ \Norm{f}_2 = √{∫_I \abs{f(x)}^2 \D x}. \]
  Montrer que $\mathscr{L}^2$ est un £ev. et que $\Norme_2$ est une norme sur $\mathscr{L}^2$.

\item
  On note $\mathscr{L}^∞$ l'ensemble des fonctions $f$ continues et bornées de $I$ dans $𝕂$.
  Pour $f ∈ \mathscr{L}^∞$, on note
  \[ \Norm{f}_∞ = \sup_{x∈I} \abs{f(x)}. \]
  Montrer qu'il s'agit d'une norme sur $\mathscr{L}^∞$.

\end{enumerate}


\Exercice

Sur $𝕂^n$, on pose, pour $p>1$, $\DS \Norm{ \nUplet x1n }_p = \left( ∑_{k=1}^n \Abs{x_k}^p \right)^{\frac1p}$.
\begin{enumerate}
\item Montrer que $\Norme_p$ est une norme;
  (on admettra l'inégalité triangulaire, un peu délicate (voir l'exercice suivant).
\item Déterminer pour $x∈𝕂^n$, $\lim\limits_{p\to+∞} \Norm{x}_p$.
\end{enumerate}

\Exercice[suite: preuve de l'inégalité triangulaire]

Soit $p > 1$. On note $q = \frac{p}{p-1}$.\begin{enumerate}
\item Montrer que $q > 1$ et que $\frac{1}{p} + \frac{1}{q} = 1$.
\item Montrer que $∀(x,y)∈ℝ_+^2$, $xy≤\frac{x^p}{p} + \frac{y^q}{q}$.
\item Soit $a = \nUplet a1n∈ℝ_+^n$ et $b = \nUplet b1n∈ℝ_+^n$.
  \begin{enumerate}
  \item Montrer que pour tout $λ∈\Rps$,
    \[∑_{k=1}^n a_k b_k ≤\frac{λ^p}{p} ∑_{k=1}^n a_k^p + \frac{1}{qλ^q} ∑_{k=1}^n b_k^q.\]
  \item En déduire l'\emph{inégalité de Hölder} :
    \[∑_{k=1}^n a_k b_k ≤\left( ∑_{k=1}^n a_k^p \right)^{\frac1p} \left( ∑_{k=1}^n b_k^q \right)^{\frac1q}.\]
  \end{enumerate}
\item Soit $u = \nUplet u1n∈ℝ_+^n$ et $v = \nUplet v1n∈ℝ_+^n$. On note $S =∑_{k=1}^n (u_k+v_k)^p$.
  \begin{enumerate}
  \item Montrer que, pour $k∈\Dcro{1,n}$,
    \[(u_k+v_k)^p = u_k (u_k+v_k)^{\frac{p}{q}} + v_k (u_k+v_k)^{\frac{p}{q}}.\]
  \item En déduire que
    \[S ≤\left( ∑_{k=1}^n u_k^p \right)^{\frac1p} S^{\frac1q} + \left( ∑_{k=1}^n v_k^p \right)^{\frac1p} S^{\frac1q}\]
    puis que
    \[\left( ∑_{k=1}^n (u_k+v_k)^p \right)^{\frac1p} ≤\left( ∑_{k=1}^n u_k^p \right)^{\frac1p} + \left( ∑_{k=1}^n v_k^p \right)^{\frac1p}.\]
  \end{enumerate}
\item Soit $x = \nUplet x1n∈𝕂^n$ et $y = \nUplet y1n∈𝕂^n$.
  On note $u_k = \Abs{x_k}$ et $v_k = \Abs{y_k}$ pour $k∈\Dcro{1,n}$.
  Montrer que $\Norm{x+y}_p ≤\Norm{x}_p + \Norm{y}_p$.
\end{enumerate}

\Exercice

Soit $E = ℝ^2$.
On pose, pour $(x,y)∈E$,
\[ N(x,y) = \sup_{t∈ℝ} \Abs{ \frac{x+ty}{1+t+t^2} }. \]
\begin{enumerate}
\item Montrer que $N$ est une norme.
\item Calculer $N(x,1)$ et $N(1,0)$. En déduire que, pour tous $(x,y)∈E^2$,
  \[ N(x,y) = \frac23√{x^2-xy+y^2} + \frac13 \Abs{2x-y}. \]
\item Dessiner la sphère unité pour la norme $N$.
\end{enumerate}

\Exercice

Soit $E$ et $F$ deux espaces vectoriels normés et $A⊂E$.
On note $\B(A,F)$ l'ensemble des applications bornées de $A$ dans $F$.
Pour $f∈\B(A,F)$, on pose $\Norm{f}_∞= \sup_{x∈A} \Norm{f(x)}_F$.\begin{enumerate}
\item Montrer qu'il s'agit d'un espace vectoriel.
\item Montrer que $\Norm{f}_∞$ existe.
\item Montrer que $\Norme_∞$ est une norme sur $\B(A,F)$.
\end{enumerate}

\Exercice

Soit $E = \MnK$ et $\Norme$ une norme sur $E$.
Montrer qu'il existe $k > 0$ telle que
$∀(A,B)∈E^2$, $\Norm{AB} ≤ k \Norm A \Norm B$.
On pourra utiliser l'exercice précédent.

\Exercice[convergence normale d'une série]

Soit $(E, \Norme)$ un espace vectoriel normé de dimension finie et $\U$ une suite à valeurs dans $E$. On suppose que la série $∑_n \Norm{u_n}$ converge (on dit que la série $∑_n u_n$ \emph{converge normalement}).

Montrer que la série $∑_n u_n$ converge.

\Exercice[théorème du point fixe]

Soit $(E, \Norme)$ un espace vectoriel normé de dimension finie et $A$ un fermé non vide de $E$.
Soit $\Fn{f}{A}{A}$ une application $k$-lipschitzienne où $k < 1$; une telle application est dite \emph{contractante}.

On se propose de montrer que $f$ admet un unique point fixe, £cad. que l'équation $f(x)=x$ admet une unique solution.
Ce résultat porte le nom de \emph{théorème du point fixe}.

Pour cela, on définit une suite récurrente par $u_0∈A$ et $u_{n+1} = f(u_n)$.
On pose également $v_n = u_{n+1} - u_{n}$.
\begin{enumerate}
\item Montrer que $\Norm{v_{n+1}} ≤ k \Norm{v_n}$, puis que $\Norm{v_n}≤k^n \Norm{v_0}$.
\item En déduire que la série $∑_n v_n$ converge;
  on pourra utiliser le résultat de l'exercice précédent.
\item Montrer que la suite $(u_n)$ converge, que sa limite est dans $A$,
  et qu'il s'agit d'un point fixe de $f$.
\item Montrer par ailleurs que $f$ admet au plus un point fixe.
\end{enumerate}

\Exercice

Soit $E$ un espace vectoriel de dimension finie.
On considère l'espace vectoriel $\LE$ muni d'une norme quelconque.
Vérifier que l'application $\Fonction{◦}{\LE^2}{\LE}{(u,v)}{u◦v}$
est continue.

\Exercice

Soit $I$ un intervalle de $ℝ$ et $\Fn fI𝕂$ une fonction de classe $\CC1$.
Montrer que $f$ est lipschitzienne £ssi. $f'$ est bornée.

\subsection{Topologie}

\Exercice[exemples de parties de $ℝ$]

Soit $E = ℝ$.
Parmi les parties suivantes, lesquelles sont ouvertes? lesquelles sont fermées?

$\intO{0,1}$,
$\intF{0,1}$,
$\intFO{0,1}$,
$ℤ$,
$\Rp$,
$ℚ$,
$ℝ∖ℤ$,
$ℝ∖ℚ$,
$∅$,
$ℝ$
$\Ensemble{\frac1n}{n∈\Ns}$,
$\Ensemble{\frac1n}{n∈\Ns}∪\Acco{0}$

\Exercice

Soit $E$ un £evndf. et $F$ un £sev. de $E$.
Montrer que $F$ est un fermé.

Le résultat est-il encore vrai si $E$ n'est pas de dimension finie?

\Exercice

\begin{enumerate}
\item Montrer que le groupe linéaire $\GLnK$ est un ouvert de $\MnK$.
\item Montrer que le groupe spécial linéaire $\M{SL}{n}{𝕂} = \Ensemble{M∈\MnK}{\det M=1}$ est un fermé de $\MnK$.
\end{enumerate}

\Exercice

Soit $A$ et $B$ deux parties d'un espace vectoriel normé $E$.
On note $A+B = \Ensemble{a+b}{(a,b) ∈A×B}$.\begin{enumerate}
\item Montrer que si $A$ est un ouvert, $A+B$ l'est également.
\item Montrer à l'aide d'un contre-exemple que $A$ et $B$ fermés n'entraîne pas nécéssairement $A+B$ fermé.
  On pourra prendre $E=ℝ$, $A=\Ns$ et $B=\Ensemble{e^{-n}-n}{n∈\Ns}$.
\end{enumerate}

\Exercice[utilisation de la densité]

Soit $\Fn{f}{ℝ}{ℝ}$ telle que $∀(x,y)∈ℝ^2$, $f(x+y) = f(x) + f(y)$.\begin{enumerate}
\item Montrer que $∀x∈ℝ$, $∀n∈ℕ$, $f(n x) = nf(x)$.
\item Montrer que $∀x∈ℝ$, $∀n∈ℤ$, $f(n x) = nf(x)$.
\item Montrer que $∀x∈ℝ$, $∀r∈ℚ$, $f(r x) = rf(x)$.
\item On suppose de plus qu'il existe un intervalle $[a,b]$ avec $a < b$ telle que $f$ soit bornée sur $[a,b]$.

  Une fonction qui ne vérifierait pas cela serait très bizarre (mais cela existe!)
  \begin{enumerate}
  \item Montrer que $f$ est bornée au voisinage de $0$, c.-à-d. qu'il existe $α>0$ tel que $f$ soit bornée sur l'intervalle $[-α,α]$.
  \item Montrer que $f$ est continue en $0$.
  \item Montrer que $f$ est continue sur $ℝ$.
  \item Montrer que $∃α∈ℝ$, $∀x∈ℝ$, $f(x) = αx$.
  \end{enumerate}
\end{enumerate}

\Exercice[frontière]

Soit $E$ un espace vectoriel normé et $A⊂E$.

On appelle \emph{frontière} de $A$
l'ensemble $∂A = \Adh A∖\Int A$.\begin{enumerate}
\item Montrer que $A$ est fermé si et seulement si $∂A⊂A$.
\item Montrer que $A$ est ouvert si et seulement si $∂A∩A =∅$.
\end{enumerate}

\Exercice[distance à une partie]

Soit $E$ un espace vectoriel normé.

Étant donné une partie $A$ de $E$ et un point $x$ de $E$,
on pose $d(x,A) = \inf_{a∈A} d(x,a) = \inf_{a∈A} \Norm{x-a}$\begin{enumerate}
\item Montrer que cet $\inf$ existe.
\item Montrer que l'application $d_A \colon x \mapsto d(x,A)$ est $1$-lipschitzienne et en déduire qu'elle est continue.
\item Montrer que $d(x,A) = 0$ si et seulement si $x∈\Adh{A}$.
\item Montrer que $d(x,A) = d(x,\Adh{A})$.
\item Montrer que si $A$ est convexe, l'application $d_A \colon x \mapsto d(x,A)$ est également convexe.
\end{enumerate}

\Exercice[critère séquentiel des ouverts]

Soit $E$ un espace vectoriel normé et $A⊂E$.

Montrer que $A$ est un ouvert si et seulement si pour tout $a∈A$, pour toute suite $\U$ à valeurs dans $E$ telle que $u_n \to a$, il existe un rang $N∈ℕ$ tel que pour tout $n≥N$ on ait $u_n∈A$.

\Exercice[critère séquentiel des fermés]

Soit $E$ un espace vectoriel normé et $A⊂E$.

Montrer que $A$ est un fermé si et seulement si pour toute suite convergente $\U$ à valeurs dans $A$, on a $\lim_\ninf u_n∈A$.

\Exercice

Soit $E$ et $F$ deux espaces vectoriels normés et $\Fn{f}{E}{F}$.

Montrer que $f$ est continue si et seulement si l'image réciproque par $f$ de tout ouvert de $F$ est un ouvert de $E$.

\Exercice[voisinages]

Soit $E$ un £evn. et $a∈E$.
Un \emph{voisinage} de $a$ est une partie $V ⊂ E$ telle qu'il existe un ouvert $U$ tel que $a ∈ U$ et $U ⊂ V$.
On note $\mathscr{V}(a)$ l'ensemble des voisinages de $a$.
\begin{enumerate}
\item Montrer que $V ∈ \mathscr{V}(a)$ £ssi. il existe $r>0$ tel que $B(a,r) ⊂ V$.
\item Soit $A$ une partie de $E$.
  Montrer que $A$ est un ouvert £ssi. $A$ est un voisinage de chacun de ses points.
\item Soit $(u_n)$ une suite à valeurs dans $E$.
  Montrer que $u_n \to ℓ$ £ssi. pour tout voisinage $V$ de $ℓ$, on a $u_n ∈ V$ à partir d'un certain rang.
\item Soit $A$ une partie de $E$, $a∈A$, $F$ un £evn. et $\Fn f AF$ une fonction.
  Montrer que $f$ est continue en $a$ £ssi. l'image réciproque par $f$ de tout voisinage de $f(a)$ est un voisinage de $a$.
\end{enumerate}

\end{document}
