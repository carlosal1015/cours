\documentclass{yann}
\yann{displaystyle}

\newcommand\me{e}
\newcommand\I{i}
\newcommand\Ei[1]{\me^{#1\I\pi}}
\DeclareMathOperator\Signe{signe}

\begin{document}
\title{Formulaire de trigonométrie}
\maketitle

\section{Relations entre les fonctions trigonométriques}

\begin{itemize}
\item
$\cos^2 x + \sin^2 x = 1$
\item
$\tan x   = \frac{\sin x}{\cos x}$
\item
$\cotan x = \frac{\cos x}{\sin x}$
\item
$1 + \tan^2 x   = \frac{1}{\cos^2 x}$
\item
$1 + \cotan^2 x = \frac{1}{\sin^2 x}$
\end{itemize}

\section{Formules d'Euler}

\begin{itemize}
\item
$\cos x     = \frac{\me^{\I x} + \me^{-\I x}}{2}$
\item
$\sin x     = \frac{\me^{\I x} - \me^{-\I x}}{2\I}$
\item
$\me^{\I x} = \cos x + \I \sin x$
\end{itemize}

Exemple d'utilisation: linéariser $\cos^2 (x) \sin^4 (x)$.
\[ \begin{aligned}
  & \cos^2 (x) \sin^4 (x) \\
  & \quad = \Pafrac{\Ei{}+\Ei{-}}{2}^2\cdot\Pafrac{\Ei{}-\Ei{-}}{2\I}^4 \\
  & \begin{split}
    \quad ={} &\frac{1}{2^6} \Pa{\Ei{2} + 2 +\Ei{-2} } \\
    &\quad \cdot \Pa{\Ei{4} - 4\Ei{2} + 6 - 4\Ei{-2} +\Ei{-4} } \end{split} \\
  & \begin{split}
    \quad ={} & \frac{1}{2^6} \Bigl( \Ei{6} - 2\Ei{4} -\Ei{2} + 4 \\
    &\qquad -\Ei{-2} - 2\Ei{-4} +\Ei{-6} \Bigr) \end{split} \\
  & \quad = \frac{\cos 6x - 2 \cos 4x - \cos 2x + 2}{32}
\end{aligned} \]

\section{Symétries}

\begin{itemize}
\item
$\cos(-x)          = \cos x$
\item
$\sin(-x)          = -\sin x$
\item
$\tan(-x)          = -\tan x$
\item
$\cos(\pi+x)         = -\cos x$
\item
$\sin(\pi+x)         = -\sin x$
\item
$\tan(\pi+x)         = \tan x$
\item
$\cos(x+n\pi)        = (-1)^n \cos x$
\item
$\cos(n\pi)          = (-1)^n$
\item
$\sin(x+n\pi)        = (-1)^n \sin x$
\item
$\sin(n\pi)          = 0$
\item
$\cos(\pi-x)         = -\cos x$
\item
$\sin(\pi-x)         = \sin x$
\item
$\cos\BigPa{\frac\pi2 - x} = \sin x$
\item
$\sin\BigPa{\frac\pi2 - x} = \cos x$
\item
$\tan\BigPa{\frac\pi2 - x} = \frac{1}{\tan x} = \cotan x$
\item
$\cos\BigPa{\frac\pi2 + x} = -\sin x = \cos'(x)$
\item
$\sin\BigPa{\frac\pi2 + x} = \cos x = \sin'(x)$
\item
$\tan\BigPa{\frac\pi2 + x} = -\frac{1}{\tan x} = -\cotan x$
\end{itemize}

\section{Formules de duplication}

\begin{itemize}
\item
$\cos 2\theta= \cos^2\theta- \sin^2\theta= 2\cos^2\theta- 1 = 1 - 2\sin^2\theta$
\item
$\sin 2\theta= 2 \cos\theta\sin\theta$
\item
$\tan 2\theta= \frac{2\tan\theta}{1-\tan^2\theta}$
\item
$\cos^2\theta= \frac{1+\cos 2\theta}{2}$
\item
$\sin^2\theta= \frac{1-\cos 2\theta}{2}$
\item
$1 + \cos\theta= 2 \cos^2 \BigPa{\frac{\theta}{2}}$
\item
$1 - \cos\theta= 2 \sin^2 \BigPa{\frac{\theta}{2}}$
\end{itemize}

\section{Formules d'addition}

\begin{itemize}
\item
$\cos(a+b)       = \cos a \cos b - \sin a \sin b$
\item
$\cos(a-b)       = \cos a \cos b + \sin a \sin b$
\item
$\sin(a+b)       = \sin a \cos b + \cos a \sin b$
\item
$\sin(a-b)       = \sin a \cos b - \cos a \sin b$
\item
$\tan(a+b)       = \frac{\tan a + \tan b}{1 - \tan(a)\tan(b)}$
\item
$\tan(a-b)       = \frac{\tan a - \tan b}{1 + \tan(a)\tan(b)}$
\item
$\sin p + \sin q =  2 \sin\BigPa{\frac{p+q}{2}} \cos\BigPa{\frac{p-q}{2}}$
\item
$\sin p - \sin q =  2 \cos\BigPa{\frac{p+q}{2}} \sin\BigPa{\frac{p-q}{2}}$
\item
$\cos p + \cos q =  2 \cos\BigPa{\frac{p+q}{2}} \cos\BigPa{\frac{p-q}{2}}$
\item
$\cos p - \cos q = -2 \sin\BigPa{\frac{p+q}{2}} \sin\BigPa{\frac{p-q}{2}}$
\end{itemize}

\section{Tangente de l'arc moitié}

\begin{itemize}
\item
$t     = \tan\BigPa{\frac{\theta}{2}}$
\item
$\cos\theta = \frac{1-t^2}{1+t^2}$
\item
$\sin\theta = \frac{2t}{1+t^2}$
\item
$\tan\theta = \frac{2t}{1-t^2}$
\end{itemize}

\section{Valeurs remarquables}

\begin{itemize}
\item
$\sin 0 = 0$, $\cos 0 = 1$, $\tan 0 = 0$
\item
$\sin\BigPa{\frac\pi6} = \frac12$, $\cos\BigPa{\frac\pi6} = \frac{\sqrt3}{2}$, $\tan\BigPa{\frac\pi6} = \frac{1}{\sqrt3}$
\item
$\sin\BigPa{\frac\pi4} = \frac{\sqrt2}2$, $\cos\BigPa{\frac\pi4} = \frac{\sqrt2}{2}$, $\tan\BigPa{\frac\pi4} = 1$
\item
$\sin\BigPa{\frac\pi3} = \frac{\sqrt3}{2}$, $\cos\BigPa{\frac\pi3} = \frac12$, $\tan\BigPa{\frac\pi3} = \sqrt3$
\item
$\sin\BigPa{\frac\pi2} = 1$, $\cos\BigPa{\frac\pi2} = 0$, $\tan\BigPa{\frac\pi2} = \infty$
\end{itemize}

Bonus:

\begin{itemize}
\item
$\cos\BigPa{\frac{2\pi}{5}} = \frac{\sqrt5-1}{4}$
\item
La valeur de $\cos\BigPa{\frac{2\pi}{17}}$ que Gauß a trouvée à 19 ans:
\end{itemize}
\[ \begin{aligned}
      & 16\cos\Pafrac{2\pi}{17} = -1 +\sqrt{17} +\sqrt{34-2\sqrt{17}} \\
      & \qquad + 2\sqrt{17 + 3\sqrt{17} -\sqrt{34-2\sqrt{17}} - 2\sqrt{34+2\sqrt{17}}}
            \end{aligned} \]

\section{Équations trigonométriques}

\begin{itemize}
\item
$\cos x = \cos y$
si et seulement si $(\exists k\in\Z\+ x = y + 2k\pi)$ ou $(\exists l\in\Z\+ x = -y + 2l\pi)$.
\item
$\sin x = \sin y$
si et seulement si $(\exists k\in\Z\+ x = y + 2k\pi)$ ou $(\exists l\in\Z\+ x =\pi-y + 2l\pi)$
\end{itemize}

\section{Fonctions réciproques}

\subsection{Arc cosinus}

Soit $\Fonction{c}{\intF{0,\pi}}{\intF{-1,1}}{x}{\cos x.}$

La fonction $c$ est bijective, on note $\arccos$ sa réciproque.

\begin{itemize}
\item
$\arccos$ est une fonction continue sur $\intF{-1,1}$
et de classe $\CC\infty$ sur $\intO{-1,1}$;
\item
$\forall x\in\intF{-1,1}$, $\cos\bigl(\arccos(x)\bigr) = x$;
\item
$\forall x\in\intF{0,\pi}$, $\arccos\bigl(\cos(x)\bigr) = x$,
mais ce n'est pas vrai pour $x\in\R\setminus\intF{0,\pi}$.
\end{itemize}

\subsection{Arc sinus}

Soit $\Fonction{s}{\left[ -\frac\pi2,\frac\pi2 \right]}{\intF{-1,1}}{x}{\sin x.}$

La fonction $s$ est bijective, on note $\arcsin$ sa réciproque.

\begin{itemize}
\item
$\arcsin$ est une fonction continue sur $\intF{-1,1}$
et de classe $\CC\infty$ sur $\intO{-1,1}$;
\item
$\forall x\in\intF{-1,1}$, $\sin\bigl( \arcsin(x) \bigr) = x$;
\item
$\forall x\in\intF{-\pi/2,\pi/2}$, $\arcsin\bigl( \sin(x) \bigr) = x$,
mais ce n'est pas vrai pour $x\in\R\setminus\intF{-\pi/2,\pi/2}$.
\end{itemize}

\subsection{Arc tangente}

Soit $\Fonction{t}{\left] -\frac\pi2,\frac\pi2 \right[}{\R}{x}{\tan x.}$

La fonction $t$ est bijective, on note $\arctan$ sa réciproque.

\begin{itemize}
\item
$\arctan$ est une fonction de classe $\CC\infty$ sur $\R$;
\item
$\forall x\in\R$, $\tan\bigl( \arctan(x) \bigr) = x$;
\item
$\forall x\in\intO{-\pi/2,\pi/2}$, $\arctan\bigl( \tan(x) \bigr) = x$,
mais ce n'est pas vrai pour $x\in\R\setminus\intO{-\pi/2,\pi/2}$.
\end{itemize}

\subsection{Identités}

\begin{itemize}
\item
$\arccos(x) + \arcsin(x)       =  \frac\pi2$
\item
$\arctan(x) + \arctan\BigPa{\frac1x} =  \Signe(x)\frac\pi2$
\end{itemize}

\section{Dérivées}

\begin{itemize}
\item
$\cos'(x)    = -\sin x$
\item
$\sin'(x)    = \cos x$
\item
$\tan'(x)    = 1 + \tan^2 x = \frac{1}{\cos^2 x}$
\item
$\cotan'(x)  = -1 - \cotan^2 x = -\frac{1}{\sin^2 x}$
\item
$\arccos'(x) = \frac{-1}{\sqrt{1-x^2}}$
\item
$\arcsin'(x) = \frac{1}{\sqrt{1-x^2}}$
\item
$\arctan'(x) = \frac{1}{1+x^2}$
\end{itemize}

\section{Trigonométrie hyperbolique}

Il faut connaître: $\ch^2 x - \sh^2 x = 1$.
Les autres identités se déduisent de
$\cos(\I z) = \ch z$ et $\sin(\I z) = \I \sh z$.

Par exemple,
\[ \begin{aligned} \ch 2\theta &= \cos(2\I\theta)
  = 1 - 2\sin^2(\I\theta)
  = 1 + 2\sh^2\theta \\
 \th 2\theta &= -\I\tan(2\I\theta)
  = -\I\cdot\frac{2\tan(\I\theta)}{1 - \tan^2(\I\theta)}
  = \frac{2\th\theta}{1 + \th^2\theta}
  \end{aligned} \]

\section{Polynômes de Tchebychev}

\subsection{Polynômes de première espèce}

\begin{itemize}
\item
$T_0(X)      = 1$
\item
$T_1(X)      = X$
\item
$T_{n+2}(X)  = 2X \, T_{n+1}(X) - T_n(X)$
\item
$T_n(X)$ est un polynôme de degré $n$ et de coefficient dominant $2^{n-1}$.
\item
$T_n(\cos\theta)  = \cos(n\theta)$
\item
$T_n(\ch\theta) = \ch(n\theta)$
\end{itemize}

Par exemple, $T_3(X) = 4X^3 - 3X$ donc \[ \cos 3x = 4\cos^3 x - 3 \cos x. \]

\subsection{Polynômes de seconde espèce}

\begin{itemize}
\item
$U_0(X)      = 1$
\item
$U_1(X)      = 2X$
\item
$U_{n+2}(X)  = 2X \, U_{n+1}(X) - U_n(X)$
\item
$U_n(X)$ est un polynôme de degré $n$ et de coefficient dominant $2^n$.
\item
$U_n(\cos\theta)  \sin\theta  = \sin\bigl((n+1)\theta\bigr)$
\item
$U_n(\ch\theta) \sh\theta = \sh\bigl((n+1)\theta\bigr)$
\end{itemize}

\end{document}
