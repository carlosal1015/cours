% Time-stamp: <2018-01-27 22:06>
\documentclass{yann}

\newcommand{\FIK}{\mathcal{F}(I,𝕂)}
\newcommand{\fn}{(f_n)_{n∈ℕ}}
\newcommand{\Sfn}{∑_n f_n}

\begin{document}
\title{Théorèmes de Lebesgue}
\maketitle

% -----------------------------------------------------------------------------

Ces deux théorèmes du programme proviennent de la théorie de l'intégrale de Lebesgue,
qui n'est pas au programme des classes préparatoires.
Ils fournissent des versions beaucoup plus puissantes des théorèmes de permutations limite/intégrale et somme/intégrale.
En particulier, ils permettent d'effectuer des permutations avec des intégrales généralisées, et non plus seulement avec des intégrales propres sur un segment $[a,b]$.

% -----------------------------------------------------------------------------
\section{Théorème de convergence dominée}

Soit $\fn$ une suite de fonctions de $I$ dans $𝕂$.
On suppose que:
\begin{enumerate}[label={\emph{\roman*)}}]
\item
pour tout $n∈ℕ$, $f_n \colon I \to𝕂$ est continue par morceaux;
\item
$f \colon I \to𝕂$ est continue par morceaux;
\item
la suite de fonctions $\fn$ converge simplement vers $f$ sur $I$;
\item
\emph{Hypothèse de domination.}
  Il existe $φ\colon I \to \Rp$ continue par morceaux telle que:
  \begin{enumerate}[label={\emph{\alph*)}}]
  \item
$∀n∈ℕ\+∀x∈I\+ \Abs{f_n(x)}≤φ(x)$;
  \item
$φ$ est \emph{intégrable sur $I$}.
  \end{enumerate}
\end{enumerate}

Alors:
\begin{itemize}
\item
pour tout $n∈ℕ$, $f_n$ est intégrable sur $I$
\item
$f$ est intégrable sur $I$
\item
$\DS é\lim_{n∞}∫_I f_n =∫_I f$.
\end{itemize}

% -----------------------------------------------------------------------------
\section{Théorème d'intégration terme à terme}

Soit $\Sfn$ une série de fonctions de $I$ dans $𝕂$.
On suppose que:
\begin{enumerate}[label={\emph{\roman*)}}]
\item
pour tout $n∈ℕ$, $f_n$ est continue par morceaux et intégrable sur $I$;
\item
la série de fonctions $∑_n f_n$ converge simplement sur $I$
  vers une fonction $f \colon I \to𝕂$, elle-même continue par morceaux;
\item
\emph{la série numérique $\DS ∑_n ∫_I \Abs{f_n}$ converge.}
\end{enumerate}

Alors:
\begin{itemize}
\item
$f$ est intégrable sur $I$
\item
la série numérique $\DS ∑_n ∫_I f_n$ converge
\item
$\DS ∫_I f =∑_{n=0}^{+∞}∫_I f_n$, autrement dit:
  \[ ∫_I ∑_{n=0}^{+∞} f_n =∑_{n=0}^{+∞} ∫_I f_n. \]
\end{itemize}

% -----------------------------------------------------------------------------
\section{Un peu de topologie}

\Para{Critère séquentiel}

Soit $\Fn{f}{I}{ℝ}$, $a$ un point de $I$ ou une extrémité de $I$ (éventuellement $±∞$)
et $ℓ$ un réel (ou $±∞$).
£LCSSE.
\begin{enumerate}[label={\emph{\roman*)}}]
\item
  $\DS \lim_{x \to a} f(x) = ℓ$;
\item
  pour toute suite $(u_n)$ à valeurs dans $I$
  telle que $u_n \Toninf a$,
  on a $f(u_n) \Toninf ℓ$.
\end{enumerate}

% -----------------------------------------------------------------------------
\section{Exercices}

\Exercice

Soit $\DS f_n(x) = \frac{n\sin(x/n)}{x(1+x^2)}$ et $\DS a_n = ∫_0^{+∞} f_n$.
Déterminer la limite de la suite $(a_n)$.

\Exercice

Soit $\Fn{f}{\Rp}{ℝ}$ continue bornée telle que $f(0)≠0$.
Déterminer un équivalent de \[ u_n = ∫_0^{+∞} f(x) e^{-nx} \D x. \]

\Exercice

Soit $\Fonction{f_n}{ℝ}{ℝ}{x}{\frac{1}{1+(x-n)^2}}$
\begin{enumerate}
\item
Montrer que la suite $\fn$ converge simplement et déterminer sa limite $f$.
\item
Calculer $\DS ∫_ℝ f_n$ et $\DS ∫_ℝf$.
\item
Expliquer ce résultat.
\end{enumerate}

\Exercice

Montrer que:
\[ ∫_0^{+∞} \frac{\sin x}{e^x - 1} \D x = ∑_{n=1}^{+∞} \frac{1}{n^2+1} \]

\emph{Astuces:} on pourra remarquer que, pour $x > 0$:
\begin{itemize}
\item
$\DS \frac{1}{e^x-1} = \frac{e^{-x}}{1-e^{-x}} = ∑_{n=1}^{+∞} e^{-nx}$.
\item
$\Abs{\sin(x)} ≤ x$.
\end{itemize}

\Exercice
\begin{enumerate}
\item
Montrer que $\DS ∫_0^{+∞} \frac{√t}{e^t - 1} \D t
  = \frac{√π}{2} ∑_{n=1}^{+∞} \frac{1}{n^{3/2}}$.

  On admettra que $\DS ∫_0^{+∞} e^{-x^2} \D x = \frac{√π}{2}$;
  il s'agit de l'\emph{intégrale de Gauß}.
\item
Plus généralement, montrer que pour tout $x > 1$, on a:
  \[ ∫_0^{+∞} \frac{t^{x-1}}{e^t-1} \D t = Γ(x)ζ(x) \]
  où $\DS ζ(x) = ∑_{n=1}^{+∞} \frac{1}{n^x}$
  et $\DS Γ(x) = ∫_0^{+∞} t^{x-1} e^{-t} \D t$.
\end{enumerate}

\Exercice

Pour $n∈\Ns$, on pose $\DS f_n(x) = \BigPa{1+\frac{x^2}{n}}^{-n}$.
Soit $\DS u_n = ∫_0^{+∞} f_n$.
\begin{enumerate}
\item
Montrer que $f_n$ est continue et intégrable sur $\Rp$.
\item
Déterminer la limite simple $f$ de la suite de fonctions $\fn$.
\item
\begin{enumerate}
\item
Justifier l'inégalité suivante, valable $∀(a,b)∈(\Rps)^2$
  et $∀λ∈[0,1]$:
  \[ \ln\Big(λa + (1-λ)b\Big)≥λ\ln a + (1-λ) \ln b \]
\item
Appliquer l'inégalité précédente avec $a=1+x^2$, $b=1$ et $λ=\frac1n$.
\item
En déduire que $∀n≥1$, $∀x∈\Rp$, $f_n(x)≤f_1(x)$.
\end{enumerate}
\item
En déduire que: $\DS u_n \Toninf∫_0^{+∞} e^{-x^2} \D x$.
\item
On note $\DS W_n = ∫_0^{\fracπ2} \cos^n(θ) \Dθ$ la
  $n$-ième intégrale de Wallis.
  Exprimer $u_n$ en fonction des intégrales de Wallis; on pourra
  effectuer le changement de variables $x=√{n}\tanθ$.
\item
On rappelle (cf. TD sur les séries numériques) que
  $W_n \sim √{π/2n}$.
  En déduire la valeur de l'intégrale de Gauß $\DS G = ∫_0^{+∞} e^{-x^2} \D x$.
\end{enumerate}

\Exercice

Soit $\Fn{f}{[0,1]}{ℝ}$ continue.
Pour $x > 0$, on pose $\DS g(x) = ∫_0^1 \frac{xf(t)}{x^2+t^2} \D t$.
\begin{enumerate}
\item
  Soit $(x_n)_{n∈ℕ}$ est une suite de réels strictement positifs qui tend vers 0.
  Montrer que $g(x_n) \Toninf \frac{πf(0)}{2}$.
  On pourra commencer par le changement de variables $t = x_n y$.
\item
En déduire que $\DS \lim_{0^+} g = \frac{πf(0)}{2}$.
\end{enumerate}

\Exercice

On pose $\DS u_n = ∫_0^1 \frac{\D x}{1+x^n}$.
\begin{enumerate}
\item
  Déterminer la limite $ℓ$ de la suite $(u_n)_{n∈ℕ}$.
\item
  Exprimer $n(1-u_n)$ sous forme d'une intégrale et effectuer le changement
  de variables $y=x^n$.
\item
  En déduire un développement asymptotique de $u_n$ à deux termes.
\item
  Bonus:
  montrer que \[ u_n = 1 - \frac{\ln 2}{n} + \frac{\pi^2}{12n^2} + o\BigPa{\frac{1}{n^2}}. \]
\end{enumerate}

\Exercice

Pour $n∈ℕ$, on pose \[ f_n(x) = \Pa{1-\frac xn}^{n-1} \ln(x) \text{ si } x∈\intOF{0,n} \]
et $f_n(x) = 0$ sinon.
\begin{enumerate}
\item
  Montrer que la suite de fonctions $(f_n)$ converge simplement sur $\Rps$ vers une fonction $f$ à préciser.
\item
  Montrer que $\DS ∫_0^n f_n \Toninf ∫_0^{+∞} f$.
\item
  Calculer $\DS ∫_0^n f_n$.
\item
  Sachant que
  $\DS ∑_{k=1}^n \frac1k = \ln(n) + γ + o(1)$,
  montrer que
  $\DS ∫_0^{+∞} e^{-x} \ln(x) \D x = -γ$.
\end{enumerate}

\Exercice

Pour $n∈ℕ$, on pose \[ a_n = ∫_0^{π/4} \tan^n(t) \D t. \]
Étudier la série entière $∑ a_n x^n$:
rayon de convergence,
étude aux bornes du domaine de définition,
calcul de la somme.

\Exercice

Soit $\DS u_n = \int_0^1 \frac{\D x}{1+x+x^2+\dots+x^n}$.
\begin{enumerate}
\item
    Déterminer la limite $\ell$ de la suite $(u_n)$.
\item
    Déterminer un équivalent de $u_n - \ell$.
\item
    Bonus: déterminer un terme de plus du développment asymptotique de $u_n$
\end{enumerate}

\end{document}
