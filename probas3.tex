\documentclass{yann}
\usepackage{dsfont}

\renewcommand{\T}{\mathscr{T}}
\newcommand{\Part}{\mathcal{P}}
\newcommand{\Pro}{\bigl(Ω,\T\bigr)}
\newcommand{\Prob}{\bigl(Ω,\T,ℙ\bigr)}

\begin{document}
\title{Probabilités discrètes}
\maketitle

% -----------------------------------------------------------------------------
\section{Dénombrabilité}

\Para{Définitions}

Soit $X$ un ensemble.
\begin{itemize}
\item
On dit que $X$ est \emph{fini} s'il existe un entier naturel $n∈ℕ$
  et une application bijective $\Fn{φ}{\Dcro{1,n}}{X}$.
\item
On dit que $X$ est \emph{infini} si $X$ n'est pas fini.
\item
On dit que $X$ est \emph{dénombrable} s'il existe une application bijective $\Fn{φ}{ℕ}{X}$.
\item
On dit que $X$ est \emph{au plus dénombrable} si $X$ est fini ou dénombrable.
\end{itemize}

\Para{Lemme} \label{lemme:ensembles-finis}

Soit $(n,p)∈ℕ^2$.
Si $\Fn{φ}{\Dcro{1,n}}{\Dcro{1,p}}$ est bijective, alors $n=p$.

cf. exercice~\ref{exo:ensembles-finis}.

\Para{Définition}

Soit $X$ un ensemble fini. On note $\Card(X)$ l'unique entier $n∈ℕ$
tel qu'il existe une application bijective $\Fn{φ}{\Dcro{1,n}}{X}$.

\Para{Lemme}

Soit $A$ une partie de $ℕ$. Alors
\begin{itemize}
\item
si $A$ est majorée, alors $A$ est finie;
\item
si $A$ est n'est pas majorée, alors $A$ est dénombrable.
\end{itemize}

\Para{Théorème}

Soit $X$ un ensemble.
\begin{itemize}
\item
Si $X$ est dénombrable, alors il existe une suite $(x_n)$ injective telle que $X = \Ensemble{x_n}{n∈ℕ}$.
\item
Si $X$ est au plus dénombrable et non vide, alors il existe une suite $(x_n)$ telle que $X = \Ensemble{x_n}{n∈ℕ}$.
\item
Si $X = \Ensemble{x_n}{n∈ℕ}$ où $(x_n)$ est une suite, alors $X$ est au plus dénombrable et non vide.
\end{itemize}

\Para{Théorème}
\begin{itemize}
\item
$ℕ$, $ℤ$ et $ℚ$ sont dénombrables.
\item
Si $X$ est dénombrable et $k≥1$, alors $X^k$ est dénombrable.
\item
$\mathcal{P}(ℕ)$, $ℝ$ et $\{0,1\}^ℕ$ ne sont pas dénombrables.
\end{itemize}

\Para{Proposition}

Soit $(X_i)_{i∈I}$ est une famille d'ensembles. On suppose que
\begin{itemize}
\item
$I$ est au plus dénombrable et
\item
pour tout $i∈I$, $X_i$ est au plus dénombrable.
\end{itemize}

Alors l'union $⋃_{i∈I} X_i$ est au plus dénombrable.

% -----------------------------------------------------------------------------
\section{Tribus}

\Para{Définition}

Soit $Ω$ un ensemble.
Une \emph{tribu} (ou $σ$-algèbre) sur $Ω$ est une partie $\T$ de $\Part(Ω)$ vérifiant les propriétés suivantes:
\begin{itemize}
\item
$∅∈\T$
\item
$∀A ∈\T$, $Ω∖A ∈\T$
\item
Si $(A_n)_{n∈ℕ}$ est une suite d'éléments de $\T$, alors $⋃_{n∈ℕ} A_n∈\T$.
\end{itemize}

\Para{Proposition}

Soit $Ω$ un ensemble.
\begin{itemize}
\item
$\{ ∅, Ω\}$ est une tribu sur $Ω$.
\item
$\Part(Ω)$ est une tribu sur $Ω$.
\end{itemize}

\Para{Définitions}

Soit $Ω$ un ensemble non vide et $\T$ une tribu sur $Ω$.
\begin{itemize}
\item
Le couple $\Pro$ s'appelle un \emph{espace probabilisable}.
\item
Un \emph{événement} est un élément de $\T$.
\end{itemize}

\Para{Proposition}

Soit $\Pro$ un espace probabilisable.
\begin{itemize}
\item
$∅$ est un événement, dit \og{}événement impossible\fg{}.
\item
$Ω$ est un événement, dit \og{}événement certain\fg{}.
\item
Si $A$ est un événement, $\bar A = Ω∖A$ est un événement, dit \og{}non-$A$\fg{}.
\item
Si $A$ et $B$ sont des événements, alors $A∖B$ est un événement.
\item
Si $A_1, \dots, A_n$ sont des événements, alors

  \begin{itemize}
  \item
$⋃_{i=1}^n A_i$ est un événement appelé \og{}disjonction de $A_1, \dots, A_n$\fg{}.
  \item
$⋂_{i=1}^n A_i$ est un événement appelé \og{}conjonction de $A_1, \dots, A_n$\fg{}.
  \end{itemize}
\item
Si $(A_n)_{n∈ℕ}$ est une suite d'événements, alors

  \begin{itemize}
  \item
$⋃_{n∈ℕ} A_n$ est un événement appelé \og{}disjonction de $A_n$ pour $n∈ℕ$\fg{}.
  \item
$⋂_{n∈ℕ} A_n$ est un événement appelé \og{}conjonction de $A_n$ pour $n∈ℕ$\fg{}.
  \end{itemize}
\end{itemize}

% -----------------------------------------------------------------------------
\section{Probabilité}

\subsection{Généralités}

\Para{Définition}

Soit $\Pro$ un espace probabilisable.
Soit $A$ et $B$ deux événements.
On dit que $A$ et $B$ sont \emph{incompatibles} si et seulement si $A∩B = ∅$.

\Para{Définition}

Soit $\Pro$ un espace probabilisable.
Une \emph{probabilité} sur $\Pro$ est une application
$\Fn{ℙ}{\T}{[0,1]}$ telle que
\begin{enumerate}
\item
$ℙ(Ω) = 1$
\item
Pour toute suite d'événements $(A_n)_{n∈ℕ}$ deux à deux incompatibles,
  la série $∑_n ℙ(A_n)$ converge et
  \[ ℙ\left( ⋃_{n∈ℕ} A_n \right) = ∑_{n∈ℕ}ℙ(A_n) \]
  Cette propriété s'appelle la \emph{$σ$-additivité}.
\end{enumerate}

\Para{Remarque}

Pourquoi ne pas systématiquement choisir $\T = \Part(Ω)$,
ce qui éviterait de parler de tribu et simplifierait sensiblement les définitions?
Parce que cela ne fonctionne malheureusement pas, cf. le théorème ci-dessous.

\Para{Théorème}[hors-programme]

Soit $Ω= \intF{0,1}$ et $\T = \Part(Ω)$.
Il n'existe pas de probabilité $ℙ$ sur $\Pro$ telle que
\[ ∀(a,b) ∈\intF{0,1}^2 \+ a≤b \implies ℙ(\intF{a,b}) = b - a. \]

Par contre, il existe une tribu $\mathcal{B}$ (dite tribu des boréliens) sur $Ω$
contenant tous les intervalles de la forme $\intF{a,b}$
et une probabilité $ℙ$ sur $(Ω,\mathcal{B})$ telle que
\[ ∀(a,b) ∈\intF{0,1}^2 \+ a≤b \implies ℙ(\intF{a,b}) = b - a. \]

\Para{Définition}

Un \emph{espace probabilisé} est un triplet $\Prob$
où $\Pro$ est un espace probabilisable
et $ℙ$ une probabilité $ℙ$ sur cet espace.

\Para{Définitions}

Soit $\Prob$ un espace probabilisé.
Soit $A$ un événement.
\begin{itemize}
\item
$A$ est dit \emph{négligeable} si et seulement si $ℙ(A) = 0$.
\item
$A$ est dit \emph{presque certain} si et seulement si $ℙ(A) = 1$.
\end{itemize}

\subsection{Propriétés}

\Para{Proposition}

Soit $\Prob$ un espace probabilisé, $A$ et $B$ deux événements.
On a
\begin{enumerate}
\item
$ℙ(∅) = 0$.
\item
$ℙ(A∪B) = ℙ(A) + ℙ(B)$ si $A∩B = ∅$.
\item
$ℙ(\bar A) = 1 - ℙ(A)$.
\item
$ℙ(A∪B) = ℙ(A) + ℙ(B) - ℙ(A∩B)$.
\item
$ℙ(A)≤ℙ(B)$ si $A⊂B$.
\end{enumerate}

\Para{Théorème}[sous-additivité]

Soit $\Prob$ un espace probabilisé.
Soit $(A_n)_{n∈I}$ une suite d'événements où $I⊂ℕ$.
Alors
\[ ℙ\left(⋃_{i∈I} A_i \right) ≤∑_{i∈I} ℙ(A_i) \]
La série de droite ne converge pas nécessairement; on pose
$∑_{i∈I} ℙ(A_i) = +∞$ si elle diverge.

\subsection{Théorèmes de continuité}

\Para{Définitions}

Soit $\Prob$ un espace probabilisé.
Soit $(A_n)_{n∈ℕ}$ est une suite d'événements, c.-à-d. une suite à valeurs dans $\T$.
\begin{itemize}
\item
On dit que $(A_n)_{n∈ℕ}$ est \emph{croissante} (au sens de l'inclusion)
  si et seulement si $∀n∈ℕ$, $A_n ⊂A_{n+1}$.
\item
On dit que $(A_n)_{n∈ℕ}$ est \emph{décroissante} (au sens de l'inclusion)
  si et seulement si $∀n∈ℕ$, $A_{n+1} ⊂A_n$.
\end{itemize}

\Para{Théorème}[continuité croissante]

Soit $\Prob$ un espace probabilisé.
Soit $(A_n)_{n∈ℕ}$ une suite croissante d'événements.
Alors
\[ ℙ(A_n) \Toninf ℙ\left(⋃_{n∈ℕ} A_n \right). \]

\Para{Théorème}[continuité décroissante]

Soit $\Prob$ un espace probabilisé.
Soit $(A_n)_{n∈ℕ}$ une suite décroissante d'événements.
Alors
\[ ℙ(A_n) \Toninf ℙ\left(⋂_{n∈ℕ} A_n \right). \]

\subsection{Définition d'une probabilité sur un univers dénombrable}

\Para{Théorème}

Soit $Ω$ un ensemble infini \emph{dénombrable}.
On peut écrire $Ω= \Ensemble{w_n}{n∈ℕ}$ où $(ω_n)$ est une suite injective.
Soit $(p_n)$ une suite de réels positifs telle que la série $∑_{n∈ℕ} p_n$ converge et soit de somme $1$.
Alors il existe une unique probabilité sur $\bigl(Ω, \Part(Ω) \bigr)$ telle que
\[ ∀n∈ℕ\+ ℙ\bigl(\{ ω_n \}\bigr) = p_n. \]
Autrement dit, une probabilité sur un ensemble dénombrable est entièrement caractérisée
par sa valeur sur les singletons.

% -----------------------------------------------------------------------------
\section{Indépendance}

\Para{Définition}

Soit $\Prob$ un espace probabilisé.
Soit $(A_i)_{n∈I}$ une famille d'événements.
\begin{itemize}
\item
Les $(A_i)_{i∈I}$ sont \emph{deux à deux indépendants}
  si et seulement si pour tous $(i,j)∈I^2$ tels que $i≠j$, on a $ℙ(A_i∩A_j)=ℙ(A_i)ℙ(A_j)$.
\item
La famille $(A_i)_{i∈I}$ sont des événements \emph{mutuellement indépendants}
  si pour toute partie finie $F⊂I$, on a
  \[ ℙ\left( ⋂_{i∈F} A_i \right) = ∏_{i∈F} ℙ(A_i) \]
\end{itemize}

\Para{Proposition}

Soit $\Prob$ un espace probabilisé.
Soit $(A_n)_{n∈ℕ}$ des événements mutuellement indépendants.
\begin{itemize}
\item
Si $I⊂ℕ$, alors $(A_i)_{i∈I}$ sont des événements mutuellement indépendants.
\item
Si pour tout $i∈ℕ$, $B_i ∈\bigl\{ ∅, A_i, \bar A_i,Ω\bigr\}$,
  alors $(B_n)_{n∈ℕ}$ sont des événements mutuellement indépendants.
\end{itemize}

% -----------------------------------------------------------------------------
\section{Probabilité conditionnelle}

\Para{Proposition-Définition}

Soit $\Prob$ un espace probabilisé.
Soit $A$ un événement tel que $ℙ(A)>0$.
L'application
\[ \Fonction{ℙ_A}{\T}{[0,1]}{B}{\frac{ℙ(A∩B)}{ℙ(A)}} \]
est une probabilité sur $\Pro$
appelée \emph{probabilité conditionnellement à $A$},
ou \emph{probabilité sachant $A$}.

On note $ℙ(B|A) = ℙ_A(B)$.

\Para{Proposition}[lien avec l'indépendance]

Soit $\Prob$ un espace probabilisé.
Soit $A$ un événement de probabilité non nulle
et $B$ un événement quelconque.
Alors les événements $A$ et $B$ sont indépendants si et seulement si $ℙ(B|A)=ℙ(B)$.

\Para{Définitions}

Soit $\Pro$ un espace probabilisable.
Soit $(A_n)_{n∈ℕ}$ une famille d'événements.
\begin{enumerate}
\item
On dit que $(A_n)$ est un \emph{système complet d'événements} £ssi.
  \begin{itemize}
  \item
$⋃_{n∈ℕ} A_n = Ω$;
  \item
$∀(n,p)∈ℕ^2$, $n≠p \implies A_n∩A_p = ∅$.
  \end{itemize}

\item
On dit que $(A_n)$ est un \emph{système quasi-complet d'événements} £ssi.
  \begin{itemize}
  \item
$ℙ(⋃_{n∈ℕ} A_n) = 1$;
  \item
$∀(n,p)∈ℕ^2$, $n≠p \implies ℙ(A_n∩A_p) = 0$.
  \end{itemize}
\end{enumerate}

\Para{Formule des probabilités totales}

Soit $\Prob$ un espace probabilisé.
Soit $(A_n)_{n∈ℕ}$ un système quasi-complet d'événements de probabilités non nulles.
Pour tout événement $B$, on a
\[ ℙ(B) = ∑_{n≥0} ℙ(B|A_n)ℙ(A_n). \]

\Para{Formule de Bayes}

Soit $\Prob$ un espace probabilisé.
Soit $(A_n)_{n∈ℕ}$ un système quasi-complet d'événements de probabilités non nulles.
Pour tout événement $B$ de probabilité non nulle,
et pour tout $k∈ℕ$, on a
\[ ℙ(A_k|B) = \frac{ ℙ(B|A_k)ℙ(A_k) }{ \DS ∑_{n≥0} ℙ(B|A_n)ℙ(A_n) }. \]

% -----------------------------------------------------------------------------
\section{Exercices}

\Exercice \label{exo:ensembles-finis}

Soit $H_p$ la propriété
\og{}pour tout $n∈ℕ$ et pour toute application $\Fn{φ}{\Dcro{1,n}}{\Dcro{1,p}}$ injective, on a $n≤p$\fg{}.
\begin{enumerate}
\item
Montrer $H_0$ et $H_1$.
\item
Montrer que $H$ est héréditaire.
\item
En déduire le lemme~\ref{lemme:ensembles-finis}.
\end{enumerate}

\Exercice

Montrer que $ℚ[X]$ et $ℚ(X)$ sont dénombrables.

\Exercice

Soit $X$ un ensemble.
\begin{enumerate}
\item
On suppose $X$ infini.

  \begin{enumerate}
  \item
Montrer qu'il existe une suite $(x_n)$ injective à valeurs dans $X$.
  \item
En déduire qu'il existe une bijection de $X$ dans $X∖\{ x_0 \}$.
  \end{enumerate}
\item
Réciproquement, on suppose qu'il existe $Y⊂X$ avec $Y≠X$
  et une application $\Fn{φ}{X}{Y}$ bijective.
  Montrer que $X$ est infini.
\end{enumerate}

Ainsi, on a démontré qu'un ensemble est infini si et seulement si il est en bijection avec une de ses parties strictes.

\Exercice

Soit $Ω$ un ensemble,
$Z$ une partie de $Ω$,
$(X_i)_{i∈I}$ et $(Y_i)_{i∈I}$ deux familles de parties de $Ω$.
\begin{enumerate}
\item
$\Pa{ ⋃_{i∈I} X_i } ∪\Pa{ ⋃_{i∈I} Y_i } = ⋃_{i∈I} (X_i∪Y_i)$
\item
$\Pa{ ⋂_{i∈I} X_i } ∩\Pa{ ⋂_{i∈I} Y_i } = ⋂_{i∈I} (X_i∩Y_i)$
\item
$Ω∖\Pa{ ⋃_{i∈I} X_i } = ⋂_{i∈I} (Ω∖X_i)$
\item
$Ω∖\Pa{ ⋂_{i∈I} X_i } = ⋃_{i∈I} (Ω∖X_i)$
\item
$⋃_{i∈I} (X_i∖Z) = \Pa{ ⋃_{i∈I} X_i } ∖Z$
\item
$⋃_{i∈I} (Z∖X_i) = Z ∖\Pa{ ⋂_{i∈I} X_i }$
\item
$⋂_{i∈I} (X_i∖Y_i) = \Pa{ ⋂_{i∈I} X_i } ∖\Pa{ ⋃_{i∈I} Y_i }$
\item
Si $I =ℕ$, montrer que

  \begin{enumerate}
  \item
$⋃_{n≥0} \Pa{ ⋃_{k=0}^n A_k } = ⋃_{n≥0} A_n$
  \item
$⋂_{n≥0} \Pa{ ⋂_{k=0}^n A_k } = ⋂_{n≥0} A_n$
  \end{enumerate}
\end{enumerate}

\Exercice

Soit $Ω= \{ a,b,c \}$.
\begin{enumerate}
\item
Déterminer la plus petite tribu contenant $\{ a \}$.
\item
Déterminer la plus petite tribu contenant $\{ a \}$ et $\{ b \}$.
\end{enumerate}

\Exercice

Soit $Ω$ un ensemble, $\T$ une tribu sur $Ω$ et $Ω'$ une partie de $Ω$.
On pose $\T' = \Ensemble{A∩Ω'}{A∈\T}$.
Montrer que $\T'$ est une tribu sur $Ω'$.

\Exercice

Soit $Ω$ un ensemble et $\Fn{f}{Λ}{Ω}$ une application.
\emph{Rappel:} pour tout $X⊂Ω$, on note $f^{-1}(X) = \Ensemble{x∈Λ}{f(x)∈X}$.
Démontrer les formules suivantes, où
$Y$ est une partie de $Ω$ et $(X_i)_{i∈I}$ une famille de parties de $Ω$:
\begin{enumerate}
\item
$f^{-1}(Ω∖Y) = Λ∖f^{-1} (Y)$
\item
$f^{-1} \Pa{ ⋃_{i∈I} X_i } = ⋃_{i∈I} f^{-1} (X_i)$
\item
$f^{-1} \Pa{ ⋂_{i∈I} X_i } = ⋂_{i∈I} f^{-1} (X_i)$
\end{enumerate}

\Exercice[image réciproque]

Soit $\Fn{f}{Λ}{Ω}$ une application et $\T$ une tribu sur $Ω$.
On pose $\mathcal{U} = \Ensemble{ f^{-1}(A) }{ A∈\T }$.
Montrer que $\mathcal{U}$ est une tribu sur $Λ$.

\Exercice

Soit $Ω$ un ensemble infini.
\begin{enumerate}
\item
Soit $\T = \Ensemble{A⊂Ω}{ \text{$A$ ou $\bar A$ est au plus dénombrable} }$.
  Montrer que $\T$ est une tribu sur $Ω$.
\item
Soit $\mathcal{U} = \Ensemble{A⊂Ω}{ \text{$A$ ou $\bar A$ est fini} }$.
  Montrer que $\mathcal{U}$ n'est pas une tribu sur $Ω$.
\end{enumerate}

\Exercice

Soit $\Prob$ un espace probabilisé
et $(A_n)_{n∈ℕ}$ une suite d'événements presque certains.
Montrer que $⋂_{n∈ℕ} A_n$ est presque certain.
Expliquer ce résultat en langage courant.

\Exercice

Soit $(A_n)_{n∈ℕ}$ une suite d'événements telle que
\[ ∀(n,p)∈ℕ^2 \+ n≠p \implies ℙ(A_n∩A_p) = 0. \]
Montrer que la série $∑_n ℙ(A_n)$ converge et que
\[ ℙ(⋃_{n∈ℕ} A_n) = ∑_{n∈ℕ} ℙ(A_n). \]

\Exercice

Soit $Ω = \cro{0,1}^ℕ$.
On pose $A_n = \Ensemble{ω∈Ω} { ω_n = 1 }$.
On admet qu'il existe une tribu $\T$ sur $Ω$ qui contient tous les
$A_n$ et une probabilité $ℙ$ sur $\Pro$ telle que $ℙ(A_n) = \frac12$.
Les outils permettant de prouver cela ne sont pas au programme en CPGE.
\begin{enumerate}
\item
Expliquer pourquoi $\Prob$ modélise le lancé d'une infinité de pièces équilibrées indépendantes.
\item
Soit $A = ⋂_{n∈ℕ} A_n$.
  Que représente $A$?
\item
Montrer que $A≠∅$ mais que $ℙ(A) = 0$.
\end{enumerate}

\Exercice

On considère un singe devant un clavier d'ordinateur,
qui appuie sur les touches au hasard, une touche chaque seconde, pour l'éternité.
Montrer qu'il est presque certain que ce singe produira l'intégrale de George R. R. Martin.

\Exercice

Soit $Ω = \intF{0,1}$, $\B$ la tribu des boréliens sur $Ω$ évoquée plus haut et $ℙ$
telle que $ℙ\bigl( \intF{a,b} \bigr) = b-a$ si $0≤a≤b≤1$.
\begin{enumerate}
\item
Montrer que $ℙ(ℚ∩[0,1]) = 0$.
\item
Soit $X$ un réel choisi aléatoirement, de façon uniforme dans $[0,1]$.
  Montrer que $X$ est presque sûrement irrationnel.
\end{enumerate}

% -----------------------------------------------------------------------------
\section{Exercices plus avancés}

\Exercice[limite inférieure et supérieure d'une suite réelle]

Soit $(u_n)_{n∈ℕ}$ une suite réelle.
\begin{enumerate}
\item
Soit $\overlineℝ=ℝ∪\{±∞\}$.
  Montrer que toute suite monotone à valeur dans $\overlineℝ$
  a une limite dans $\overlineℝ$.
\item
On pose $v_n = \inf \Ensemble{ u_k }{ k≥n }$.
  Montrer que $(v_n)$ existe dans $ℝ∪\{ -∞\}$ et que $(v_n)$ est croissante.

  On note alors $\liminf_\ninf u_n$ la limite de la suite $(v_n)$ dans $\overline ℝ$,
  et on l'appelle \emph{limite inférieure} de la suite $(u_n)_{n∈ℕ}$.
\item
On pose $w_n = \sup \Ensemble{ u_k }{ k≥n }$.
  Montrer que $(w_n)$ existe dans $ℝ∪\{ +∞\}$ et que $(w_n)$ est décroissante.

  On note alors $\limsup_\ninf u_n$ la limite de la suite $(w_n)$ dans $\overline ℝ$,
  et on l'appelle \emph{limite supérieure} de la suite $(u_n)_{n∈ℕ}$.
\item
Déterminer $\liminf_\ninf (-1)^n$ et $\limsup_\ninf (-1)^n$.
\item
Montrer que $\liminf u_n ≤\limsup u_n$.
\item
Montrer que la suite $(u_n)$ admet une limite dans $\overline ℝ$
  si et seulement si $\liminf u_n = \limsup u_n$, et que dans ce cas les trois limites sont égales.
\item
Montrer que si $(u_n)$ est une suite bornée, alors $\liminf u_n$ et $\limsup u_n$ sont dans $ℝ$.
\item
Montrer que:

  \begin{enumerate}
  \item
$\limsup u_n = +∞$ si et seulement si $(u_n)$ n'est pas majorée.
  \item
$\liminf u_n = -∞$ si et seulement si $(u_n)$ n'est pas minorée.
  \item
$\limsup u_n = -∞$ si et seulement si $u_n \to -∞$.
  \item
$\liminf u_n = +∞$ si et seulement si $u_n \to +∞$.
  \end{enumerate}
\end{enumerate}

\Exercice[limite inférieure et supérieure d'événements]

Soit $\Prob$ un espace probabilisé
et $(A_n)_{n∈ℕ}$ une suite d'événements.
\begin{enumerate}
\item
On pose $\DS B = ⋃_{n∈ℕ} \Pa{ ⋂_{p=n}^{+∞} A_p }$.

  \begin{enumerate}
  \item
Montrer que $B$ est un événement.
  \item
Décrire $B$ en langage courant.

    $B$ s'appelle la \emph{limite inférieure} de la suite d'événements $(A_n)_{n∈ℕ}$
  \end{enumerate}
\item
On pose $\DS C = ⋂_{n∈ℕ} \Pa{ ⋃_{p=n}^{+∞} A_p }$.

  \begin{enumerate}
  \item
Montrer que $C$ est un événement.
  \item
Décrire $C$ en langage courant.

    $C$ s'appelle la \emph{limite supérieure} de la suite d'événements $(A_n)_{n∈ℕ}$.
  \end{enumerate}
\item
Montrer que $B⊂C$.
\item
On rappelle que pour $X∈\T$, la fonction indicatrice de $X$ est définie par
  \[ \Fonction{\mathds{1}_X}{Ω}{ℝ}{ω}{\begin{cases}
      1 & \text{si }ω∈X  \\
      0 & \text{si }ω∉X
  \end{cases}} \]

  \begin{enumerate}
  \item
Montrer que $\mathds{1}_B = \liminf \mathds{1}_{A_n}$.
  \item
Montrer que $\mathds{1}_C = \limsup \mathds{1}_{A_n}$.
  \end{enumerate}
\end{enumerate}

\Exercice[lemme de Fatou]

Soit $\Prob$ un espace probabilisé.
Soit $(A_n)_{n∈ℕ}$ une suite d'événements.
\begin{enumerate}
\item
Montrer que $ℙ\Pa{\liminf_\ninf A_n} ≤\liminf_\ninf ℙ(A_n)$.
\item
Montrer que $ℙ\Pa{\limsup_\ninf A_n} ≥\limsup_\ninf ℙ(A_n)$.
\end{enumerate}

\Exercice[lemme de Borel-Cantelli 1]

Soit $\Prob$ un espace probabilisé.
Soit $(A_n)_{n∈ℕ}$ une suite d'événements.
On suppose que la somme des probabilités des $A_n$ est finie,
c.-à-d. \[ ∑_{n∈ℕ} ℙ(A_n) < ∞ \]
\begin{enumerate}
\item
Montrer que $ℙ(\limsup A_n) = 0$.
\item
En déduire que presque certainement,
  seul un nombre fini d'événements $A_n$ sont réalisés.
\end{enumerate}

\Exercice[lemme de Borel-Cantelli 2]

Soit $\Prob$ un espace probabilisé.
Soit $(A_n)_{n∈ℕ}$ une suite d'événements \emph{indépendants}.
On suppose que la somme des probabilités des $A_n$ est infinie,
£cad. \[ ∑_{n∈ℕ} ℙ(A_n) \text{ diverge} \]
\begin{enumerate}
\item
Soit $(a_n)$ une suite à valeurs dans $\intFO{0,1}$ tels que $∑_n u_n$ diverge.
  Montrer que $∏_{n≥0} (1 - u_n) = 0$.
\item
Montrer que $ℙ(\limsup A_n) = 1$.
\item
En déduire que presque certainement,
  une infinité d'événements $A_n$ sont réalisés.
\end{enumerate}

\Exercice

Pour $n∈ℕ^*$, on considère le jeu $J_n$ suivant:
dans une urne, il y a $n^2$ boules, dont une seule noire.
Avant de jouer, on doit miser $1$ €.
On tire une boule au hasard, et si on tire la boule noire, on gagne $n^2$ €, rien sinon.

On note $X_n$ le gain (positif ou négatif) du jeu $J_n$,
et $S_n$ le gain d'une personne qui jouerait successivement aux jeux $J_1, \dots, J_n$.
\begin{enumerate}
\item
\begin{enumerate}
\item
Déterminer $𝔼(X_n)$ pour $n∈ℕ^*$.
\item
En déduire $𝔼(S_n)$ pour $n∈ℕ^*$.
\end{enumerate}
\item
On suppose que l'on continue indéfiniment de jouer aux jeux $J_1, \dots, J_n, \dots$

  \begin{enumerate}
  \item
En utilisant le lemme de Borel-Cantelli 1, montrer que,
    presque sûrement, on ne gagnera qu'à un nombre fini de jeux.
    On pourra noter $A_n$ l'événement \og{}on gagne dans le jeu $J_n$\fg{}.
  \item
On note $B$ l'événement \og{}$S_n \to -∞$\fg{}.
    En déduire que $ℙ(B) = 1$.
  \end{enumerate}
\item
Commentaires?
\end{enumerate}

\end{document}
