\documentclass{yann}
\usepackage{dsfont}

\renewcommand{\T}{\mathscr{T}}
\newcommand{\Part}{\mathcal{P}}
\newcommand{\Pro}{\bigl(Ω,\T\bigr)}
\newcommand{\Prob}{\bigl(Ω,\T,ℙ\bigr)}
\newcommand{\SEnsemble}[2]{\{ #1 \;|\; #2 \}}
\newcommand{\LProb}{\LL1\Prob}
\newcommand{\LL}[1]{\mathcal{L}^{#1}}
\newcommand{\Cov}{\mathop{\mathrm{Cov}}}
\newcommand{\me}{e}
\newcommand{\I}{i}

\begin{document}
\title{Variables aléatoires discrètes}
\maketitle

\Para{Notations}

Dans tout le chapitre
\begin{itemize}
\item
  $\Pro$ désigne un espace probabilisable;
\item
  $\Prob$ désigne un espace probabilisé;
\item
  Toutes les variables aléatoires considérées seront des variables aléatoires \emph{discrètes}.
\end{itemize}

% -----------------------------------------------------------------------------
\section{Généralités}

\subsection{Variables aléatoires discrètes}

\Para{Définition}

Une \emph{variable aléatoire discrète} $X$ sur $\Pro$ est une fonction $\Fn{X}{Ω}{E}$ telle que
\begin{itemize}
\item
  l'image $X(Ω)$ de $X$ est au plus dénombrable;
\item
  pour tout $x∈X(Ω)$, l'ensemble $X^{-1}\bigl(\{ x \}\bigr)$ est un élément de la tribu $\T$.
\end{itemize}

\Para{Lemme}

Dans ces conditions, pour toute partie $A$ de $E$, on a $X^{-1}(A)∈\T$.
On note \og{}$X∈A$\fg{} l'événement $X^{-1}(A)$.

\Para{Définition.}

Soit $X$ une variable aléatoire sur $\Pro$.
On dit que \emph{$X$ est à valeurs dans $F$} £ssi. $X(Ω)⊂F$.

\subsection{Loi d'une variable aléatoire discrète}

\Para{Définition}

Soit $X$ une variable aléatoire sur $\Prob$ à valeurs dans $E$.
La \emph{loi} de $X$ est l'application
\[ \Fonction{ℙ_X}{\Part(E)}{[0,1]}{A}{ ℙ(X∈A) = ℙ\bigl(X^{-1}(A)\bigr) } \]
Il s'agit d'une probabilité sur l'espace probabilisable $\bigl(E,\Part(E)\bigr)$.

\Para{Proposition}

Soit $X$ une variable aléatoire sur $\Prob$ à valeurs dans $E$.
On suppose $E = \SEnsemble{x_n}{n∈ℕ}$ où les $x_n$ sont deux à deux distincts.
Alors la suite $p_n = ℙ(X = x_n)$ caractérise la loi de la suite $ℙ_X$.
Plus précisément, pour tout $A⊂E$ on a
\[ ℙ_X(A) = ∑_{\substack{n∈ℕ\\ x_n∈A}} p_n \]

\Para{Théorème d'existence}

Soit $X$ une variable aléatoire sur $\Pro$ telle que
$X(Ω) = \SEnsemble{x_n}{n∈ℕ}$ où les $x_n$ sont deux à deux distincts.
Soit $(p_n)_{n∈ℕ}$ une suite de réels positifs telle que la série $∑_n p_n$ converge et $∑_{n≥0} p_n = 1$.
Alors il existe une probabilité $ℙ$ sur $\Pro$ telle que
\[ ∀n∈ℕ\+ℙ(X=x_n) = p_n. \]

\Para{Remarque}

Il n'y a pas unicité en général.

\subsection{Fonction de répartition d'une variable aléatoire réelle}

\Para{Définition}

Soit $X$ une variable aléatoire réelle sur $\Prob$.
La \emph{fonction de répartition de $X$} est la fonction $\Fn{F_X}{ℝ}{ℝ}$ définie par
\[ F_X(x) = ℙ(X≤x) = ℙ_X(\intOF{-∞,x}). \]

\Para{Propriétés}

Soit $X$ une variable aléatoire réelle sur $\Prob$
et $F_X$ sa fonction de répartition.
Alors, pour $(a,b) ∈ℝ^2$, on a
\begin{enumerate}
\item
  $F_X$ est croissante sur $ℝ$;
\item
  $\DS \lim_{-∞} F_X = 0$;
\item
  $\DS \lim_{+∞} F_X = 1$;
\item
  Si $a < b$, alors $\DS ℙ( a < X ≤b ) = F_X(b) - F_X(a)$;
\item
  $F_X$ est \emph{continue à droite}, c.-à-d.
  $\DS\lim_{x\to a^+} F_X(x) = F_X(a)$;
\item
  $\DS \lim_{x \to a^-} F_X(x) = ℙ(X < a)$;
\item
  $\DS ℙ(X = a) = F_X(a) - \lim_{x \to a^-} F_X(x)$.
\end{enumerate}

\Para{Théorème}

Soit $X$ et $Y$ deux variables aléatoires réelles sur $\Prob$.
Alors $X$ et $Y$ ont la même fonction de répartition £ssi. $X$ et $Y$ ont même loi.
\[ F_X = F_Y \iff ℙ_X = ℙ_Y. \]

\subsection{Fonction d'une ou de plusieurs variables aléatoires}

\Para{Lemme}

Soit $\Prob$ un espace probabilisé, $\Fn{X}{Ω}{E}$ et $\Fn{Y}{Ω}{F}$ deux variables aléatoires.
On pose $Z = (X,Y)$, c.-à-d.
\[ \Fonction{Z}{Ω}{E×F}{ω}{\bigl( X(ω), Y(ω) \bigr)} \]
Alors $Z$ est une variable aléatoire discrète à valeurs dans $E×F$.

Cela se généralise immédiatement au cas de $n$ variables aléatoires.

\Para{Proposition-Définition}

Soit $X$ une variable aléatoire sur $\Pro$ à valeurs dans $E$
et $\Fn{f}{E}{F}$ une application quelconque.
On note $f(X)$ la fonction $f◦X$; il s'agit d'une variable aléatoire discrète.

\Para{Corollaire}

Soit $\Prob$ un espace probabilisé.
On considère $n$ variables aléatoires discrètes $\Fn{X_i}{Ω}{E_i}$ pour $i∈\Dcro{1,n}$.
Soit $\Fn{f}{∏_{i=1}^n E_i}{F}$ une application quelconque.
Alors $Y = f(X_1,X_2,\dots,X_n)$ est une variable aléatoire discrète.

% -----------------------------------------------------------------------------
\section{Espérance}

\subsection{Variables aléatoires discrètes d'espérances finies}

\Para{Définition}

Soit $X$ une variable aléatoire réelle sur $\Prob$.
On suppose que $X(Ω)⊂\SEnsemble{x_n}{n∈ℕ}⊂ℝ$ où les $x_n$ sont deux à deux distincts.
On dit que $X$ est \emph{d'espérance finie} £ssi. la série $∑_n x_n \, ℙ(X = x_n)$ converge absolument.
Quand c'est le cas, appelle espérance de $X$ le réel
\[ 𝔼(X) = ∑_{n=0}^{+∞} x_n \, ℙ(X=x_n) \]

On admet que cette définition ne dépend pas du choix de la suite $(x_n)_{n∈ℕ}$.

\Para{Formule de transfert}

Soit $E = \SEnsemble{x_n}{n∈ℕ}$ où les $x_n$ sont deux à deux distincts.
Soit $X$ une variable aléatoire à valeurs dans $E$
et $\Fn{f}{E}{ℝ}$ une application quelconque.
Alors la variable aléatoire $f(X)$ est d'espérance finie
£ssi. la série $∑_n f(x_n) \, ℙ(X = x_n)$ converge absolument.
Dans ce cas, on a
\[ 𝔼\bigl( f(X) \bigr) = ∑_{n=0}^{+∞} f(x_n) \, ℙ(X=x_n). \]

\Para{Proposition}[variables aléatoires égales p.s.]

Soit $X$ et $Y$ deux variables aléatoires discrètes à valeurs dans $E$.
On suppose que $X = Y$ presque sûrement, £cad. que $ℙ(X=Y) = 1$.
\begin{enumerate}
\item
  Dans le cas $E = ℝ$,
  $X$ est d'espérance finie £ssi. $Y$ l'est.
  Dans ce cas, on a $𝔼(X) = 𝔼(Y)$.
\item
  Plus généralement, si $f$ est une fonction $E \to ℝ$,
  $f(X)$ est d'espérance finie £ssi. $f(Y)$ l'est.
  Dans ce cas, on a $𝔼(f(X)) = 𝔼(f(Y))$.
\end{enumerate}

\Para{Proposition}

Soit $X$ et $Y$ deux variables aléatoires réelles sur $\Prob$.
\begin{enumerate}
\item
  $X$ est d'espérance finie £ssi. $\Abs{X}$ est également d'espérance finie.
\item
  Si $\Abs{X}≤Y$ presque sûrement, c.-à-d. si $ℙ\bigl(\Abs{X}≤Y\bigr) = 1$,
  et si $Y$ est d'espérance finie, alors $X$ est également d'espérance finie.
\item
  En particulier, si $X$ est bornée, alors $X$ est d'espérance finie.
\item
  Si $X$ et $Y$ sont d'espérances finies, et $(λ,μ)∈ℝ^2$,
  alors $λX+μY$ est également d'espérance finie.
\end{enumerate}

\Para{Proposition}

Soit $X$ et $Y$ deux variables aléatoires réelles
d'espérances finies sur $\Prob$.
On a
\begin{enumerate}
\item
  Si $λ$ et $μ$ sont des réels, alors $𝔼(λX +μY) = λ𝔼(X) + μ𝔼(Y)$.
\item
  Si $X≥0$ presque sûrement, alors $𝔼(X)≥0$.
\item
  Si $X≤Y$ presque sûrement, alors $𝔼(X)≤𝔼(Y)$.
\item
  Si $X = a$ presque sûrement, alors $𝔼(X) = a$.
\item
  Si $𝔼\bigl( \Abs{X} \bigr) = 0$, alors $X = 0$ presque sûrement.
\item
  Si $A⊂ℝ$, alors $ℙ(X∈A) = 𝔼\bigl( \mathds{1}_A(X) \bigr)$.
\item
  $\bigl| 𝔼(X) \bigr| ≤𝔼\bigl( |X| \bigr)$.
\end{enumerate}

\Para{Définition}

Une variable aléatoire réelle $X$ d'espérance finie est dite \emph{centrée} £ssi. $𝔼(X) = 0$.

\Para{Proposition}[inégalité de Markov]

Soit $\Prob$ un espace probabilisé et $X$ une variable aléatoire discrète d'espérance finie.
Si $X$ est à valeurs dans $ℝ^+$ (presque sûrement) et si $a > 0$, alors
\[ ℙ(X≥a) ≤ \frac{𝔼(X)}{a}. \]

\subsection{Moments d'une variable aléatoire réelle}

\Para{Définition}

Soit $X$ une variable aléatoire réelle sur $\Prob$.
Soit $p∈ℕ$.
On dit que \emph{$X$ admet un moment d'ordre $p$} £ssi. la variable aléatoire $X^p$ est d'espérance finie,
et on appelle \emph{moment d'ordre $p$ de $X$} le réel $𝔼(X^p)$.

On note $\LL p\Prob$ l'ensemble des variables aléatoires réelles sur $\Prob$ qui admettent un moment d'ordre $p$.
On notera $\LL p$ au lieu de $\LL p\Prob$ lorsque le contexte le permettra.

\Para{Proposition}

Pour tout $p∈ℕ$, $\LL p$ est un espace vectoriel.
De plus,
\begin{itemize}
\item
  $\LL0$ est l'ensemble des variables aléatoires réelles discrètes sur $\Pro$;
\item
  $\LL1$ est l'ensemble des variables aléatoires réelles discrètes sur $\Prob$ d'espérance finie.
\end{itemize}

\Para{Proposition}

Soit $(p,q)∈ℕ^2$ tels que $p≤q$.
Alors $\LL q ⊂\LL p$.

\Para{Corollaire}

Soit $X$ une variable aléatoire réelle. Si $X$ admet un moment d'ordre 2,
alors $X$ est d'espérance finie.

\Para{Proposition}

Si $X$ et $Y$ sont deux variables aléatoires de $\LL2$,
alors $XY∈\LL1$.
De plus, on a l'inégalité de Cauchy-Schwarz,
\[ 𝔼(XY)^2 ≤𝔼(X^2) \,𝔼(Y^2). \]

\Para{Définition}

Soit $X$ une variable aléatoire réelle d'espérance finie.
Notons $Y = \bigl(X -𝔼(X)\bigr)^2$.
Si $Y$ est d'espérance finie,
on dit que $X$ est \emph{de variance finie},
et on appelle \emph{variance de $X$} le réel $𝔼(Y)$.

\Para{Proposition}

Une variable aléatoire réelle est de variance finie £ssi. elle est dans $\LL2$.

\Para{Proposition}

Soit $X∈\LL2$.
\begin{enumerate}
\item
  \emph{Formule de König-Huygens}: $𝕍(X) = 𝔼(X^2) -𝔼(X)^2$.
\item
  Si $a∈ℝ$, alors $𝕍(aX) = a^2 𝕍(X)$ et $𝕍(X+a) = 𝕍(X)$.
\item
  On a $𝕍(X) = 0$ £ssil. existe $a∈ℝ$ tel que $X = a$ presque sûrement.
  Dans ce cas, $a = 𝔼(X)$.
\end{enumerate}

\Para{Inégalité de Bienaymé-Tchebychev}

Soit $X$ une variable aléatoire réelle de variance finie $σ^2$ et d'espérance $μ$.
Pour tout $α>0$, on a
\[ ℙ \bigPa{ \Abs{X-μ} ≥ α } ≤ \frac{σ^2}{α^2} \]

\subsection{Fonctions génératrices}

On s'intéresse ici au cas des variables aléatoires discrètes à valeurs dans $ℕ$.

\Para{Définition}

Soit $X$ une variable aléatoire réelle à valeurs dans $ℕ$.
On appelle \emph{fonction génératrice de $X$} la fonction $G_X$ définie par
$G_X(t) = 𝔼\bigl(t^X\bigr)$.

\Para{Proposition}

Avec les mêmes notations, posons $p_n = ℙ(X=n)$. On a
\[ G_X(t) = ∑_{n=0}^{+∞} p_n t^n. \]
$G_X$ est la somme d'une série entière.
De plus,
\begin{itemize}
\item
  son rayon de convergence $R_X$ vérifie $R_X≥1$;
\item
  elle converge absolument sur $\intF{-1,1}$ (au moins);
\item
  elle est continue sur $\intF{-1,1}$;
\item
  elle est de classe $\CC∞$ sur $\intO{-R_X,R_X}$, et en particulier sur $\intO{-1,1}$.
\end{itemize}

\Para{Proposition}

Soit $X$ et $Y$ deux variables aléatoires sur $\Prob$ à valeurs dans $ℕ$.
Alors $X$ et $Y$ ont la même f £ssi. $X$ et $Y$ ont la même loi.
\[ G_X = G_Y \iff ℙ_X = ℙ_Y. \]

\Para{Théorème}

Soit $X$ une variable aléatoire sur $\Prob$ à valeurs dans $ℕ$.
\begin{enumerate}
\item
  $X$ est d'espérance finie £ssi. $G_X$ est dérivable en $1$.
  Dans ce cas, $G_X'(1) = 𝔼(X)$.
\item
  $X$ est de variance finie £ssi. $G_X$ est deux fois dérivable en $1$.
  Dans ce cas, on a $G_X''(1) =𝔼\bigl[ X(X-1) \bigr]$,
  de sorte que $𝕍(X) = G_X''(1) + G_X'(1) - G_X'(1)^2$.
  Il faut savoir retrouver cette dernière formule.
\end{enumerate}

\Para{Remarque}

Si $X$ n'est pas à valeurs dans $ℕ$, la fonction génératrice $G_X$ n'est pas définie.
On travaille généralement avec la \emph{fonction caractéristique}
$φ_X(t) = 𝔼\bigl(\me^{\I t X}\bigr)$, qui est toujours définie sur $ℝ$.

% -----------------------------------------------------------------------------
\section{Couples et suites de variables aléatoires}

\subsection{Couples de variables aléatoires}

\Para{Définition}

Soit $\Prob$ un espace probabilisé,
$\Fn{X}{Ω}{E}$ et $\Fn{Y}{Ω}{F}$ deux variables aléatoires discrètes.
Notons $Z$ la variable aléatoire discrète $Z=(X,Y)$.
La loi de $Z$ s'appelle la \emph{loi conjointe} du couple $(X,Y)$,
et les lois de $X$ et de $Y$ s'appellent les \emph{lois marginales}.
Plus explicitement, notons $E = \SEnsemble{x_i}{i∈ℕ}$ et $F = \SEnsemble{y_j}{j∈ℕ}$.
\begin{itemize}
\item
  La loi conjointe du couple $(X,Y)$ est caractérisée par
  la famille $(p_{i,j})_{(i,j)∈ℕ^2}$ où $∀(i,j)∈ℕ^2$,
  \[ p_{i,j} = ℙ\bigl(Z = (x_i,y_j)\bigr) = ℙ( X = x_i, Y = y_j ) \]
\item
  La loi marginale de $X$ est caractérisée par
  la famille $(q_i)_{i∈ℕ}$ où $∀i∈ℕ$,
  \[ q_i = ℙ(X = x_i) \]
\item
  La loi marginale de $Y$ est caractérisée par
  la famille $(r_j)_{j∈ℕ}$ où $∀j∈ℕ$,
  \[ r_j = ℙ(Y = y_j) \]
\end{itemize}

\Para{Proposition}

Connaissant la loi conjointe d'un couple de variables aléatoires,
on peut déterminer les lois marginales.
En effet, avec les mêmes notations, on a
\[ ∀i∈ℕ\+ q_i = ∑_{j∈ℕ} p_{i,j} \quad \text{et}\quad ∀j∈ℕ\+ r_j = ∑_{i∈ℕ} p_{i,j}. \]

Attention toutefois, il n'est pas possible de déterminer la loi conjointe à
partir des lois marginales, sauf si l'on sait que $X$ et $Y$ sont indépendantes (voir ci-après).

\subsection{Indépendance}

\Para{Définition}

Soit $\Fn{X}{Ω}{E}$ et $\Fn{Y}{Ω}{F}$ deux variables aléatoires sur $\Prob$.
On dit que $X$ et $Y$ sont \emph{indépendantes} £ssi. $∀(x,y)∈E×F$,
\[ ℙ(X = x, Y = y) = ℙ(X = x) \, ℙ(Y = y). \]

\Para{Proposition}

Soit $\Fn{X}{Ω}{E}$ et $\Fn{Y}{Ω}{F}$ deux variables aléatoires indépendantes sur $\Prob$.
Si $A⊂E$ et $B⊂F$, alors \[ ℙ(X∈A, Y∈B) = ℙ(X∈A) \, ℙ(Y∈B). \]

\Para{Proposition}

Soit $\Prob$ un espace probabilisé, $\Fn{X}{Ω}{E}$ et $\Fn{Y}{Ω}{F}$ deux variables aléatoires discrètes. Soit $\Fn{f}{E}{E'}$ et $\Fn{g}{F}{F'}$ deux fonctions quelconques.
Si $X$ et $Y$ sont indépendantes, alors les variables aléatoires $f(X)$ et $g(Y)$ sont également indépendantes.

\Para{Proposition}

Soit $X∈\LL1$ et $Y∈\LL1$.
Si $X$ et $Y$ sont indépendantes, alors $XY∈\LL1$
et $𝔼(XY) = 𝔼(X) \,𝔼(Y)$.

\Para{Corollaire}

Soit $X$ et $Y$ deux variables aléatoires discrètes à valeurs dans $E$ et $F$ respectivement.
Soit $\Fn f Eℝ$ et $\Fn gFℝ$ deux fonctions quelconques.
Si $X$ et $Y$ sont indépendantes et si $f(X)$ et $g(Y)$ sont d'espérances finies,
alors $f(X)\, g(Y)$ est également d'espérance finie et
\[ 𝔼\bigPa{f(X)\,g(Y)} = 𝔼\bigPa{f(X)} \, 𝔼\bigPa{g(Y)}. \]

\Para{Proposition}[fonction génératrice de la somme de deux variables aléatoires indépendantes]

Soit $\Prob$ un espace probabilisé,
$X$ et $Y$ deux variables aléatoires discrètes à valeurs dans $ℕ$.
Notons $G_X$ (respectivement $G_Y$) la fonction génératrice de $X$ (resp. $Y$),
qui est absolument convergente sur $\mathcal{D}_X$ (resp. $\mathcal{D}_Y$)
et de rayon de convergence $R_X$ (resp. $R_Y$).
Si $X$ et $Y$ sont indépendantes, alors
\begin{itemize}
\item
  $\mathcal{D}_{X+Y} ⊃\mathcal{D}_X ∩\mathcal{D}_Y$,
\item
  $R_{X+Y} ≥\min(R_X, R_Y)$,
\item
  $∀t∈\mathcal{D}_X ∩\mathcal{D}_Y$, on a $G_{X+Y} (t) = G_X(t) G_Y(t)$.
\end{itemize}

\Para{Exemple}

Utiliser ce résultat pour montrer que si $X_1, \dots, X_n$ sont des variables (mutuellement) indépendantes qui suivent une loi de Bernoulli $\mathcal{B}(p)$, alors $S =∑_{k=1}^n X_k$ suit une loi binomiale $\mathcal{B}(n,p)$.

\subsection{Covariance, corrélation}

\Para{Proposition-Définitions}

Soit $X$ et $Y$ deux variables aléatoires de $\LL2$ d'espérances $μ_X$ et $μ_Y$.
Alors la variable aléatoire $(X - μ_X) (Y - μ_Y)$ est d'espérance finie.
On appelle \emph{covariance de $X$ et de $Y$} le réel
\[ \Cov(X,Y) = 𝔼\bigl[ (X-μ_X)(Y-μ_Y) \bigr]. \]
On a également
\[ \Cov(X,Y) = 𝔼(XY) - μ_X μ_Y = 𝔼(XY) - 𝔼(X)𝔼(Y). \]

\Para{Proposition}

Soit $X$, $Y$ et $Z$ trois variables aléatoires de $\LL2$
et $(a,b)∈ℝ^2$. On a
\begin{enumerate}
\item
  $\Cov(⋅,⋅)$ est une forme bilinéaire symétrique positive
  (mais non définie positive) sur $\LL2$, c.-à-d.

  \begin{enumerate}
  \item
    $\Cov(aX+bY,Z) = a\Cov(X,Z) + b\Cov(Y,Z)$;
  \item
    $\Cov(X,aY+bZ) = a\Cov(X,Y) + b\Cov(X,Z)$;
  \item
    $\Cov(X,Y) = \Cov(Y,X)$;
  \item
    $\Cov(X,X) = 𝕍(X)≥0$.
  \end{enumerate}
\item
  $𝕍(aX+bY) = a^2𝕍(X) + b^2𝕍(Y) + 2ab \Cov(X,Y)$.
\item
  Si $X$ et $Y$ sont indépendantes, $\Cov(X,Y) = 0$.
\end{enumerate}

\Para{Définition}

Soit $X$ et $Y$ deux variables aléatoires de $\LL2$ et de variances non nulles.
On appelle \emph{coefficient de corrélation linéaire de $X$ et de $Y$} le réel
\[ ρ(X,Y) = \frac{\Cov(X,Y)}{√{𝕍(X)𝕍(Y)}} \]

\Para{Proposition}

Dans ces conditions,
\begin{itemize}
\item
  $\Abs{ρ(X,Y)}≤1$
\item
  $ρ(X,Y) = ±1$ £ssil. existe $(a,b)∈ℝ^2$ tels que
  $Y = aX+b$ presque sûrement.
\end{itemize}

\subsection{Suites de variables aléatoires}

\Para{Définition}

Soit $X_1, \dots, X_n$ des variables aléatoires sur $\Prob$, avec
$X_i$ à valeurs dans $E_i$.
On dit que les variables $X_1, \dots, X_n$ sont \emph{mutuellement indépendantes}
£ssi. pour tous $(x_1, \dots, x_n) ∈∏_{i=1}^n E_i$, on a
\[ ℙ(X_1 = x_1, X_2 = x_2, \dots, X_n = x_n) = ∏_{i=1}^n ℙ(X_i = x_i). \]

\Para{Définition}

Soit $(X_n)_{n∈ℕ}$ une suite de variables aléatoires.
On dit que les variables aléatoires $(X_n)_{n∈ℕ}$ sont \emph{mutuellement indépendantes} £ssi. pour tout $n∈ℕ$, les variables aléatoires $X_0, \dots, X_n$ sont mutuellement indépendantes.

\Para{Remarques}
\begin{itemize}
\item
  \og{}Mutuellement indépendantes\fg{} entraîne \og{}deux à deux indépendantes\fg{}; \emph{la réciproque est fausse}.
\item
  En l'absence de précision, \og{}indépendantes\fg{} signifie \og{}mutuellement indépendantes\fg{}.
\end{itemize}

\Para{Théorème d'existence}

Si on se donne pour chaque $n∈ℕ$ une loi de probabilité discrète $\mathcal{P}_n$ sur $ℝ$,
alors il existe un espace probabilisé $\Prob$
et une suite de variables aléatoires réelles discrètes \emph{indépendantes} $(X_n)_{n∈ℕ}$ telles que $X_n$ suit la loi $\mathcal{P}_n$ pour tout $n$.

Plus précisément,
soit $(x_{n,k})_{(n,k)∈ℕ^2}$ et $(p_{n,k})_{(n,k)∈ℕ^2}$ deux familles de réels telles que
\begin{itemize}
\item
  pour tout $n∈ℕ$ fixé, la suite $(x_{n,k})_{k∈ℕ}$ soit injective;
\item
  pour tous $(n,k)∈ℕ^2$, $p_{n,k}≥0$;
\item
  pour tout $n∈ℕ$, la série $∑_k p_{n,k}$ converge et $∑_{k=0}^{+∞} p_{n,k} = 1$.
\end{itemize}

Pour $n∈ℕ$, on pose $E_n = \SEnsemble{x_{n,k}}{k∈ℕ}$.
Alors il existe un espace probabilisé $\Prob$
et une suite de variables aléatoires réelles discrètes $(X_n)_{n∈ℕ}$ sur $\Pro$ telles que:
\begin{itemize}
\item
  les variables aléatoires $(X_n)_{n∈ℕ}$ sont indépendantes;
\item
  pour tout $n∈ℕ$, la variable aléatoire $X_n$ est à valeurs dans $E_n$
  et sa loi est définie par
  \[ ∀k∈ℕ\+ℙ(X_n = x_{n,k}) = p_{n,k}. \]
\end{itemize}

\Para{Remarque}

En général, $Ω$ n'est pas dénombrable et $\T≠\Part(Ω)$.

\subsection{Résultats asymptotiques}

\Para{Théorème}[approximation de la loi binomiale par une loi de Poisson]

Soit $(p_n)_{n∈ℕ}$ une suite numérique à valeurs dans $[0,1]$.
Soit $(X_n)_{n∈ℕ}$ une suite de variables aléatoires réelles
de loi binomiale $\mathcal{B}(n,p_n)$.
Si $np_n \Toninfλ∈ℝ$, alors
\[ ℙ(X_n = k) \Toninf \frac{\me^{-λ} λ^k}{k!}. \]

\Para{Théorème}[loi faible des grands nombres]

Soit $(X_n)_{n≥1}$ une suite de variables aléatoires réelles
deux à deux indépendantes, de même loi et admettant un moment d'ordre 2.
Notons $μ= 𝔼(X_1)$ et
$S_n = ∑_{k=1}^n X_k$.
Alors
\[ ℙ\left( \left| \frac{S_n}{n} - μ\right| ≥ε\right) \Toninf 0. \]

\Para{Remarques}
\begin{itemize}
\item
  La quantité $\frac{S_n}{n}$ s'appelle la \emph{moyenne empirique} de $(X_n)_{n≥1}$.
\item
  Le résultat précédent reste vrai si l'on remplace l'hypothèse \og{}admettre un moment d'ordre 2\fg{}
  par l'hypothèse plus faible \og{}être d'espérance finie\fg{}.
\end{itemize}

% -----------------------------------------------------------------------------
\section{Lois usuelles}

\subsection{Lois finies}

\Para{Loi uniforme}

On dit que $X$ suit la loi uniforme sur l'ensemble fini $F$,
et on note $X↪\mathcal{U}(F)$
si $X$ est une variable aléatoire discrète à valeurs dans $F$ telle que
\[ ∀x∈F \+ ℙ(X=x) = \frac{1}{\Card(F)}. \]
Par exemple, si $F = \Dcro{1,n}$, alors:
\[ 𝔼(X) = \frac{n+1}{2}, \quad 𝕍(X) = \frac{n^2-1}{12}, \quad G_X(t) = \frac{t(1-t^n)}{n(1-t)}. \]

\Para{Loi de Bernoulli}

On dit que $X$ suit la loi de Bernoulli de paramètre $p∈[0,1]$,
et on note $X↪\mathcal{B}(p)$
si $X$ est une variable aléatoire discrète à valeurs dans $\{0,1\}$ telle que
\[ ℙ(X=0) = q, \quad ℙ(X=1) = p, \quad \text{où } q = 1-p. \]
\[ 𝔼(X) = p, \quad 𝕍(X) = pq, \quad G_X(t) = q + pt. \]

\Para{Loi binomiale}

On dit que $X$ suit la loi binomiale de paramètres $n∈ℕ$ et $p∈[0,1]$,
et on note $X↪\mathcal{B}(n,p)$
si $X$ est une variable aléatoire discrète à valeurs dans $\Dcro{0,n}$ telle que
\[ ∀k∈\Dcro{0,n} \+ ℙ(X=k) = \binom{n}{k} p^k q^{n-k},
\quad \text{où } q = 1-p. \]
\[ 𝔼(X) = np, \quad 𝕍(X) = npq, \quad G_X(t) = (q + pt)^n. \]

\subsection{Lois discrètes}

\Para{Loi géométrique}

On dit que $X$ suit la loi géométrique de paramètre $p∈\intO{0,1}$,
et on note $X↪\mathcal{G}(p)$
si $X$ est une variable aléatoire discrète à valeurs dans $ℕ^*$ telle que
\[ ∀k∈ℕ^* \+ ℙ(X=k) = q^{k-1} p,
\quad \text{où } q = 1-p. \]
\[ 𝔼(X) = \frac{1}{p}, \quad 𝕍(X) = \frac{q}{p^2}, \quad G_X(t) = \frac{pt}{1-qt}. \]

\Para{Loi de Poisson}

On dit que $X$ suit la loi de Poisson de paramètre $λ∈ℝ^+$,
et on note $X↪\mathcal{P}(λ)$
si $X$ est une variable aléatoire discrète à valeurs dans $ℕ$ telle que
\[ ∀k∈ℕ\+ ℙ(X=k) = \frac{\me^{-λ} λ^k}{k!}. \]
\[ 𝔼(X) = λ, \quad 𝕍(X) = λ, \quad G_X(t) = \me^{λ(t-1)}. \]

% -----------------------------------------------------------------------------
\section{Exercices}

\Exercice

Démontrer les formules de la partie \og{}lois usuelles\fg{}.

\Exercice[paradoxe de St Petersbourg]

Une banque vous propose de jouer au jeu suivant:
votre mise est de 100€.
Ensuite, vous lancez une pièce équilibrée jusqu'à ce que vous obteniez \og{}pile\fg{}
On note $X$ le nombre de lancés nécessaires.
Votre gain est de $G = 2^X$.
\begin{enumerate}
\item
  \begin{enumerate}
  \item
    Montrer que $X$ est à valeurs dans $ℕ^*∪\{+∞\}$.
  \item
    Calculer $ℙ(X = +∞)$.
  \item
    Reconnaître la loi de $X$.
  \end{enumerate}
\item
  Calculer $𝔼(G)$. Devriez-vous jouer à ce jeu?
\item
  En fait, si $G$ est trop élevé, la banque fera faillite.
  D'après Wikipedia, le PIB mondial pour 2013 est environ $73~\mathrm{TUS\$}$ (Wikipedia).
  On note $H = \min(G,2^{50})$ le gain corrigé.
  Déterminer l'espérance de $H$.
  Devriez-vous jouer à ce jeu?
\end{enumerate}

\Exercice

Environ $5\%$ des réservations aériennes sur une ligne donnée ne sont
pas utilisées, et c'est pourquoi une compagnie vend 100 billets pour
97 places; on parle de \og{}surbooking\fg{}.
Quelle est la probabilité pour que tous les passagers aient une place?

\Exercice[loi de Pascal]

Soit $(X_n)_{n∈ℕ^*}$ une suite de variables aléatoires indépendantes
suivant une loi de Bernoulli $\mathcal{B}(p)$ avec $p∈\intOF{0,1}$.
Soit $k∈ℕ$ fixé.
On note $T = \min \Ensemble{n∈ℕ}{X_1 + \dots + X_n = k}∈ℕ∪\acco{∞}$.
\begin{enumerate}
\item
  Décrire en français ce que représente $T$.
\item
  Montrer que $T<∞$ presque sûrement.
  On pourra introduire $S_n = X_1 + \dots + X_n$
  et utiliser la loi des grands nombres pour montrer que
  $ℙ\bigPa{\frac{S_n}{n} ≤\frac{p}{2}} \to 0$ quand $n\to+∞$.
\item
  Déterminer la loi de $T$.
  On dit que $T$ suit la loi de Pascal de paramètres $k$ et $p$.
\item
  Déterminer l'espérance, la variance de $T$
  et la fonction génératrice de $T$.
\item
  Soit $U = Y_1 + \dots + Y_k$ où $Y_i↪\mathcal{G}(p)$.
  Montrer que $T$ et $U$ ont la même loi.
\end{enumerate}

\Exercice

Soit $(X_{i,j})_{1≤i,j≤n}$ une famille de variables aléatoires indépendantes de même loi telles que $P(X_{i,j} = 1) = ℙ(X_{i,j} = -1) = \frac12$.
On note $M$ la matrice dont les coefficients sont les $(X_{i,j})$.
Quelle est l'espérance du déterminant de $M$?

\Exercice

On considère une expérience aléatoire ayant une probabilité $p$ de réussir et $1-p$ d'échouer.
On répète l'expérience de façon indépendante jusqu'à obtention de $n$ succès.
On note $X$ le nombre d'essais nécessaires à l'obtention de ces $n$ succès.
\begin{enumerate}
\item
  Reconnaître la loi de $X$ lorsque $n=1$.
\item
  Déterminer la loi de $X$ dans le cas général.
\item
  Exprimer le développement en série entière de
  $(1-t)^{-n-1}$.
\item
  Déterminer la fonction génératrice de $X$
  et en déduire l'espérance de $X$.
\end{enumerate}

\Exercice
\begin{enumerate}
\item
  Soit $X↪\mathcal{P}(λ)$.
  Montrer que l'événement \og{}$X$ est pair\fg{} est plus probable que l'événement \og{}$X$ est impair\fg{}.
\item
  Même question avec $X↪\mathcal{G}(p)$.
\end{enumerate}

\Exercice

Soit $X↪\mathcal{P}(λ)$ et $Y↪\mathcal{P}(μ)$ deux variables aléatoires indépendantes.
Quelle est la loi de $X+Y$?

\Exercice

Le jour de l'examen de fin d'année, $n$ élèves n'ont pas assez soigné la présentation de leur copie.
Le correcteur, plutôt que de s'escrimer à lire d'infâmes brouillons, décide de noter au hasard, de manière indépendante, les $n$ copies, en leur attribuant une note entière, au hasard, entre 0 et 20.
On note $X_n$ la variable aléatoire égale à la meilleure note du groupe.
\begin{enumerate}
\item
  Soit $k∈\Dcro{0,20}$.
  Calculer la probabilité que toutes les notes soient inférieures ou égales à $k$.
\item
  \begin{enumerate}
  \item
    Calculer $ℙ(X_n < k \mid X_n≤k)$.
  \item
    En déduire $ℙ( X_n = k \mid X_n≤k)$.
  \end{enumerate}
\item
  Calculer $ℙ(X_n = k)$ et déterminer sa limite lorsque $n \to∞$.
  Interpréter.
\end{enumerate}

\Exercice[lois sans viellissement]

Soit $X$ une variable aléatorie à valeurs dans $ℕ^*$,
telle que $∀n∈ℕ^*$, $ℙ(X=n) > 0$.
On suppose de plus que pour tous $(n,p)∈ℕ^2$,
\[ ℙ(X > n+p \;|\; X > p) = ℙ(X > n). \]
Montrer que $X$ suit une loi géométrique.

\Exercice

Expliquer pourquoi $99,\!9\,\%$ des gens possèdent un nombre de jambes strictement supérieur à la moyenne.

\Exercice

Soit $X↪\mathcal{G}(p)$.
On pose $Y = 1/X$. Calculer $𝔼(Y)$.

\Exercice[à connaître]

Soit $X$ une variable aléatoire à valeurs dans $ℕ$.
\begin{enumerate}
\item
  \begin{enumerate}
  \item
    Montrer que que $ℙ(X = n) = ℙ(X > n-1) - ℙ(X > n)$ pour tout $n∈ℕ^*$.
  \item
    Si $X∈\LL1$, montrer que $nℙ(X>n) \Toninf 0$.
  \item
    Si $X∈\LL1$, montrer que \[ 𝔼(X) = ∑_{n≥0} ℙ(X > n). \]
  \item
    Si $X∉\LL1$, montrer que la série $∑_n ℙ(X>n)$ diverge.
  \end{enumerate}
\item
  Si $X∈\LL2$, montrer de même que
  \[ 𝔼(X^2) = ∑_{n≥0} (2n+1) \, ℙ(X > n). \]
\end{enumerate}

\Exercice

Soit $X↪\mathcal{P}(λ)$.
Montrer que $ℙ\bigl(X≤\fracλ2\bigr)≤\frac4λ$
et $ℙ\bigl(X≥2λ\bigr)≤\frac1λ$.

\Exercice

Soit $X$ et $Y$ deux variables aléatoires de lois de Bernoulli respectives $\mathcal{B}(p)$ et $\mathcal{B}(p')$.
Montrer qu'elles sont indépendantes £ssi. $\Cov(X,Y) = 0$.

\Exercice

Soit $X$ une variable aléatoire discrète à valeurs dans $E$
et $\Fn{f}{E}{F}$ une application.
À quelle condition les variables aléatoires $X$ et $f(X)$ sont-elles indépendantes?

\Exercice

Soit $X_1, \dots, X_n$ des variables aléatoires indépendantes à valeurs dans $ℕ$.
On pose $Y = α_1 X_1 + \dots + α_n X_n$ où $\nUpletα1n ∈ℕ^n$.
Exprimer la série génératrice $G_Y$ en fonction des $G_{X_i}$.

\Exercice

On considère une succession (infine) de tirages à \og{}pile ou face\fg{}, avec une pièce donnant \og{}pile\fg{} avce une probabilité $p∈\intO{0,1}$.
On note $N$ le nombre de tirages nécessaires pour avoir le premier \og{}pile\fg{}.
On effectue alors $N$ nouveaux tirages, et l'on note $X$ le nombre de \og{}pile\fg{} obtenus pendant ces nouveaux tirages.
Montrer que $𝔼(X) = 1$.

\Exercice[loi binomiale et loi de Poisson]

Soit $(X_n)$ une suite de variables aléatoires telles que $X_n↪\mathscr{B}(n,p_n)$
avec $p_n = \frac{λ}{n}$.
\begin{enumerate}
\item
  Déterminer la limite de $𝔼(X_n)$ et de $𝕍(X_n)$.
\item
  Déterminer la limite de $ℙ(X_n = k)$ quand $n\to+∞$.
\item
  Comment interpréter ces résultats?
\end{enumerate}

\Exercice

Dans une grande ville, il y a en moyenne un suicide par jour.
Combien y a-t-il, en moyenne, de jours de l'année où au moins $5$ personnes
se suicident?

\Exercice

Tous les soirs, au lieu de réviser ses cours de prépa, Kevin va rusher Stratholme
dans l'espoir de looter la monture du Baron Vaillefendre.
\begin{enumerate}
\item
  Étant donné que WoW est un programme informatique (considéré comme constant),
  par quelle loi modéliseriez-vous cette expérience?
\item
  Il semblerait que la probabilité de dropper la monture soit de $0,7\%$.
  Au bout de combien de jours Kevin peut-il espérer frimer sur le destrier de la mort?
  Au fait, d'où peut provenir ce chiffre de $0,7\%$?
\item
  Quelle est la probabilité d'arriver à la dropper en moins de 210 jours
  (entre le début et la fin des cours)?
  On pourra utiliser la calculatrice, ou l'approximation par une loi de Poisson.
\item
  Question subsidiaire: que pensez-vous des chances de réussite aux concours de Kevin?
\end{enumerate}

\Exercice[moindres carrés]
\begin{enumerate}
\item
  Soit $X∈\LL2$.
  Quelle valeur de $a∈ℝ$ minimise-t-elle la quantité
  $𝔼\bigl[ (X-a)^2 \bigr]$?
\item
  Soit $X$ et $Y$ deux variables aléatoires de $\LL2$.
  Déterminer $(a,b)∈ℝ^2$ qui minimise
  $𝔼\bigl[ (Y - aX-b)^2 \bigr]$.
\end{enumerate}

\Exercice

Soit $X$ et $Y$ deux variables aléatoires à valeurs dans $ℕ$ de loi conjointe
\[ ∀(i,j)∈ℕ^2 \+ ℙ(X = i, Y = j) = \frac{a}{i!j!}. \]
\begin{enumerate}
\item
  Déterminer $a$.
\item
  Reconnaître les lois marginales de $X$ et de $Y$.
\item
  $X$ et $Y$ sont-elles indépendantes?
\end{enumerate}

\Exercice[inégalité de Jensen]

Soit $I$ un intervalle de $ℝ$, $\Fn{f}{I}{ℝ}$ une fonction convexe et
$X$ une variable aléatoire réelle à valeurs dans $I$ d'espérance finie.
L'inégalité de Jensen affirme alors que, si $f(X)$ est d'espérance finie, on a
\[ 𝔼\bigl(f(X)\bigr)≥f\bigl(𝔼(X) \bigr). \]
\begin{enumerate}
\item
  On suppose $f$ dérivable sur $I$.

  \begin{enumerate}
  \item
    Soit $μ∈I$. Montrer qu'il existe $(a,b)∈ℝ^2$ tels que
    $∀x∈I$, $f(x)≥ax+b$ et $f(μ) = aμ+ b$.
  \item
    Montrer que $𝔼(X)∈I$.
  \item
    Conclure.
  \end{enumerate}
\item
  (dur) Montrer que le résultat de la question 1a. reste vrai
  si l'on ne suppose plus $f$ dérivable. Conclure.
\end{enumerate}

\Exercice[identité de Wald]

Soit $N$ et $(X_n)_{n≥1}$ des variables aléatoires indépendantes à valeurs dans $ℕ$ et d'espérances finies.
On suppose que les $(X_n)_{n≥1}$ ont toutes la même loi.
On pose \[ S = ∑_{k=1}^N X_k. \]
\begin{enumerate}
\item
  Montrer que $G_S(t) = G_N\bigl( G_{X_1}(t) \bigr)$ pour $\Abs{t}<1$.
  On admettra que l'on peut permuter les sommes.
\item
  Établir l'\emph{identité de Wald}:
  \[ 𝔼(S) = 𝔼(N) \, 𝔼(X_1). \]
\end{enumerate}

\end{document}
