\documentclass{yann}

\newcommand{\me}{e}
\newcommand{\eq}[1]{\mathrm{(#1)}}
\newcommand{\mtag}[1]{\tag{$\mathrm{#1}$}}
\newcommand{\solI}[1]{\mathcal{S}_I(#1)}
\newcommand{\solJ}[1]{\mathcal{S}_J(#1)}
\newcommand{\VnK}{\mathrm{M}_{n,1}(𝕂)}

\begin{document}
\title{Équations différentielles}
\maketitle

Ce chapitre reprend rapidement et étend l'étude des équations différentielles linéaires vues en première année.

\Para{Notations}

Dans ce chapitre,
\begin{itemize}
\item $𝕂$ désigne le corps $ℝ$ ou $ℂ$,
\item $I$ désigne un intervalle (non vide et non réduit à un point) de $ℝ$,
\item $J$ désigne un intervalle (non vide et non réduit à un point) \emph{inclus dans $I$},
\item toutes les fonctions qui interviennent sont (au moins) continues,
\item $t∈I$ est la variable par rapport à laquelle on dérive,
\item dans le cas scalaire, la fonction inconnue est $x \colon t \mapsto x(t)∈𝕂$,
\item dans le cas vectoriel,
  \begin{itemize}
  \item on identifie $\VnK$ et $𝕂^n$,
  \item la fonction inconnue est $X \colon t \mapsto X(t)∈𝕂^n$.
  \end{itemize}
\end{itemize}

% -----------------------------------------------------------------------------
\section{Équations différentielles linéaires scalaires du premier ordre}

\subsection{Généralités}

\Para{Définitions}

Une \emph{équation différentielle linéaire scalaire du premier ordre}
est une équation de la forme
\[\mtag{E} a(t) x'(t) + b(t) x(t) = c(t),\]
où $a$, $b$ et $c$ sont des fonctions \emph{continues} de $I$ dans $𝕂$,
et $x$ une fonction inconnue de $I$ dans $𝕂$.

Une \emph{solution sur $J⊂I$} de cette équation est une fonction $f$ de $J$ dans $𝕂$,
dérivable en tout point de $J$, telle que pour tout $t∈J$, on ait
\[a(t) f'(t) + b(t) f(t) = c(t).\]
On note $\solJ{E}$ l'ensemble des solutions de $\eq{E}$ sur $J$.

\emph{Résoudre} $\eq{E}$, c'est déterminer $\solJ{E}$
pour tout intervalle $J⊂I$.

\Para{Définition}

Avec les mêmes notations, dans le cas $𝕂=ℝ$, on considère l'équation différentielle
\[\mtag{E} a(t) x'(t) + b(t) x(t) = c(t).\]
Les courbes représentatrices des solutions de $\eq{E}$ s'appellent \emph{courbes intégrales} de $\eq{E}$.

\Para{Définitions}

On considère l'équation différentielle

\[\mtag{E} a(t) x'(t) + b(t) x(t) = c(t).\]
\begin{itemize}
\item $\eq{E}$ est dite \emph{à coefficients constants}
  si et seulement si les fonctions $a$ et $b$ sont constantes.
\item $\eq{E}$ est dite \emph{normalisée} ou \emph{résolue en $x'$}
  si et seulement si \[∀t∈I\+ a(t) = 1.\]
\item $\eq{E}$ est dite \emph{homogène} ou \emph{sans second membre}
  si et seulement si \[∀t∈I\+ c(t) = 0.\]
\item On appelle \emph{équation différentielle homogène associée} à $\eq{E}$
  ou \emph{équation différentielle sans second membre associée} à $\eq{E}$
  l'équation différentielle
  \[\mtag{E_0} a(t) x'(t) + b(t) x(t) = 0.\]
\end{itemize}

\Para{Définition}

Un \emph{problème de Cauchy} du premier ordre
est la donnée d'une équation différentielle du premier ordre
normalisée et d'une condition initiale.

Un problème de Cauchy linéaire du premier ordre est donc de la forme
\[\mtag{C} \begin{cases}
  ∀t∈I\+ x'(t) + b(t)x(t) = c(t), \\
  x(t_0) = x_0 \text{ où } t_0∈I \text{ et } x_0∈𝕂.
\end{cases}\]

\subsection{Étude théorique}

\Para{Remarque}

Étant donné l'équation différentielle linéaire scalaire du premier ordre
\[\mtag{E} a(t) x'(t) + b(t) x(t) = c(t),\]
on peut se ramener à une équation normalisée:
\begin{enumerate}
\item si $a$ ne s'annule pas sur $I$, il suffit de diviser par $a(t)$;
\item par contre, si $a$ s'annule, on découpe $I$ en intervalles
  où $a$ ne s'annule pas, puis on fait une étude sur chaque intervalle.
  À la fin, on cherche à \og{}raccorder\fg{} les solutions.
\end{enumerate}

\Para{Proposition}

Soit $f$ une solution sur $J$ de l'équation différentielle normalisée
\[\mtag{E} x'(t) + b(t) x(t) = c(t).\]
Alors $f$ est de classe $\CC1$ sur $J$.

De plus, si $b$ et $c$ sont de classe $\CC p$ sur $J$,
alors $f$ est de classe $\CC{p+1}$ sur $J$;
si $b$ et $c$ sont de classe $\CC∞$ sur $J$, alors $f$ l'est également.

\Para{Théorème}[Cauchy-Lipschitz linéaire]

Soit $\Fn bI𝕂$ et $\Fn cI𝕂$ deux fonctions \emph{continues}.
Soit $t_0∈I$ et $x_0∈𝕂$.
Alors le problème de Cauchy
\[\mtag{C}
\begin{cases}
  ∀t∈I\+ x'(t) + b(t) x(t) = c(t), \\
  x(t_0) = x_0
\end{cases}\]
admet une unique solution sur tout intervalle $J$
tel que $t_0∈J$ et $J⊂I$.

De plus, la solution sur $J$ n'est autre que la restriction à $J$
de la solution sur $I$.

\Para{Corollaire}

Soit $\eq{E}$ une équation différentielle linéaire scalaire normalisée
du premier ordre sur $I$
\[\mtag{E} x'(t) + b(t)x(t) = c(t).\]
Alors les courbes intégrales de $\eq{E}$ sont disjointes.
Plus précisément, elles forment une partition de $I×ℝ$.

\Para{Proposition}

Soit $f$ une solution sur $J$ de l'équation différentielle linéaire scalaire
normalisée homogène du premier ordre
\[\mtag{E_0} x'(t) + b(t) x(t) = 0.\]
Alors
\begin{itemize}
\item soit $f$ est identiquement nulle sur $J$,
\item soit $f$ ne s'annule pas sur $J$.
\end{itemize}

\Para{Théorème}[de structure]

On considère les équations différentielles linéaires scalaires
normalisées du premier ordre
\[\mtag{E}   x'(t) + b(t)x(t) = c(t)\]
\[\mtag{E_0} x'(t) + b(t)x(t) = 0\]
\begin{itemize}
\item $\solJ{E_0}$ est un sous-espace vectoriel de $\CC1(I,𝕂)$ de dimension $1$.
\item $\solJ{E}$ est un sous-espace \emph{affine} de $\CC1(I,𝕂)$ de dimension $1$ et de direction $\solJ{E_0}$.
\end{itemize}

\subsection{Résolution pratique}

\subsubsection{Principe de superposition}

\Para{Corollaire}

Étant donné une équation linéaire $\eq{E}$,
la solution générale de $\eq{E}$ est donnée
par la somme de la solution générale de l'équation homogène $\eq{E_0}$
et d'une solution particulière de $\eq{E}$.

Plus précisément, soit $x_p$ une solution particulière de $\eq{E}$ sur $J$.
Alors toute solution $x$ de $\eq{E}$ sur $J$ est de la forme
\[x = x_0 + x_p,\]
où $x_0$ est une solution de $\eq{E_0}$ sur $J$.

\emph{Pour résoudre $\eq{E}$, il suffit donc de résoudre $\eq{E_0}$ et de trouver une solution particulière de $\eq{E}$.}

\Para{Remarque}

Si $\eq{E}$ est de la forme
\[\mtag{E} a(t) x'(t) + b(t) x(t) = c_1(t) + c_2(t),\]
pour trouver une solution particulière de $\eq{E}$, il suffit de
\begin{itemize}
\item trouver une solution particulière $x_{p,1}$ de
  \[\mtag{E_1} a(t) x'(t) + b(t) x(t) = c_1(t),\]
\item trouver une solution particulière $x_{p,2}$ de
  \[\mtag{E_2} a(t) x'(t) + b(t) x(t) = c_2(t),\]
\item poser $x_p = x_{p,1} + x_{p,2}$.
\end{itemize}

Dans ces conditions, $x_p$ est une solution particulière de $\eq{E}$.

\subsubsection{Résolution de l'équation homogène}

\Para{Proposition}

On considère l'équation différentielle
\[\mtag{E_0} x'(t) + a(t) x(t) = 0,\]
où $\Fn aI𝕂$ est une fonction continue.

On note $A$ une primitive de $a$ sur $I$.
Alors, pour $J⊂I$,
\[\solJ{E_0} = \Bigl\{ \Fn fJ𝕂\,\Bigm|\, ∃K∈𝕂\+ ∀t∈J \+ f(t) = K \me^{-A(t)} \Bigr\}.\]
Moins formellement, on dit que la \emph{solution générale} de $(E_0)$
sur $J$ est donnée par
\[x(t) = K \me^{-A(t)} \quad \text{où} \quad K∈𝕂.\]

\Para{Remarque}

La \og{}résolution\fg{} suivante peut permettre de retrouver la formule
\begin{enumerate}
\item $x'(t) + a(t) x(t) = 0$
\item $\frac{x'(t)}{x(t)} = -a(t)$
\item $\ln\Abs{x(t)} = -A(t) + C_0$ où $C_0∈ℝ$
\item $\Abs{x(t)} = \me^{C_0} \me^{-A(t)} = C_1 \me^{-A(t)}$ où $C_1∈\Rp$
\item $x(t) = K \me^{-A(t)}$ où $K∈ℝ$
\end{enumerate}

Néanmoins, \emph{il ne s'agit pas d'une preuve} à cause
\begin{itemize}
\item de la division par $x(t)$ qui pourrait s'annuler, et
\item du passage de $\Abs{x(t)}$ à $x(t)$ nécessite des justifications
  (on pourrait imaginer que $x$ change de signe et donc que $K$ soit une fonction de $t$...)
\end{itemize}

Bref, à éviter sur une copie, sauf si l'on sait déjà que $∀x∈I$, $x(t)>0$.

\subsubsection{Recherche d'une solution particulière}

\Para{Proposition}[cas d'une équation à coefficients constants]

On considère l'équation différentielle
\[\mtag{E} x'(t) + ax(t) = P(t) \me^{αt},\]
où $P∈𝕂_d[X]$.

Il existe une unique solution particulière de la forme
\begin{enumerate}
\item Si $α≠-a$, on prend $x_p(t) = Q(t) \me^{αt}$
  où $Q∈𝕂_d[X]$ à déterminer.
\item Si $α= -a$, on prend $x_p(t) = tQ(t) \me^{αt}$
  où $Q∈𝕂_d[X]$ à déterminer.
\end{enumerate}

Bien sûr, si le second membre est une somme de termes de ce type,
il suffit de superposer les solutions particulières correspondantes.

\Para{Exemples}
\begin{enumerate}
\item Pour $x' + x = t\me^t$, on prend $x_p = (at+b)\me^t$.
\item Pour $x' - x = (t+1)\me^t$, on prend $x_p = t(at+b)\me^t$.
\item Pour $x' + x = \sin t$, on prend $x_p = a\cos t + b\sin t$.
\item Pour $x' + 3x = t\cos 2t$, on prend $x_p = (at+b)\cos 2t + (ct+d)\sin 2t$.
\item Pour $x' + x = t\me^t - \me^{2t}$, on prend $x_p = (at+b)\me^t + c\me^{2t}$.
\end{enumerate}

\Para{Proposition}[méthode de variation de la constante]

On considère les équations différentielles
\[\mtag{E}   x'(t) + a(t) x(t) = b(t),\]
\[\mtag{E_0} x'(t) + a(t) x(t) = 0.\]
Soit $φ$ une solution non nulle de $\eq{E_0}$ sur $J$.
On rappelle que $φ$ ne s'annule pas et que
la solution générale de $\eq{E_0}$ est donnée par
\[x_0(t) = Kφ(t) \quad \text{où } K∈𝕂.\]

On recherche alors les solutions de $\eq{E}$ sous la forme
$x(t) =λ(t)φ(t)$, et l'on obtient
\[\mtag{E'}λ'(t)φ(t) = b(t),\]
qui est facile à résoudre.

\Para{Corollaire}

La solution du problème de Cauchy
\[\begin{cases}
∀t∈I\+ x'(t) + a(t)x(t) = b(t), \\
x(t_0) = x_0
\end{cases}\]
est donnée par
\[x(t) = \me^{-A(t)} \left( \me^{A(t_0)} x_0 + ∫_{t_0}^{t} \me^{A(u)}b(u) \D u \right),\]
où $A$ est une primitive de $a$ sur $I$.

Cette formule \emph{n'est pas à connaître}, mais à savoir retrouver au besoin.

\subsubsection{Problèmes de raccords}

\Para{Remarque}

On considère l'équation différentielle sur $I$
\[\mtag{E} a(t)x'(t) + b(t)x(t) = c(t).\]
Si $a$ s'annule sur $I$, on est face à un problème de raccord.

Supposons par exemple que $I=ℝ$ et que $a$ s'annule uniquement en $1$.
\begin{itemize}
\item On pose alors $I_1 = \intO{-∞,1}$ et $I_2 = \intO{1,+∞}$.
  Sur ces deux intervalles, $\eq{E}$ est une équation normalisée, donc
  on peut utiliser les résultats précédents.
\item On résoud $\eq{E}$ sur $I_1$ et sur $I_2$.
\item On détermine, généralement par la méthode d'analyse-synthèse,
  quelles sont les fonctions $f$
  solutions de $\eq{E}$ sur $ℝ$, sachant que la restriction de $f$ à $I_k$
  est nécessairement une solution de $\eq{E}$ sur $I_k$.
\end{itemize}

\Para{Exemples}
\begin{enumerate}
\item Résoudre sur $ℝ$ l'équation $tx'(t) - 2x(t) = 0$.
\item Résoudre sur $ℝ$ l'équation $2tx'(t) - x(t) = 0$.
\end{enumerate}

% -----------------------------------------------------------------------------
\section{Systèmes différentiels linéaires}

\Para{Exemple}

On cherche à résoudre le système différentiel
\[\mtag{S} \left\{ \begin{aligned}
  x'(t) &= a(t) x(t) + b(t) y(t) + c(t) z(t) + j(t) \\
  y'(t) &= d(t) x(t) + e(t) y(t) + f(t) z(t) + k(t) \\
  z'(t) &= g(t) x(t) + h(t) y(t) + i(t) z(t) + l(t)
\end{aligned} \right.\]
Ce système peut s'écrire
\[\mtag{S} X'(t) = A(t) X(t) + B(t)\]
où $X(t) = \Matrix{x(t); y(t); z(t)}$,
$A(t) = \Matrix{a(t),b(t),c(t); d(t),e(t),f(t); g(t),h(t),i(t)}$, \\
et $B(t) = \Matrix{j(t); k(t); l(t)}$.

\subsection{Généralités}

\Para{Définitions}

Un \emph{système différentiel linéaire du premier ordre}
est une équation de la forme
\[\mtag{S} X'(t) = A(t) X(t) + B(t)\]
où $\Fn{A}{I}{\MnK}$ et $\Fn{B}{I}{𝕂^n}$ sont des fonctions continues,
et $\Fn{X}{I}{𝕂^n}$ est une fonction inconnue.

Une \emph{solution sur $J⊂I$} de ce système est une fonction $f$ de $J$ dans $𝕂^n$,
dérivable en tout point de $J$, telle que pour tout $t∈J$, on ait
\[f'(t) = A(t) f(t) + B(t).\]
On note $\solJ{S}$ l'ensemble des solutions de $\eq{S}$ sur $J$.

\emph{Résoudre} $\eq{S}$, c'est déterminer $\solJ{S}$
pour tout intervalle $J⊂I$.

\Para{Définitions}

On considère le système différentiel linéaire du premier ordre
\[\mtag{S} X'(t) = A(t) X(t) + B(t)\]
\begin{itemize}
\item $\eq{S}$ est dit \emph{à coefficients constants}
  si et seulement si la fonction $A$ est constante.
\item $\eq{S}$ est dit \emph{homogène} ou \emph{sans second membre}
  si et seulement si \[∀t∈I \+ B(t) = 0.\]
\item On appelle \emph{système différentiel homogène associe} à $\eq{S}$
  le système différentiel
  \[\mtag{S_0} X'(t) = A(t) X(t).\]
\end{itemize}

\Para{Définition}

Un \emph{problème de Cauchy} du premier ordre
est la donnée d'un système différentiel linéaire du premier ordre
et d'une condition initiale.

Un problème de Cauchy linéaire du premier ordre est donc de la forme
\[\mtag{C} \begin{cases}
  ∀t∈I\+ X'(t) = A(t)X(t) + B(t), \\
  X(t_0) = X_0 \text{ où $t_0 ∈I$ et $X_0∈𝕂$.}
\end{cases}\]

\subsection{Étude théorique}

\Para{Proposition}

Soit $f$ une solution sur $J$ du système différentiel
\[\mtag{S} X'(t) = A(t) X(t) + B(t).\]
Alors $f$ est de classe $\CC1$ sur $J$.

De plus, si $B$ et $C$ sont de classe $\CC p$ sur $J$,
alors $f$ est de classe $\CC{p+1}$ sur $J$;
si $b$ et $c$ sont de classe $\CC∞$ sur $J$, alors $f$ l'est également.

\Para{Théorème}[Cauchy-Lipschitz linéaire]

Soit $\Fn{A}{I}{\MnK}$ et $\Fn{B}{I}{𝕂^n}$ deux fonctions \emph{continues}.
Soit $t_0∈I$ et $X_0∈𝕂^n$.
Alors le problème de Cauchy
\[\mtag{C} \begin{cases}
  ∀t∈I \+ X'(t) = A(t) X(t) + B(t), \\
  X(t_0) = X_0
\end{cases}\]
admet une unique solution sur tout intervalle $J$ tel que $t_0∈J$ et $J⊂I$.

De plus, la solution sur $J$ n'est autre que la restriction à $J$
de la solution sur $I$.

\Para{Proposition}

Soit $t_0∈J$.
On considère le système différentiel homogène
\[\mtag{S_0} X'(t) = A(t)X(t)\]
Alors l'application
\[\Fonction{Φ}{\solJ{S_0}}{𝕂^n}{X}{X(t_0)}\]
est un isomorphisme (c.-à-d. une application linéaire bijective).

\Para{Théorème}[de structure]

On considère les systèmes différentiels linéaires
\[\mtag{S}   X'(t) = A(t)X(t) + B(t)\]
\[\mtag{S_0} X'(t) = A(t)X(t)\]
Alors:
\begin{itemize}
\item $\solJ{S_0}$ est un sous-espace vectoriel de $\CC1(I,𝕂^n)$ de dimension $n$.
\item $\solJ{S}$ est un sous-espace \emph{affine} de $\CC1(I,𝕂^n)$ de dimension $n$ et de direction $\solJ{S_0}$.
\end{itemize}

\subsection{Résolution pratique}

\subsubsection{Principe de superposition}

\Para{Corollaire}

Étant donné un système différentiel linéaire $\eq{S}$,
la solution générale de $\eq{S}$ est donnée
par la somme de la solution générale du système homogène associé $\eq{S_0}$
et d'une solution particulière de $\eq{S}$.

Plus précisément, soit $X_p$ une solution particulière de $\eq{S}$ sur $J$.
Alors toute solution $X$ de $\eq{S}$ sur $J$ est de la forme
\[X = X_0 + X_p,\]
où $X_0$ est une solution de $\eq{S_0}$ sur $J$.

\emph{Pour résoudre $\eq{S}$, il suffit donc de résoudre $\eq{S_0}$ et de trouver une solution particulière de $\eq{S}$.}

\Para{Remarque}

Si $\eq{S}$ est de la forme
\[\mtag{S} X'(t) = A(t) X(t) + B_1(t) + B_2(t),\]
pour trouver une solution particulière de $\eq{S}$, il suffit de
\begin{itemize}
\item trouver une solution particulière $X_{p,1}$ de
  \[\mtag{S_1} X'(t) = A(t) X(t) + B_1(t),\]
\item trouver une solution particulière $X_{p,2}$ de
  \[\mtag{S_2} X'(t) = A(t) X(t) + B_2(t),\]
\item poser $X_p = X_{p,1} + X_{p,2}$.
\end{itemize}

Dans ces conditions, $X_p$ est une solution particulière de $\eq{S}$.

\subsubsection{Résolution d'un système différentiel homogène à coefficients constants}

\Para{Théorème}

Soit $A∈\MnK$ et $\Fn{X}{ℝ}{𝕂^n}$ une solution de
\[\mtag{S} X'(t) = A X(t).\]
On suppose que $A$ est diagonalisable:
soit $\nUplet X1n$ une base de vecteurs propres
associés aux valeurs propres $\nUpletλ1n$.
Alors il existe $\nUplet a1n∈𝕂^n$ tels que
\[∀t∈ℝ\+ X(t) = ∑_{k=1}^n a_k \me^{λ_k t} X_k.\]

\Para{Remarque}

Si on ne peut pas diagonaliser $A$, mais seulement la trigonaliser,
$A = P T P^{-1}$, on pose $X(t) = P Y(t)$, et le système devient
\[\mtag{S'} Y'(t) = T Y(t),\]
ce qui donne un système triangulaire que l'on peut résoudre,
en résolvant successivement les équations différentielles scalaires.

% -----------------------------------------------------------------------------
\section{Équations différentielles linéaires scalaires du second ordre}

\subsection{Généralités}

\Para{Définitions}

Une \emph{équation différentielle linéaire scalaire du second ordre}
est une équation de la forme
\[\mtag{E} a(t) x''(t) + b(t) x'(t) + c(t) x(t) = d(t),\]
où $a$, $b$, $c$ et $d$ sont des fonctions \emph{continues} de $I$ dans $𝕂$,
et $x$ une fonction inconnue de $I$ dans $𝕂$.

Une \emph{solution sur $J⊂I$} de cette équation est une fonction $f$ de $J$ dans $𝕂$,
deux fois dérivable en tout point de $J$, telle que pour tout $t∈J$, on ait
\[a(t) f''(t) + b(t) f'(t) + c(t) f(t) = d(t).\]
On note $\solJ{E}$ l'ensemble des solutions de $\eq{E}$ sur $J$.

\emph{Résoudre} $\eq{E}$, c'est déterminer $\solJ{E}$ pour tout intervalle $J⊂I$.

\Para{Définition}

Avec les mêmes notations, dans le cas $𝕂=ℝ$, on considère l'équation différentielle
\[\mtag{E} a(t) x''(t) + b(t) x'(t) + c(t) x(t) = d(t).\]
Les courbes représentatrices des solutions de $\eq{E}$ s'appellent
\emph{courbes intégrales} de $\eq{E}$.

\Para{Définitions}

On considère l'équation différentielle
\[\mtag{E} a(t) x''(t) + b(t) x'(t) + c(t) x(t) = d(t).\]
\begin{itemize}
\item $\eq{E}$ est dite \emph{à coefficients constants}
  si et seulement si les fonctions $a$, $b$ et $c$ sont constantes.
\item $\eq{E}$ est dite \emph{normalisée} ou \emph{résolue en $x''$}
  si et seulement si \[∀t∈I\+ a(t) = 1.\]
\item $\eq{E}$ est dite \emph{homogène} ou \emph{sans second membre}
  si et seulement si \[∀t∈I\+ d(t) = 0.\]
\item On appelle \emph{équation différentielle homogène associée} à $\eq{E}$
  ou \emph{équation différentielle sans second membre associée} à $\eq{E}$
  l'équation différentielle
  \[\mtag{E_0} a(t) x''(t) + b(t) x'(t) + c(t) x(t) = 0.\]
\end{itemize}

\Para{Définition}

Un \emph{problème de Cauchy} du second ordre
est la donnée d'une équation différentielle du second ordre
normalisée et de deux conditions initiales.

Un problème de Cauchy linéaire du second ordre est donc de la forme
\[\mtag{C}
\begin{cases}
  ∀t∈I\+ x''(t) + b(t)x'(t) + c(t) x(t) = d(t), \\
  x(t_0) =α, \\
  x'(t_0) =β\text{ où } (t_0,α,β)∈I×𝕂^2.
\end{cases}\]

\subsection{Étude théorique}

\Para{Remarque}

Étant donné l'équation différentielle linéaire scalaire du premier ordre
\[\mtag{E} a(t) x''(t) + b(t) x'(t) + c(t) x(t) = d(t),\]
on peut se ramener à une équation normalisée:
\begin{enumerate}
\item si $a$ ne s'annule pas sur $I$, il suffit de diviser par $a(t)$;
\item par contre, si $a$ s'annule, on découpe $I$ en intervalles
  où $a$ ne s'annule pas, puis on fait une étude sur chaque intervalle.
  À la fin, on peut chercher à \og{}raccorder\fg{} les solutions.
\end{enumerate}

\Para{Remarque}

On considère l'équation différentielle suivante:
\[\mtag{E} x''(t) + b(t) x'(t) + c(t) x(t) = d(t).\]
On peut se ramener à un système différentiel
\[\mtag{S} X'(t) = A(t) X(t) + B(t)\]
en posant
$X(t) = \begin{pmatrix} x(t) \\ x'(t) \end{pmatrix}$,
$A(t) = \begin{pmatrix} 0 & 1 \\ -c(t) & -b(t) \end{pmatrix}$,
$B(t) = \begin{pmatrix} 0 \\ d(t) \end{pmatrix}$

\Para{Proposition}

Soit $f$ une solution sur $J$ de l'équation différentielle normalisée
\[\mtag{E} x''(t) + b(t) x'(t) + c(t) x(t) = d(t).\]
Alors $f$ est de classe $\CC2$ sur $J$.

De plus, si $b$, $c$ et $d$ sont de classe $\CC p$ sur $J$,
alors $f$ est de classe $\CC{p+2}$ sur $J$;
si $b$, $c$ et $d$ sont de classe $\CC∞$ sur $J$, alors $f$ l'est également.

\Para{Remarque}

Toute équation différentielle linéaire scalaire du second ordre sous forme résolue
\[\mtag{E} x''(t) + b(t) x'(t) + c(t) x(t) = d(t)\]
peut s'écrire sous la forme d'un système différentiel
\[\mtag{S} X'(t) = A(t) X(t) + B(t)\] où
$X(t) = \begin{pmatrix} x(t) \\ x'(t) \end{pmatrix}$,
$A(t) = \begin{pmatrix} 0 & 1 \\ -c(t) & -b(t) \end{pmatrix}$, \\
$B(t) = \begin{pmatrix} 0 \\ d(t) \end{pmatrix}$.

\Para{Théorème}[Cauchy-Lipschitz linéaire]

Soit $\Fn bI𝕂$, $\Fn cI𝕂$ et $\Fn dI𝕂$
trois fonctions \emph{continues}.
Soit $t_0∈I$ et $(α,β)∈𝕂^2$.

Alors le problème de Cauchy
\[\mtag{C}
\begin{cases}
  ∀t∈I\+ x''(t) + b(t) x'(t) + c(t) x(t) = d(t), \\
  x(t_0) =α, \\
  x'(t_0) =β
\end{cases}\]
admet une unique solution sur tout intervalle $J$
tel que $t_0∈J$ et $J⊂I$.

De plus, la solution sur $J$ n'est autre que la restriction à $J$
de la solution sur $I$.

\Para{Remarque}

Les courbes intégrales de $\eq{E}$ ne sont pas disjointes.

\Para{Théorème}[de structure]

On considère les équations différentielles linéaires scalaires normalisées du second ordre
\[\mtag{E} x''(t) + b(t)x'(t) + c(t) x(t) = d(t)\]
\[\mtag{E_0} x''(t) + b(t)x'(t) + c(t) x(t) = 0\]
\begin{itemize}
\item $\solJ{E_0}$ est un sous-espace vectoriel de $\CC2(I,𝕂)$ de dimension 2.
\item $\solJ{E}$ est un sous-espace \emph{affine} de $\CC2(I,𝕂)$ de dimension 2 et de direction $\solJ{E_0}$.
\end{itemize}

\Para{Définition}

On appelle \emph{système fondamental} des solutions de l'équation homogène
\[\mtag{E_0} x''(t) + b(t)x'(t) + c(t) x(t) = 0\]
toute base $(φ,ψ)$ de $\solI{E_0}$.

\Para{Proposition}

Soit $(φ,ψ)$ un système fondamental de $\eq{E_0}$.
Alors
\[\begin{split} \solJ{E_0} &= \Bigl\{ \Fn fJ𝕂\;\Bigm|\; ∃(A,B)∈𝕂\, \\
& ∀t∈J\+, f(t) = Aφ(t) + Bψ(t) \Bigr\}. \end{split}\]

Autrement dit, la solution générale de $\eq{E_0}$ est
\[x_0(t) = Aφ(t) + Bψ(t)\]
où $(A,B)∈𝕂^2$.

\Para{Définition}

Avec les mêmes notations,
soit $φ$ et $ψ$ deux solutions de $\eq{E_0}$.
On appelle \emph{Wronskien} de $φ$ et $ψ$ la quantité
\[W(t) = \begin{vmatrix} φ(t) & ψ(t) \\ φ'(t) & ψ'(t) \end{vmatrix} =φ(t)ψ'(t) -φ'(t)ψ(t).\]

\Para{Remarque}

$W$ est solution de l'équation différentielle
\[W'(t) + b(t)W(t) = 0.\]

\Para{Proposition}

Les conditions suivantes sont équivalentes:
\begin{enumerate}
\item $(φ,ψ)$ est un système fondamental pour $\eq{E_0}$,
\item $(φ,ψ)$ est une famille libre,
\item $W$ ne s'annule pas sur $I$,
\item $W$ n'est pas identiquement nul sur $I$.
\end{enumerate}

\subsection{Résolution pratique}

\subsubsection{Principe de superposition}

\Para{Proposition}

Étant donné une équation linéaire $\eq{E}$,
la solution générale de $\eq{E}$ est donnée
par la somme de la solution générale de l'équation homogène $\eq{E_0}$
et d'une solution particulière de l'équation complète $\eq{E}$.

Plus précisément, soit $x_p$ une solution particulière de $\eq{E}$ sur $J$.
Alors toute solution $x$ de $\eq{E}$ sur $J$ est de la forme
\[x = x_0 + x_p,\]
où $x_0$ est une solution de $\eq{E}$ sur $J$.

Pour résoudre $\eq{E}$,
il suffit donc de résoudre $\eq{E_0}$
et de trouver une solution particulière de $\eq{E}$.

\Para{Remarque}

Si $\eq{E}$ est de la forme
\[\mtag{E} a(t) x''(t) + b(t)x'(t) + c(t)x(t) = d_1(t) + d_2(t),\]
pour trouver une solution particulière de $\eq{E}$,
il suffit de
\begin{itemize}
\item trouver une solution particulière $x_{p,1}$ de
  \[\mtag{E_1} a(t) x''(t) + b(t) x'(t) + c(t) x(t) = d_1(t),\]
\item trouver une solution particulière $x_{p,2}$ de
  \[\mtag{E_2} a(t) x''(t) + b(t) x'(t) + c(t) x(t) = d_2(t),\]
\item poser $x_p = x_{p,1} + x_{p,2}$.
\end{itemize}

Dans ces conditions, $x_p$ est une solution particulière de $\eq{E}$.

\subsubsection{Résolution de l'équation homogène}

\Para{Proposition}[cas des coefficients constants]

Soit l'équation différentielle
\[\mtag{E} ax''(t) + bx'(t) + cx(t) = 0,\]
où $(a,b,c)∈𝕂^3$, $a≠0$.

On forme l'\emph{équation caractéristique} de discriminant $Δ= b^2-4ac$
\[\mtag{E_c} ar^2 + br + c = 0.\]

Alors, dans le cas $𝕂=ℝ$,
\begin{itemize}
\item Si $Δ> 0$, la solution générale est donnée par
  \[x(t) = A \me^{r_1t} + B \me^{r_2t},\]
  où $(A,B)∈ℝ^2$ et $r_1$ et $r_2$ sont les racines distinctes de $\eq{E_c}$.
\item Si $Δ= 0$, la solution générale est donnée par
  \[x(t) = (A t+B) \me^{r_0t},\]
  où $(A,B)∈ℝ^2$ et $r_0$ est la racine double de $\eq{E_c}$.
\item Si $Δ< 0$, la solution générale est donnée par
  \[x(t) = \me^{αt} \Bigl( A \cos(ωt) + B \sin(ωt) \Bigr),\]
  où $(A,B)∈ℝ^2$ et $α±iω$ sont les
  racines complexes conjuguées de $\eq{E_c}$.
\end{itemize}

Alors, dans le cas $𝕂=ℂ$,
\begin{itemize}
\item Si $Δ≠0$, la solution générale est donnée par
  \[x(t) = A \me^{r_1 t} + B \me^{r_2 t},\]
  où $(A,B)∈ℂ^2$ et $r_1$ et $r_2$ sont les racines distinctes de $\eq{E_c}$.
\item Si $Δ= 0$, la solution générale est donnée par
  \[x(t) = (A t+B) \me^{r_0 t},\]
  où $(A,B)∈ℂ^2$ et $r_0$ est la racine double de $\eq{E_c}$.
\end{itemize}

\Para{Remarque}[cas général]

\emph{Il n'existe pas de méthode générale!}
\begin{itemize}
\item Dans le cas d'une équation à coefficients constants, on a des formules.
\item On peut tenter de chercher une solution sous la forme
  d'un monôme, d'un polynôme, d'une série entière, etc...
\item L'énoncé peut suggérer un changement de variable.
\item Si on a trouvé une solution $f$ qui ne s'annule pas,
  on peut chercher les autres solutions sous la forme $x(t) = f(t) y(t)$
  où $y$ est la nouvelle fonction inconnue.
\end{itemize}

et c'est à peu près tout.

\subsubsection{Recherche d'une solution particulière}

\Para{Proposition}[cas d'une équation à coefficients constants]

On considère l'équation différentielle
\[\mtag{E} ax''(t) + bx'(t) + cx(t) = P(t) \me^{αt},\]
où $a≠0$ et $P∈𝕂_d[X]$.
On note $\eq{E_c}$ l'équation caractéristique.

Il existe une unique solution particulière de la forme
\begin{enumerate}
\item Si $α$ n'est pas racine de $\eq{E_c}$,
  on prend $x_p(t) = Q(t) \me^{αt}$ où $Q∈𝕂_d[X]$ à déterminer.
\item Si $α$ est racine simple de $\eq{E_c}$,
  on prend $x_p(t) = tQ(t) \me^{αt}$ où $Q∈𝕂_d[X]$ à déterminer.
\item Si $α$ est racine double de $\eq{E_c}$,
  on prend $x_p(t) = t^2Q(t) \me^{αt}$ où $Q∈𝕂_d[X]$ à déterminer.
\end{enumerate}

Bien sûr, si le second membre est une somme de termes de ce type,
il suffit de superposer les solutions particulières correspondantes.

\Para{Théorème}[Hors-programme, méthode de variation des constantes]

On considère l'équation différentielle
\[\mtag{E} x''(t) + b(t)x'(t) + c(t)x(t) = d(t)\]
et l'on note $\eq{E_0}$ l'équation homogène associée.
On suppose que $(φ,ψ)$ est un système fondamental de solutions de $\eq{E_0}$.

Alors la solution générale de $\eq{E}$ est donnée par
\[x(t) =λ(t)φ(t) +μ(t)ψ(t),\]
où $λ$ et $μ$ sont solutions du système de Cramer
\[\mtag{S} \left\{ \begin{aligned}
  λ'(t)φ(t)  + μ'(t)ψ(t)  &= 0 \\
  λ'(t)φ'(t) + μ'(t)ψ'(t) &= d(t).
\end{aligned} \right.\]

% -----------------------------------------------------------------------------
\section{Notions sur les équations différentielles non linéaires (hors-programme)}

\Para{Contexte}

Les équations différentielles non linéaires sont bien plus générales
et bien plus complexes que les équations différentielles linéaires.

Par exemple, les équations différentielles scalaires du premier ordre
sous forme résolue sont les équations de la forme
\[x'(t) = f(t, x(t)),\]
où $f$ est une fonction de deux variables, définie sur un ouvert $Ω$
de $ℝ^2$.

La théorie générale n'est pas au programme, nous traiterons quelques cas particuliers.

\subsection{Équations à variables séparables}

\Para{Exemple}

On considère l'équation différentielle
\[\mtag{E} (1+t^2)x'(t) - 1 - x^2(t) = 0.\]
À cause du terme en $x^2$, il ne s'agit pas d'une équation linéaire.
Celle-ci n'est pas trop méchante, car on peut \og{}séparer les variables\fg{}:
\[\frac{x'}{1+x^2} = \frac{1}{1+t^2},\]
ce qui s'intègre en
\[\mtag{E'} \arctan x(t) = \arctan t + C.\]
Avec un peu de travail, on obtient les solutions suivantes:
\begin{itemize}
\item $t \mapsto \frac{kt+1}{k-t}$
  sur $\intO{-∞,k}$ et sur $\intO{k,+∞}$ où $k∈ℝ$
\item $t \mapsto t$ sur $ℝ$
\end{itemize}

\Para{Remarque}

Au lieu de l'équation
\[\frac{x'}{1+x^2} = \frac{1}{1+t^2},\]
on écrit parfois, de façon un peu abusive,
\[\frac{\D x}{1+x^2} = \frac{\D t}{1+t^2},\]
ce qui rend encore plus visible encore l'aspect
\og{}à variables séparées\fg{}.
On résout en
\[∫\frac{\D x}{1+x^2} =∫\frac{\D t}{1+t^2}.\]

\Para{Définition}

Une équation différentielle scalaire est dite
\og{}à variables séparables\fg{} si elle est de la forme
\[a(x(t))x'(t) = b(t),\]
où $a$ et $b$ sont deux fonctions numériques continues.

\Para{Proposition}

On considère l'équation différentielle
\[\mtag{E} a(x(t))x'(t) = b(t),\]
où $a$ et $b$ sont continues.
Notons $A$ (resp. $B$) une primtive de $a$ (resp. $b$).

L'équation $\eq{E}$ s'intègre en
\[A(x(t)) = B(t) + C,\]
où $C$ est une constante.

On ne peut pas toujours exprimer $x(t)$ en fonction de $t$,
mais on peut généralement tracer les courbes intégrales.

\Para{Exemple}

Déterminer les courbes intégrales de l'équation
\[xx' + t = 0.\]

\subsection{Systèmes autonomes}

\Para{Définition}

Un \emph{système autonome} de deux équations différentielles du premier ordre
est un système différentiel de la forme
\[\mtag{S} \left\{ \begin{aligned}
  \frac{\D x}{\D t} &=φ(x,y) \\
  \frac{\D y}{\D t} &=ψ(x,y)
\end{aligned} \right.\]
Ce système est qualifié d'\emph{autonome}
car $φ$ et $ψ$ ne dépendent pas de $t$.

\Para{Proposition}

Avec les mêmes notations, on suppose de plus que $φ$ et $ψ$ sont
de classe $\CC1$ sur un ouvert $Ω$ de $ℝ^2$ et que $φ$ ne s'annule pas sur $Ω$.
Alors on peut exprimer $y$ en fonction de $x$ en remarquant que
\[\frac{\D y}{\D x} = \frac{ψ(x,y)}{φ(x,y)}.\]
La courbe reliant $x$ et $y$ s'appelle \emph{courbe intégrale}
du système différentiel~$\eq{S}$.

On dit parfois aussi qu'il s'agit de la courbe intégrale
du champ de vecteur
\[(x,y) \mapsto \bigl(φ(x,y),ψ(x,y) \bigr).\]

\Para{Exemple}[pendule simple]

L'équation du pendule est donnée par (après normalisation)
\[\frac{\mathrm{d}^2 x}{\mathrm{d}t^2} + \sin x = 0.\]
En notant $v = \frac{\D x}{\D t}$, on obtient le système autonome en $(x,v)$
\[\left\{ \begin{aligned}
  \frac{\D x}{\D t} &= v \\
  \frac{\D v}{\D t} &= -\sin x
\end{aligned} \right.\]
On en déduit
\[\frac{\D v}{\D x} = -\frac{\sin x}{v},\]
ce qui est une équation différentielle non linéaire.

Pour la résoudre, il faut une petite astuce
\[v\frac{\D v}{\D x} = -\sin x\]
\[\frac12 v^2 = \cos x + C^\mathrm{ste}\]

Au passage, nous retrouvons l'énergie du système
\[E = \frac12 v^2 - \cos x = C.\]

Avec cela, nous pouvons tracer les courbes intégrales.
\[v = ±√{2\cos x + C}.\]

% -----------------------------------------------------------------------------
\section{Exercices}

\Exercice

Résoudre les équations différentielles suivantes:
\begin{enumerate}
\item $(1+x^2)y'+4xy=0$
\item $x^n y' -αy = 0$ où $(n,α)∈\Ns×\Rps$.
\item $y'+2xy-e^{x-x^2} = 0$
\item $x(x^2+1)y' + y + x = 0$
\item $xy'-y + \ln x = 0$
\item $x^3\ln\Abs{x}y' - x^2y - (2\ln\Abs{x}+1)=0$
\item $(\sin x)y'-y+1=0$
\item $(\sh x \ch^3 x)y' + 3(\ch^4 x)y - 1 = 0$
\end{enumerate}

\Exercice

Résoudre les équations différentielles suivantes:
\begin{enumerate}
\item $y''-2y'+y = 2\sh x$
\item $2y''+2y'+y = xe^{-x}$
\item $y''-4y = 4e^{-2x}$
\item $y''-3y'+2y = (x^2+1)e^x$
\item $y''+2y'+y = \sin^2 x$
\item $y''+2y'+2y = \ch x \cos x$
\item $y''+y = \Abs{x}+1$
\item $y''-2y'+y = e^{\Abs x}$
\end{enumerate}

\Exercice

Résoudre les équations différentielles suivantes:
\begin{enumerate}
\item $(2t+1)x'' + (4t-2)x' - 8x = 0$
\item $x'' + (4e^t-1)x' + 4e^{2t}x = 0$
\item $(t^2+1)^2x'' + 2t(t^2+1)x' + x = (t^2+1)^2$
\item $\begin{vmatrix} x'' & x' & x  \\  -\sin t & \cos t & \sin t  \\  -4\sin 2t &  2 \cos 2t &  \sin 2t \end{vmatrix} = 0$
\item $y''+y = \Abs{x^2-π^2}$ avec $y(0)=y'(0) = 0$
\end{enumerate}

\Exercice
\begin{enumerate}
\item Trouver toutes les applications $\Fn fℝℝ$ dérivables telles que:
  \[∀x∈ℝ\+ f'(x) f(-x) = 1\]
\item Trouver toutes les applications $\Fn fℝℝ$ dérivables telles que:
  \[∀x∈ℝ\+ f'(x) + f(-x) = e^x\]
\item Trouver toutes les applications $\Fn fℝℝ$ dérivables en $0$ telles que:
  \[∀x∈ℝ\+∀y∈ℝ\+ f(x+y) = e^x f(y) + e^y f(x)\]
\end{enumerate}

\Exercice

Soit $f∈\mathcal{C}(ℝ,ℝ)$ ayant une limite finie en $+∞$.
Montrer que toute solution de l'équation différentielle $y'+y = f$
admet une limite finie en $+∞$.

\Exercice

Soit l'équation différentielle \[x^2 y'' + 4xy' + (2-x^2)y = 1.\]
Résoudre cette équation différentielle sur $ℝ_+^*$ et sur $ℝ_-^*$
en posant $u = x^2 y$, puis étudier le raccordement en~$0$.

\Exercice

Calculer explicitement les fonctions suivantes:
\[\begin{aligned}
u(x) &= ∫_0^{+∞} \frac{e^{-t} \cos(xt)}{√t} \D t \\
v(x) &= ∫_0^{+∞} \frac{e^{-t} \sin(xt)}{√t} \D t
\end{aligned}\]

\Exercice

Résoudre le système différentiel:
\[\left\{
\begin{aligned}
  x' &= 5x  -   y  -  2z  +  e^t    \\
  y' &=  x  +  3y  -  2z  +  e^{2t} \\
  z' &= -x  -   y  +  4z  +  t e^t
\end{aligned}
\right.\]

\Exercice

Soit $\Fn fℝℝ$ de classe $\CC2$ telle que
\[∀x∈ℝ\+ f''(x) + f(x)≥0.\]
Montrer que $∀x∈ℝ\+ f(x) + f(x+π)≥0$.
On pourra introduire
\[W(x) = \begin{vmatrix} f(x) & \sin(x) \\ f'(x) & \cos(x) \end{vmatrix}.\]

\Exercice

Soit $\Fn Aℝℝ$ continue et bornée et $k > 0$.
Montrer que l'équation différentielle
\[y' - ky = A\]
admet une unique solution bornée.

\Exercice

Soit $\Fn f\Rpℝ$ de classe $\CC1$ telle que
\[\lim_{x \to +∞} \Bigl( f(x) + f'(x) \Bigr) = 0.\]
Que dire de $\lim\limits_{x \to +∞} f(x)$?

\Exercice

Montrer que l'équation \[x'' - 2x' + x = \frac{1}{√{πt}}\]
possède une unique solution $x$ définie sur $ℝ_+^*$ telle que,
par prolongement $x(0) = x'(0) = 0$.

\Exercice

Résoudre l'équation \[(-x^2+4x)f'(x) + (x+2)f(x) = x.\]
On pourra commencer par rechercher les solutions
développables en série entière au voisinage de 0.

\Exercice

Soit $\Fn{f}{[a,b]}{ℝ}$ non nulle de classe $\CC2$ et solution de l'équation différentielle
$y'' + py' + qy = 0$, où $p$ et $q$ sont continues de $[a,b]$ dans $ℝ$.
Montrer que $f$ admet un nombre fini de zéros.

\Exercice

Soit $I$ un intervalle de $ℝ$, $r$ et $s$ deux fonctions continues $I \toℝ$ telles que $r≤s$.
Soit $x$ une solution non nulle de l'équation différentielle $x'' + rx = 0$,
et $y$ une solution non nulle de l'équation différentielle $y'' + sy = 0$.
Soit $t_1$ et $t_2$ deux zéros consécutifs de $x$.
Montrer que $y$ s'annule sur $\intO{t_1,t_2}$ sauf si $x$ et $y$ sont proportionnelles.

\Exercice[applications de l'exercice précédent]

Soit $\Fn rIℝ$ une fonction continue,
et $x$ une solution non nulle de l'équation différentielle
\[x'' + rx = 0.\]
\begin{enumerate}
\item On suppose $∀t∈I, r(t)≤0$.
  Montrer que $x$ s'annule au plus une fois dans $I$.
\item Soit $μ> 0$. On suppose $∀t∈I, r(t)≤μ^2$.
  Montrer que $t_2≥t_1 + \frac{π}{μ}$.
\item Soit $λ> 0$. On suppose $∀t∈I, r(t)≥λ^2$.
  Soit $t_1∈I$ telle que $t_1 + \frac{π}{λ}∈I$.

  Montrer que toute solution de l'équation différentielle $x'' + rx = 0$
  s'annule au moins une fois sur $\intO{t_1,t_1+\frac{π}{λ}}$.
\end{enumerate}

\Exercice

Que peut-on dire des zéros des solutions sur $ℝ_+^*$ de l'équation différentielle suivante?
\[x'' + \frac{x'}{t} + \left( 1 - \frac{α^2}{t^2} \right) x = 0\]

\Exercice[lemme de Gronwall]

Soit $\Fn{a}{\Rp}{ℝ}$ une fonction continue et
$\Fn{f}{ℝ}{ℝ}$ de classe $\CC1$ telle que
\[∀x≥0 \+ f'(x)≤a(x) f(x).\]

Montrer que
\[∀x≥0 \+ f(x)≤f(0) \exp\left( ∫_0^x a(t) \D t \right).\]

\end{document}
