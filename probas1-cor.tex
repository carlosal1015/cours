% Time-stamp: <2017-10-29 22:05:19 yann>
\documentclass{yann}

\begin{document}
\title{Probabilités finies: corrections}
\maketitle

\setcounter{ExoNum}{20}

\Exercice

Notons $B_n$ l'événement \enquote{la $n$-ième boule tirée est blanche}.
Il est clair que $ℙ(B_1) = \frac{a}{a+b}$.

Comme $(B_1, \overline{B_1})$ est un système complet d'événements,
on a par la formule des probabilités totales
\[ ℙ(B_2) = ℙ(B_2 \mid B_1) ℙ(B_1) + ℙ(B_2 \mid \overline{B_1}) ℙ(\overline{B_1}). \]
On se convainc facilement que $ℙ(B_2 \mid B_1) = \frac{a+c}{a+b+c}$
et de même que $ℙ(B_2 \mid \overline{B_1}) = \frac{a}{a+b+c}$,
on réinjecte et on trouve $ℙ(B_2) = ℙ(B_1)$.

Pour $ℙ(B_3)$ on peut procéder de même, en utilisant le système complet d'événements $\bigPa{B_1∩B_2, B_1∩\overline{B_2}, \overline{B_1}∩B_2, \overline{B_1}∩\overline{B_2}}$.
On trouve
\begin{align*}
  ℙ(B_3) &= \frac{a+2c}{a+b+2c} ⋅ \frac{a}{a+b} ⋅ \frac{a+c}{a+b+c} \\
  &\quad + \frac{a+c}{a+b+2c} ⋅ \frac{a}{a+b} ⋅ \frac{b}{a+b+c} \\
  &\quad + \frac{a+c}{a+b+2c} ⋅ \frac{b}{a+b} ⋅ \frac{a}{a+b+c} \\
  &\quad + \frac{a}{a+b+2c} ⋅ \frac{b}{a+b} ⋅ \frac{b+c}{a+b+c} \\
  &= \frac{a}{(a+b)(a+b+c)(a+b+2c)} × \\
  & \qquad \BigCro{ (a+2c)(a+c) + 2(a+c)b + b(b+c) } \\
  &= ℙ(B_1)
\end{align*}

car $ (a+2c)(a+c) + 2(a+c)b + b(b+c) = (a+b+c)(a+b+2c) $.

On peut conjecturer que $ℙ(B_n) = ℙ(B_1)$, mais la méthode directe pour le prouver ne semble pas très agréable.

L'astuce est de le faire par récurrence et d'utiliser la formule des probabilités totales avec le système complet d'événements $(B_1, \overline{B_1})$.
Ainsi,
\[ ℙ(B_{n+1}) = ℙ(B_{n+1} \mid B_1) ℙ(B_1) + ℙ(B_{n+1} \mid \overline{B_1}) ℙ(\overline{B_1}). \]
Il ne reste plus qu'à remarquer qu'on peut déterminer, \emph{sans calcul}, les deux probabilités conditionnelles ci-dessus et conclure par récurrence.

\end{document}
