\documentclass{yann}
\usepackage{tcolorbox}

\yann{layout=onecolumn}
\newcommand\Exo[1]{\paragraph{Exercice #1}}

\begin{document}
\title{Quantificateurs: corrigés}
\maketitle

\Exo{3}

Toutes ces phrases ressemblent à la définition de la continuité que l'on notera~(0).
Pour plus de clarté, on remplacera
\enquote{$y$} par \enquote{$ε$} et
\enquote{$z$} par \enquote{$η$}.

\setcounter{equation}{-1}
\begin{gather}
  ∀x∈ℝ \+ ∀ε>0 \+ ∃η>0 \+ ∀t∈ℝ \+ \abs{x-t}≤η \implies \abs{f(x)-f(t)}≤ε \\
  ∀x∈ℝ \+ ∀ε>0 \+ ∃η>0 \+ ∀t∈ℝ \+ \abs{x-t}<η \implies \abs{f(x)-f(t)}<ε \\
  ∀x∈ℝ \+ ∀ε≥0 \+ ∃η>0 \+ ∀t∈ℝ \+ \abs{x-t}<η \implies \abs{f(x)-f(t)}<ε \\
  ∀x∈ℝ \+ ∀ε>0 \+ ∃η≥0 \+ ∀t∈ℝ \+ \abs{x-t}<η \implies \abs{f(x)-f(t)}<ε \\
  ∀x∈ℝ \+ ∀ε>0 \+ ∃η>0 \+ ∀t∈ℝ \+ \abs{x-t}≤η \implies \abs{f(x)-f(t)}<ε \\
  ∀x∈ℝ \+ ∀ε>0 \+ ∃η>0 \+ ∀t∈ℝ \+ \abs{x-t}<η \implies \abs{f(x)-f(t)}≤ε \\
  ∀x∈ℝ \+ ∀ε≥0 \+ ∃η>0 \+ ∀t∈ℝ \+ \abs{x-t}<η \implies \abs{f(x)-f(t)}≤ε
\end{gather}

\begin{itemize}

\item

La phrase~(2) n'est jamais vérifiée. En effet, supposons par l'absurde que $f$ vérifie (2).
On choisit $x = ε = 0$, d'après l'énoncé il existe $η>0$ tel que
\[ ∀t ∈ℝ\+ \abs{x-t} < η \implies \abs{f(x)-f(t)} < ε. \]
Pour $t = 0$, on a bien $\abs{x-t} = 0 < η$, mais $\abs{f(x)-f(t)} = 0$ qui n'est pas strictement inférieur à $ε$, ce qui est absurde.

\item

La phrase~(3) est toujours vérifiée. En effet, soit $x ∈ℝ$ et $ε> 0$.
On choisit $η= 0$, et on doit montrer que
\[ ∀t ∈ℝ\+ \abs{x-t} < η \implies \abs{f(x)-f(t)} < ε. \]
Comme la partie gauche de l'implication est toujours fausse, l'implication elle-même est toujours vraie, d'où le résultat.

\item

Montrons que (0), (1), (4) et~(5) sont équivalents.
Pour cela, on va montrer que $(4) \implies (0)$ et $(0) \implies (5)$ sont triviaux,
que $(5) \implies (1)$ s'obtient avec $ε/2$
et $(1) \implies (4)$ avec $η/2$.

\medskip

\begin{itemize}

\item[$(4) \Rightarrow (0)$]

Soit $f$ vérifiant la phrase~(4).
Montrons que $f$ vérifie la phrase~(0).
Soit $x ∈ℝ$, $ε> 0$.
D'après l'hypothèse, il existe $η> 0$ tel que
\[ ∀t ∈ℝ\+ \abs{x-t} ≤η \implies \abs{f(x)-f(t)} < ε. \]

Soit $t ∈ℝ$ tel que $\abs{x-t} ≤η$. On a donc $\abs{f(x)-f(t)} < ε$
et donc a fortiori $\abs{f(x)-f(t)}≤ε$
donc $f$ vérifie bien la phrase~(0).

\item[$(0) \Rightarrow (5)$]

Soit $f$ vérifiant la phrase~(0).
Soit $x ∈ℝ$, $ε> 0$.
D'après l'hypothèse, il existe $η> 0$ tel que
\[ ∀t ∈ℝ\+ \abs{x-t} ≤η \implies \abs{f(x)-f(t)}≤ε. \]

Soit $t ∈ℝ$ tel que $\abs{x-t} < η$.
On peut appliquer l'hypothèse,
d'où $\abs{f(x)-f(t)}≤ε$
donc $f$ vérifie bien la phrase~(5).

\item[$(5) \Rightarrow (1)$]

Soit $f$ vérifiant la phrase~(5).
Soit $x ∈ℝ$, $ε> 0$.
On applique l'hypothèse avec $x$ et $ε/2$, donc il existe $η> 0$ tel que
\[ ∀t ∈ℝ\+ \abs{x-t} < η \implies \abs{f(x)-f(t)}≤ε/2. \]

Soit $t ∈ℝ$ tel que $\abs{x-t} < η$.
On peut appliquer l'hypothèse,
d'où $\abs{f(x)-f(t)}≤ε/2 < ε$
donc $f$ vérifie bien la phrase~(1).

\item[$(1) \Rightarrow (4)$]

Soit $f$ vérifiant la phrase~(1).
Soit $x ∈ℝ$, $ε> 0$.
On applique l'hypothèse, donc il existe $η' > 0$ tel que
\[ ∀t ∈ℝ\+ \abs{x-t} < η' \implies \abs{f(x)-f(t)} < ε. \]

On pose $η= η'/2$. Vérifions que
\[ ∀t ∈ℝ\+ \abs{x-t} ≤η \implies \abs{f(x)-f(t)} < ε. \]
Soit $t ∈ℝ$ tel que $\abs{x-t} ≤η$.
On a donc $\abs{x-t} < η'$,
d'où $\abs{f(x)-f(t)} < ε$
donc $f$ vérifie bien la phrase~(4).

\end{itemize}

Ainsi, les phrases (1), (4) et (5) signifient que $f$ est continue sur $ℝ$

\item

La phrase~(6) est sensiblement plus délicate. Montrons que $f$ vérifie (6) £ssi. $f$ est une fonction constante sur $ℝ$. D'abord, remarquons que si $f$ est constante, elle vérifie immédiatement (6). Réciproquement, soit $f$ vérifiant~(6).
Soit $a < b$ deux réels. Montrons que $f(a) = f(b)$.

Pour tout $x ∈ℝ$, on applique l'hypothèse~(6) avec $ε=0$;
il existe donc $η_x > 0$ tel que
\[ ∀t ∈ℝ\+ \abs{x-t} < η_x \implies \abs{f(x)-f(t)}≤ε. \]
Ainsi, si $t ∈\intO{x-η_x,x+η_x}$, on a donc $\abs{f(x)-f(t)}≤ε= 0$, donc $f(t) = f(x)$.
Autrement dit, \tcbox{$f$ est constante sur tous les intervalles de la forme $\intO{x-η_x,x+η_x}$.}

Posons $E = \Ensemble{x ∈ℝ}{x ≥a \text{ et $f$ constante sur l'intervalle $\intFO{a,x}$} }$.
L'ensemble $E$ est une partie de $ℝ$, non vide car $η_a ∈E$.

Supposons par l'absurde que $E$ est majoré. Il admet alors une borne supérieure $α$. Par définition, $α-η_α$ n'est pas un majorant de $E$, donc il existe un $x ∈E$ tel que $x > α- η_α$.
On sait alors que la fonction $f$ est constante sur $\intFO{a,x}$, mais aussi sur $\intO{α-η_α,α+η_α}$. Ces deux intervalles n'étant pas disjoints, $f$ est constante sur l'union des deux: $\intFO{a,α+η_α}$.
Ainsi, $α+η_α∈E$ ce qui est absurde car $\sup E = α< α+ η_α$.

Ainsi, $E$ n'est pas majoré. Il existe donc un $x > b$ tel que $x ∈E$.
Comme $f$ est alors constante sur $\intFO{a,x}$, on a $f(a) = f(b)$ pour tous $(a,b) \in \R^2$ tels que $a < b$.
Bref, la fonction $f$ est bien constante.

\medskip

On a ainsi prouvé que la phrase~(6) est vérifiée £ssi. $f$ est constante.

\end{itemize}

\end{document}
